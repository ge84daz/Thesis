
\documentclass[ 
   12pt,                  % Schriftgroesse 12pt
   a4paper,               % Layout fuer Din A4
   twoside,               % Layout fuer beidseitigen Druck
   headinclude,           % Kopfzeile wird Seiten-Layouts mit beruecksichtigt
   headsepline,           % horizontale Linie unter Kolumnentitel
   plainheadsepline,      % horizontale Linie auch beim plain-Style
   BCOR=12mm,             % Korrektur fuer die Bindung
   DIV=18,                % DIV-Wert fuer die Erstellung des Satzspiegels, siehe scrguide
   parskip=half,          % Absatzabstand statt Absatzeinzug
   openany,               % Kapitel können auf geraden und ungeraden Seiten beginnen
   bibliography=totoc,    % Literaturverz. wird ins Inhaltsverzeichnis eingetragen
   toc=listof,            % Abbildungs- Tabellen- und sonstige verzeichnisse mit ins INhaltsverz.
   numbers=noenddot,      % Kapitelnummern immer ohne Punkt
   captions=tableheading, % korrekte Abstaende bei TabellenUEBERschriften
   fleqn,                 % fleqn: Gleichungen links (statt mittig)
   % draft                % Keine Bilder in der Anzeige, overfull hboxes werden angezeigt
   % To be edited by the author:
   % german,              % deutsche Sprache, global
   % bmcolorlinks,
   % bmshowlabels
   % mpinclude,           % mehr Platz für Randnotzzen
      ]{bmvorlagen/bmthesis}

%%%%%%%%%%%%%%%%%%%%%%%%%%%%%%%%%%%
% Packages/Settings
%%%%%%%%%%%%%%%%%%%%%%%%%%%%%%%%%%%
\usepackage[utf8]{inputenc}  					 % Input-Encodung: ansinew fuer Windows
\usepackage[T1]{fontenc}
%\usepackage[latin1]{inputenc}    				 % Input-Encodung: latin1 fuer Unix
\newcommand{\bmtitle}{Musterthesis ...}
\newcommand{\bmauthor}{B.Sc. ...}
\newcommand{\bmkeywords}{Thesis, Example}
\newcommand{\bmstartpage}{5}
\graphicspath{{bilder/}{bilder_svg/}{plots/}}    % Falls texinput nicht gesetzt -> Bildverzeichnisse
\hyphenation{Post-pro-cess-ing--In-te-gral}
\raggedbottom

%%%%%%%%%%%%%%%%%%%%%%%%%%%%%%%%%%%
% Setting for new version of svg-Package
% to handle file names as the old one did
% fixing a bug
% date 04.11.2019
\makeatletter
\def\set@curr@file#1{\def\@curr@file{#1}}
\makeatother

%%%%%%%%%%%%%%%%%%%%%%%%%%%%%%%%%%%

%%%%%%%%%%%%%%%%%%%%%%%%%%%%%%%%%%%
% Dokument an sich
%%%%%%%%%%%%%%%%%%%%%%%%%%%%%%%%%%%
% Wenn man nur ein kapitel übersetzten möchte, da reinschreiben
%\includeonly{}
%%%%%%%%%%%%%%%%%%%%%%%%%%%%%%%%%%%%%%%%%%%%%%%%%%%%%%%%%%%%%%%




\begin{document}
\frontmatter
\pagenumbering{Roman}

\ifthenelse{\boolean{bmgerman}}
{
\begin{titlepage}
%\layout

\begin{center}
{
\fontsize{18}{18}\selectfont 

\includegraphics*[width=3cm, keepaspectratio=true]{bmvorlagen/logos/tumlogo} \hfill \includegraphics*[width=2.5cm, keepaspectratio=true]{bmvorlagen/logos/bmlogo}
\vspace{0.5cm}
\hrule

\vspace{1cm}
Lehrstuhl für Baumechanik\\
Technische Universität München\\

\vspace{1.4cm}

\vspace{2cm}


\vspace{3mm}



\bmtitle \\[5ex]


\bmauthor \\[7ex]


Masterarbeit im Studiengang Bauingenieurwesen\\
Vertiefungsrichtung Baumechanik\\

\vspace{1.8cm}

{\fontsize{12pt}{12} \selectfont%
\begin{tabular}{rll}
Referent&:& Univ.-Prof. Dr.-Ing. Gerhard Müller\\
& & Dr.-Ing. Francesca Taddei\\[0.5ex]
Betreuer&:& M.Sc. XXX\\[0.5ex]
Eingereicht&:& \today
\end{tabular}
}
                  

}
\end{center}

\end{titlepage}
}
{
\begin{titlepage}
	%\layout
	
	\begin{center}
		{
			\fontsize{18}{18}\selectfont 
			
			\includegraphics*[width=3cm, keepaspectratio=true]{bmvorlagen/logos/tumlogo} \hfill \includegraphics*[width=2.5cm, keepaspectratio=true]{bmvorlagen/logos/bmlogo}
			\vspace{0.5cm}
			\hrule
			
			\vspace{1cm}
			Chair of Structural Mechanics\\
			Technical University of Munich\\
			
			\vspace{1.4cm}
			
			\vspace{2cm}
			
			
			\vspace{3mm}
			
			
			
			\bmtitle \\[5ex]
			
			
			\bmauthor \\[7ex]
			
			
			Masterthesis in the study program civil engineering\\
			Focus area Structural Dynamics\\
			
			\vspace{1.8cm}
			
			{\fontsize{12pt}{12} \selectfont%
				\begin{tabular}{rll}
					Editor&:& Univ.-Prof. Dr.-Ing. Gerhard Müller\\
					& & Dr.-Ing. Francesca Taddei\\[0.5ex]
					Supervisor&:& M.Sc. XXX\\[0.5ex]
					Handed in&:& \today
				\end{tabular}
			}
			
			
		}
	\end{center}
	
\end{titlepage}
}

                % Deckblatt
\cleardoubleemptypage 
\addchap{Abstract}
\label{cha:abtract} 

An abstract is a greatly condensed version of a longer piece of writing that highlights the major points covered, and concisely describes the content and scope of the writing. 

Abstracts give readers a chance to quickly see what the main contents and sometimes methods of a piece of writing are. They enable readers to decide whether the work is of interest for them. Using key words in an abstract is important because of today's electronic information systems. A web search will find an abstract containing certain key words.  

Keywords: 
\begin{itemize}
 \item Keyword 1
 \item Keyword 2
\end{itemize}
            % Abstract der Arbeit
\addchap{Erklärungen}
\markboth{Erklärungen}{Erklärungen}
\label{cha:erkl}

\subsubsection{Danksagung}
An erster Stelle möchte ich mich bei Herrn Professor Dr.-Ing. Gerhard Müller sowie dem gesamten Lehrstuhl für Baumechanik dafür bedanken, dass ich meine Bachelorarbeit dort schreiben durfte. 

Bei meinem Betreuer Tom Hicks möchte ich mich ebenfalls bedanken

\subsubsection{Erklärung}
Hiermit versichere ich, die vorliegende Arbeit selbstständig und ohne fremde Hilfe angefertigt zu haben. Die verwendete Literatur und sonstige Hilfsmittel sind vollständig angegeben.
\vfill
München, \today \\
\vspace{3cm} \\
Anton Bönisch
\vfill

\cleardoubleemptypage         % Das Inhaltsverzeichnis auf einer rechten Seite beginnen

\begin{spacing}{1.0}          % Verzeichnisse werden mit einzeiligem Abstand gesetzt
 \tableofcontents             % Inhaltsverzeichnis
% \addcontentsline{toc}{chapter}{Abbildungsverzeichnis} so kann man von hand was in Inhaltsverzeichnis schreiben
 \listoffigures              % Abbildungsverzeichnis
% \listoftables               % Tabellenverzeichnis
% \lstlistoflistings          % Verzeichnis der Listings
\end{spacing}

\addchap{Symbolverzeichnis}
\markboth{Symbolverzeichnis}{Symbolverzeichnis}
\label{cha:symbolverzeichnis}

Der Zusatz $\hat{\medspace}$ bezeichnet eine komplexe Größe.

\section*{Griechische Buchstaben}
\begin{longtable}[l]{lcp{8cm}l}
$\Gamma$ & & Berandung des Problems \\
$\delta\left(\left|\vec{x}_{0} - \vec{x}\right|\right)$ & & Dirac-Funktion am Punkt $\vec{x}$ \\
\end{longtable}

\section*{Lateinische Buchstaben}
\begin{longtable}[l]{lcp{8cm}l}
%\hspace*{2.5cm}\= \hspace*{1cm} \=  Schallschnelle senkrecht zu einer Oberfläche \= \kill
$c$ & \einheit{\frac{m}{s}} & Schallgeschwindigkeit & $c = \frac{E}{\rho}$ \\
$E$ & \einheit{\frac{N}{m^{2}}} & Elastizitätsmodul\\
$f$ & \einheit{\frac{1}{s}} & Frequenz & $f = \frac{\omega}{2 \pi}$ \\ 
\end{longtable}             % Symbolverzeichnis 

\cleardoubleplainpage         % Das erste Kapitel des Hauptteils auf einer rechten Seite beginnen

\mainmatter                   % den Hauptteil beginnen
\chapter{Einleitung}
\label{kap:einleitung}

\section{Ausgangssituation}
\label{sec:ausgang}

             % Einleitung
\chapter{Testchapter}
\label{cha:beispiele}

\section{How to ...}
\label{sec:chief}

\subsection{Citations}
\label{sec:citations}
The usual citation should look like \citep{Stroud1966}.

Additionally one can extend both directions \citep[before][after p.1-3]{Stroud1966}: 
\citep{Unknown2018}

\citep{Unknown2018}

\citep{Hackenberg2016}

\cite{Fruehe2010}


\subsection{Equations}
\label{sec:equations}
In the following different typs of equations ar shown:

Equation:
%
\begin{equation}
		\int\limits_{(\Gamma)} \left( i \rho \omega \hat{v}_{n} \hat{G}\right)~d\Gamma  = - C(\mathbf{P}) \cdot \hat{p}(\mathbf{P})  + \int\limits_{(\Gamma)} \left(\hat{p} \frac{\partial \hat{G}}{\partial \vec{n}}\right)~d\Gamma.
	\label{eq:chief1}
\end{equation}

Please take care that there is no empty line before the equation because \LaTeX will interpret this as a new paragraph. Whether or not there is a free line (paragraph) after an equation is up to the author.

Gather:
\begin{gather}
\left[(\lambda+2\mu)\phi|^j_j-\rho\ddot{\Phi}\right]|^i+\left[\mu\Psi_l|_j^j-\rho\ddot{\Psi}_l\right]|_k\epsilon^{ikl}=0
\end{gather}

Subequations:
\begin{subequations}\label{eq:Wellengleichung_FT}
\begin{align}
		 \left[-k_x^2 - k_y^2 + k_p^2 +\frac{\partial ^2}{\partial z^2}\right]\hat{\Phi} \;(k_x,k_y,z,\omega) &= 0 \label{eq:2.5}\\[10pt]
		 \left[-k_x^2 - k_y^2 + k_s^2 +\frac{\partial ^2}{\partial z^2}\right]\hat{\Psi}_i(k_x,k_y,z,\omega) &= 0 \label{eq:2.6}
\end{align}
\end{subequations}
mit Kompressionswellenzahl $k_p$ und Scherwellenzahl $k_s$.

Matrices with different brackets:
\begin{equation}
\begin{pmatrix}
\hat u_x \\
\hat u_y\\
\hat u_z
\end{pmatrix}
%
=
% 
\begin{bmatrix}
i k_x & 0 & - \frac{\partial}{\partial z} \\
i k_y & \frac{\partial}{\partial z} & 0\\
\frac{\partial}{\partial z} & -i k_y & i k_x
\end{bmatrix}
%
\begin{pmatrix}
\hat \Phi\\
\hat \Psi_x\\
\hat \Psi_y
\end{pmatrix}
%
\label{eq_027} 
\end{equation}

Align $\rightarrow$ multiple equations with numeration each:
\begin{align}
	\hat{\sigma}_{zx}(k_x,k_y,z=0,\omega) &=-\hat{p}_{zx}(k_x,k_y,\omega) \\[10pt]
		 \hat{\sigma}_{zy}(k_x,k_y,z=0,\omega) &=-\hat{p}_{zy}(k_x,k_y,\omega)
\end{align}

Aligned $\rightarrow$ multiple equations with one number:
\begin{equation}\label{eq:RB_OF}
	\begin{aligned}
		 \hat{\sigma}_{zx}(k_x,k_y,z=0,\omega) &=-\hat{p}_{zx}(k_x,k_y,\omega) \\[10pt]
		 \hat{\sigma}_{zy}(k_x,k_y,z=0,\omega) &=-\hat{p}_{zy}(k_x,k_y,\omega) \\[10pt]
		 \hat{\sigma}_{zz}(k_x,k_y,z=0,\omega) &=-\hat{p}_{zz}(k_x,k_y,\omega) 
		\end{aligned}
\end{equation}




\subsection{Self written functions for easy writing}
\label{sec:helpers}

\begin{tabbing}
\hspace*{4cm}\=\hspace*{3cm}\= \\
	 units: \> \einheit{kg} oder \einheit{\frac{kg}{m^2}} (works also in Math-environment)\\
	 birth and death: \> H. A. Schenck \lived{1901}{1970}\\
	 vectors \> $\mathbf{g}$ bold symbol for vector. Usable in text or math enviroment.\\
	 matrices \> $\mathbf{G}$ brackets around Matrix name. Usable in text or math enviroment.\\
\end{tabbing}

\subsection{Possible options in the documentclass bmthesis}

\begin{description}
	\item[bmcolorlinks:]  makes hyperlinks in the pdf colourful. Please skip / delete for the printed version of the theses.
	\item[bmshowlabels:] prints the labels. Please skip / delete for the printed version of the theses.
	\item[german:] used for german language. Has influence on hyphenation, titles and names, style of the bibliography... . For english thesis': skip it.
\end{description}


\section{Some more examples}
\label{sec:more_ex}
\subsection{Table}
\label{sec:table}
Eine einfache Tabelle:

\begin{table}[htb]
	\centering
		\begin{tabular}{ccc} \firsthline
	   &  $ka$ & f [Hz] \\\hline
		$1$ & $\pi$ & $171,5$ \\
		$2$ & $2\pi$ & $343$ \\
		$3$ & $3\pi$ & $514,5$ \\\lasthline
		\end{tabular}
	\caption[Die ersten drei symmetrischen Eigenwerte für das innere Problem des Kugelkörpers]{}
	\label{tab:tabelle1}
\end{table}

Die ersten Kugelflächenfunktionen lauten:
\begin{table}[H]
	\centering
	\begin{small}
	\renewcommand*{\arraystretch}{1.0}
		\begin{tabular}{l||l|l|l|l}
    $Y_m^l(\vartheta, \varphi)$ & $l=0$ & $l=1$ & $l=2$ & $l=3$\\
    \hline \hline
		$m=-3$ &  &  & & $\;\;\:\sqrt{\frac{35}{64 \pi}} \;\sin^{3}{\vartheta}\;e^{-3i \varphi}$\\
		\hline
    $m=-2$ &  &  & $\;\;\:\sqrt{\frac{15}{32\pi}}\;\sin^2{\vartheta}\;e^{-2i\varphi}$ & $\;\;\:\sqrt{\frac{105}{32\pi}} \;\sin^{2}{\vartheta}\cos{\vartheta}\;e^{-2i \varphi}$ \\
		\hline
		$m=-1$ &  & $\;\;\:\sqrt{\frac{3}{8\pi}}\;\sin{\vartheta}\;e^{-i\varphi}$  & $\;\;\:\sqrt{\frac{15}{8\pi}}\;\sin{\vartheta}\;\cos{\vartheta}\;e^{-i\varphi}$ & $\;\;\:\sqrt{\frac{21}{64 \pi}} \;\sin{\vartheta}\left( 5 \cos^{2}{\vartheta} - 1\right)\;e^{-i \varphi}$\\
		\hline
		$m=\;\;\:0$ & $\;\;\:\sqrt{\frac{1}{4\pi}}$ & $\;\;\:\sqrt{\frac{3}{4\pi}}\;\cos{\vartheta}$ & $\;\;\:\sqrt{\frac{5}{16\pi}}\;\left(3\cos^2{\vartheta}-1\right)$ & $\;\;\:\sqrt{\frac{7}{16 \pi}} \; \left( 5 \cos^{3}{\vartheta} - 3 \cos{\vartheta}\right)$\\
		\hline
		$m=\;\;\:1$ &  & $-\sqrt{\frac{3}{8\pi}}\;\sin{\vartheta}\;e^{i\varphi}$ & $-\sqrt{\frac{15}{8\pi}}\;\sin{\vartheta}\;\cos{\vartheta}\;e^{i\varphi}$ & $-\sqrt{\frac{21}{64\pi}} \sin{\vartheta}\left( 5 \cos^{2}{\vartheta} - 1\right)\;e^{i \varphi} $\\
		\hline
		$m=\;\;\:2$ &  &  & $\;\;\:\sqrt{\frac{15}{32\pi}}\;\sin^2{\vartheta}\;e^{2i\varphi}$ & $\;\;\:\sqrt{\frac{105}{32 \pi}}\; \sin^{2}{\vartheta}\cos{\vartheta}\;e^{2i \varphi}$\\
		\hline
		$m=-3$ &  &  & & $-\sqrt{\tfrac{35}{64 \pi}}\; \sin^{3}{\vartheta}\;e^{3i \varphi}$
		\end{tabular}
		\end{small}
	\caption{Kugelflächenfunktionen für $l=0, 1, 2, 3$ und zugehörige $l=-m, ..., m$}
	\label{tab:Kugelflächenfunktionen}
\end{table}

\subsection{Aufzählungen}

Die itemize Umgebung kann in sich selbst bis zu vier Ebenen tief geschachtelt werden. 

\begin{itemize}
\item erste Ebene
\begin{itemize}
\item zweite Ebene
\begin{itemize}
\item dritte Ebene
\begin{itemize}
\item vierte Ebene
\end{itemize}
\end{itemize}
\end{itemize}
\end{itemize}

Die Ausgabe der Label kann verändert werden. Am Anfang ein Beispiel für die Verwendung der Option des item Befehls. item[Option] Hier kann ein Label als Option eingestellt werden. 

\begin{itemize}
\item[a)] Ein Stichpunkt
\item[*)] Noch ein Stichpunkt
\end{itemize}

Die enumerate Umgebung in LATEX stellt eine nummerierte Auflistung zur Verfügung. 

\begin{enumerate}
\item erstes 
\item zweites 
\end{enumerate}

Standardmäßig erfolgt die Nummerierung auf der ersten Ebene mit arabischen Ziffern/Zahlen., auf der zweiten Ebene mit (kleiner lateinischer Buchstabe), auf der dritten Ebene mit kleinen römischen Ziffern/Zahlen. und auf der vierten Ebene mit großen lateinischen Buchstaben.. 

\begin{enumerate}
\item erste Ebene
\begin{enumerate}
\item zweite Ebene
\begin{enumerate}
\item dritte Ebene
\begin{enumerate}
\item vierte Ebene
\end{enumerate}
\item wieder auf dritter Ebene 
\item noch ein Eintrag 
\end{enumerate}
\item hier ist die zweite Ebene
\end{enumerate}
\item und hier die erste Ebene
\end{enumerate}



\section{Including graphics}
\label{sec:figs}

\subsection{Include SVGs}
\begin{figure}[H]
	\begin{center}
		\includesvg[width=0.9\textwidth]{svg/Helmholtz}
	\end{center}
	\caption{Zerlegung des Verschiebungsfeldes mit Satz von Helmholtz vlg. \citep{Mueller1989}}
	\label{abb:Satz_von_Helmholtz}
\end{figure}

\subsection{Include PNGs and PDFs}

\begin{figure}[h]
	\centering
		\includegraphics[width=0.6\textwidth]{bilder/Assoziierte_Legendre.pdf}
	\caption{Assoziierte Legendre Funktionen Pdf}
	\label{fig:Assoziierte_Legendre}
\end{figure}

\begin{figure}[H]
	\centering
		\includegraphics[width=0.8\textwidth]{bilder/Latex_logo.png}
	\label{fig:Latex_Logo}
	\caption{Latex Logo PNG}
\end{figure}

\subsection{Figures with subfigures}
Figures are defined as floating structures. See \citep[][page 91]{Sturm2010}.


\begin{figure}[H]
\centering%
\begin{subfigure}[c]{0.49\textwidth}
                \includegraphics[width=7.6cm, keepaspectratio=true]{Assoziierte_Legendre.pdf} \label{abb:315_re}
                \subcaption{Assoziierte Legendre Funktionen Pdf}
\end{subfigure}
\begin{subfigure}[c]{0.49\textwidth}
\includegraphics[width=7.6cm, keepaspectratio=true]{Assoziierte_Legendre.pdf} \label{pic:497_re}
\subcaption{Assoziierte Legendre Funktionen Pdf}
\end{subfigure}
                \caption{Assoziierte Legendre Funktionen Pdf}
                \label{abb:bild1-2}
\end{figure}


\newpage 
\subsection{Code}
\label{sec:code}

% ----------------------------	
% this command replaces the letters ü and so on with its unicode equivalents to be available in the listings environment
	\lstset{literate=%
  {Ö}{{\"O}}1
  {Ä}{{\"A}}1
  {Ü}{{\"U}}1
  {ß}{{\ss}}1
  {ü}{{\"u}}1
  {ä}{{\"a}}1
  {ö}{{\"o}}1
   }
% ----------------------------	

\begin{lstlisting}[style=matlab,
  label=lst:prozess,
  firstnumber=1,
  float={!htb},
  caption={Ein Matlab-Programm}
  ]
  
%Übungsbeispiel Volumenelemente
%by Martin Buchschmid und Siegfried Seipelt

clear all;

femesh('reset');

%Deklaration der benötigten Startknoten
FEnode=[1  0 0 0  0 0 0 ;
        2  0 0 0  0 0.05 0 ;
        3  0 0 0  0 1 0];%nicht verwendete Knoten stellen kein Problem dar
         
%Vorgabe der zunächst leeren Felder für die FE-Elementierung															   
FEelt=[];
FEel0=[];

%Erzeugen eines Balkenelementes zwischen den Knoten 1 und 2
femesh('objectbeamline 1 2');

%Extrudieren des Grundelementes jeweils 10-fach in die Richtungen x=1 und dann z=1 (Die zuerst erzeugte Fläche wird so zu einem Volumenelement)
femesh('extrude 80 0.05 0 0');
femesh('extrude 80 0 0 0.05');
femesh('repeatsel 4 0 0.05 0');


\end{lstlisting}

\appendix                     % Anhang
 \chapter{Anhang}
\label{cha:anhang}

\section{Verwendete Software}
\label{cha:software}

\subsection{Maple\texttrademark}
\label{app:maple}
\emph{Maple\texttrademark} ist ein kommerzielles Mathematik-Paket der Firma Maplesoft. Seine Stärke liegt darin, dass es mathematische Ausdrücke algebraisch verarbeiten kann. Im vorliegenden Fall werden auf diese Weise Ableitungen und Integrale berechnet, ohne auf numerische Aspekte Rücksicht zu nehmen. Dies ermöglicht einen kurzen und trotzdem voll funktionalen Code. \emph{Maple\texttrademark} wurde in den Version 12 und 14 verwendet.\\
Internet: \url{http://www.maplesoft.com/}

\subsection{gnuplot}
\emph{gnuplot} wurde verwendet, um für diese Arbeit Diagramme und Plots anzufertigen. Es ist ein interaktives, Kommandozeilen-gesteuertes Programm, das auf fast allen gängigen Betriebssystemen lauffähig ist. Es ist in der Lage, sowohl 2D als auch 3D Plots zu erstellen und diese in verschiedenen Dateiformaten zu exportieren. \emph{gnuplot} ist ein Open Source Projekt und wird immer noch weiterentwickelt. Hier wurde die zum Zeitpunkt der Niederschrift aktuelle Version 4.2.3 verwendet.\\
Internet: \url{http://www.gnuplot.info/}
die Anzahl der Knoten auf 1000 beschränkt ist. Glücklicher Weise ist die in der REM eine annehmbare Menge, mit der sich durchaus einige Problemen behandeln lassen.

\clearpage

\section{Maple\texttrademark -Programm für ein 2D-Problem}
\label{sec:maplecode}

%
%\lstinputlisting[style=maple,
  %label=lst:maplecode,
  %firstnumber=1,
  %caption={[Maple\texttrademark -Programm für ein 2D-Problem] \texttt{maple2D.mw}}
  %] {code/maplebsp.txt}
  

%\section{{\texttt{C++}} code of dragLift.C}
%\label{sec:draglift}
%
%\lstinputlisting[style=cppcode,
%  label=draglift,
%  firstnumber=1,
%  ] {code/dragLift.C}
%  
%\section{{\texttt{C++}} code of stlIcoFoam.C}
%\label{sec:stlicofoam}
%
%\lstinputlisting[style=cppcode,
%  label=stlicofoam,
%  firstnumber=1,
%  ] {code/stlIcoFoam.C}

%%% Local Variables: 
%%% mode: latex
%%% TeX-master: "main"
%%% End: 
    

\backmatter
\begin{spacing}{1.0}          % Verzeichnisse werden mit einzeiligem Abstand gesetzt
\inputencoding{latin2}
 \bibliography{main_musterthesis}
\inputencoding{utf8}
\end{spacing}

\end{document}

