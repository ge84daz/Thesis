\chapter{Integraltransformationsmethode}
\label{cha:ITM}


\section{Vorbemerkung}
\label{sec:Vorbemerkung_ITM}

Um die unendliche Ausdehnung des Bodens und die in Kapitel (\ref{cha:Grundgleichungen}) beschriebenen elastischen Wellen analytisch zu berechnen, ist die Integraltransformationsmethode (ITM) eine geeignete Berechnungsmethode \citep{Mueller2007}.

Da die ITM allerdings nur einfache Geometrien beschreiben kann, müssen komplexere Strukturen durch eine Überlagerung von Fundamentalsystemen abgebildet werden. 

In der vorliegenden Arbeit wird ein Tunnel im ungeschichteten Boden untersucht.
Für diese geometrische Situation liegt keine Lösung vor, sodass für die Berechnung die Fundamentalsysteme Halbraum und Vollraum mit zylindrischem Hohlraum in den Kapiteln (\ref{sec:Halbraum}) und (\ref{sec:Zylinder}) herangezogen werden.
Im darauffolgenden Kapitel (\ref{sec:Superposition}) werden die Fundamentalsysteme überlagert, um so eine analytische Lösung für die untersuchte geometrische Situation zu erhalten.

Zu beachten ist, dass die in den Kapiteln (\ref{sec:Halbraum}) und (\ref{sec:Zylinder}) beschriebenen Lösungen eine dynamische Belastung  (\(\omega \neq 0\)) voraussetzen. Aufgrund der gewählten Lösungsmethoden ist eine statische Belastung nicht zulässig \citep{Fruehe2010}.

\section{Fundamentalsystem Halbraum}
\label{sec:Halbraum}

\begin{figure}[H]
	\hspace*{45mm}
%	\centering
	\includesvg[height=4.5cm,keepaspectratio]{svg/cha_02_svg_01_hs}
	\caption{Fundamentalsystem homogener Halbraum - basiert auf \citep{Freisinger2022}}
	\label{fig:cha09_hs}
\end{figure}
Wie \cite{Fruehe2010} beschreibt, werden die im Kapitel (\ref{sec:Helmholtz}) hergeleiteten Potentiale $\Phi$ und $\Psi_{i}$ für den Halbraum in kartesischen Koordinaten \((x^1 = x,\ x^2 = y,\ x^3 = z)\) beschrieben. Dadurch ergibt sich ein System entkoppelter partieller Differentialgleichungen zu:
\begin{subequations}\label{eq:partielle_DGL}
		\begin{align}
			&\left[\frac{\partial^{2}}{\partial x^{2}}
			+ \frac{\partial^{2}}{\partial y^{2}}
			+ \frac{\partial^{2}}{\partial z^{2}}
			- \frac{1}{c_p^{2}}\,\frac{\partial^{2}}{\partial t^{2}}\right]
			\Phi(x,y,z,t) = 0
			\label{eq:phi_wave} \\[6pt]
			&\left[\frac{\partial^{2}}{\partial x^{2}}
			+ \frac{\partial^{2}}{\partial y^{2}}
			+ \frac{\partial^{2}}{\partial z^{2}}
			- \frac{1}{c_s^{2}}\,\frac{\partial^{2}}{\partial t^{2}}\right]
			\Psi_{i}(x,y,z,t) = 0
			\label{eq:psi_wave}
		\end{align}
\end{subequations}
mit $\Psi_{i}$ in den drei kartesichen Raumrichtungen (\( i = x, y, z.\))

\newcommand{\mapto}{\mathrel{\laplace}}
Um dieses partielle System zu lösen, erfolgt eine Überführung zu gewöhnlichen Differentialgleichungen mittels dreifacher Fouriertransformation. 
Dazu werden die Ortskoordinaten $x$ und $y$ aus dem Originalraum zu Wellenzahlen \(k_x\) und \(k_y\) im fouriertransformierten Bildraum \((x \mapto k_x,\; y \mapto k_y)\) sowie die Zeit $t$ in den Frequenzbereich $\omega$ \((t \mapto \omega\)) überführt.\\
Um entlang der $z$-Koordinate  im Folgenden Randbedingungen vergeben zu können, bleibt diese untransformiert im Originalraum \citep{Mueller2007}.

Die Transformation des Systems partieller Differentialgleichungen (\ref{eq:partielle_DGL}) in den fouriertransfmorierten Bildraum führte zur Umwandlung in das System gewöhnlicher Differentialgleichungen und lässt sich darstellen durch:
\begin{subequations}\label{eq:gewöhnliche_DGL}
	{	\begin{align}
			&\left[-k_x^{2}-k_y^{2}+k_p^{2}+\frac{\partial^{2}}{\partial z^{2}}\right]
			\hat{\Phi}(k_x,k_y,z,\omega) = 0 \label{eq:phi_helmholtz}\\[6pt]
			&\left[-k_x^{2}-k_y^{2}+k_s^{2}+\frac{\partial^{2}}{\partial z^{2}}\right]
			\hat{\Psi}_{i}(k_x,k_y,z,\omega) = 0 \label{eq:psi_helmholtz}
		\end{align}
	}
\end{subequations}
mit der Kompressionswellenzahl \( k_p = \frac{\omega}{c_p} \;\text{und der Scherwellenzahl}\; k_s = \frac{\omega}{c_s} \).

Das Symbol (\^{}) kennzeichnet dabei ab sofort die fouriertransformierten Größen.


Zur Lösung zieht \cite{Fruehe2010} einen analytischen Exponentialansatz (\ref{eq:expo_solutions}) heran. Da außerdem gemäß \cite{Long1967} für das Potential $\Psi_{z} = 0$ gilt, wird die folgende Lösung in $\alpha=x,y$ definiert.
\begin{subequations}\label{eq:expo_solutions}
	{\begin{align}
			&\hat{\Phi} = A_{1} e^{\lambda_{1} z} + A_{2} e^{-\lambda_{1} z}\label{eq:phi_sol}\\[4pt]
			&\hat{\Psi}_{\alpha} = B_{\alpha1} e^{\lambda_{2} z} + B_{\alpha2} e^{-\lambda_{2} z}\label{eq:psi_sol}
		\end{align}
	}
\end{subequations}
mit \( \lambda_{1} = \sqrt{k_x^{2}+k_y^{2}-k_p^{2}},\;
\lambda_{2} = \sqrt{k_x^{2}+k_y^{2}-k_s^{2}} \)

sowie den unbekannten Koeffizienten
\(A_1, A_2, B_{\alpha1}, B_{\alpha2}\)

Der Lösungsansatz (\ref{eq:expo_solutions}) der Potentiale $\hat{\Phi}$ und $\hat{\Psi}_{\alpha}$ ermöglicht die Berechnung der Verschiebungen \(\hat{u}_{k}\) und Spannugen \(\hat{\sigma}_{k}\) im fouriertransformierten Raum. 
Der Index (\({}_k\)) steht dabei für den Halbraum, der im kartesischen Koordinatensystem beschrieben wird.

So lässt sich die Verschiebung im fouriertransformierten Raum \(\hat{u}_{k}\) über die Matrix $\bigl[\hat{H}_{k}\bigr]$, dessen Einträge den Exponentialansatz (\ref{eq:expo_solutions}) enthalten, und dem Vektor $C_k$, der die unbekannten Koeffizienten enthält, wie folgt darstellen \citep{Fruehe2010}: 
\begin{equation}\label{eq:displ_hs}
	\hat{u}_{k} = \bigl[\hat{H}_{k}\bigr]\cdot C_{k}
\end{equation}
%Die Einträge der Gleichung (\ref{eq:displ_hs}) sind im Anhang (\ref{cha:Halbraum}) aufgeführt. Eine genauere Herleitung der Lösung ist außerdem in den Dissertationen von \cite{Hackenberg2016} und \cite{Freisinger2022} zu finden.

%ablage:
%Dabei enthalten die Matrizen $\bigl[\hat{H}_{z}\bigr]$ und $\bigl[\hat{K}_{z}\bigr]$ wieder Einträge, die aus dem gewählten Lösungsansatz resultieren.

Über die Verschiebungs-Verzerrungsbeziehung, die Spannungs-Verzerrungsbeziehung sowie einer Transfomation in den Fourierraum lassen sich die Spannungen \(\hat{\sigma}_{k}\) nach \cite{Mueller2007} hergeleiten. Sie lassen sich darstellen durch:
\begin{equation}\label{eq:sigma_factorization}
	\hat{\sigma}_{k} = \bigl[\hat{K}_{k}\bigr]\cdot C_{k}
\end{equation}
Dabei enthalten die Matrizen $\bigl[\hat{H}_{k}\bigr]$ und $\bigl[\hat{K}_{k}\bigr]$ Einträge, die aus dem gewählten Lösungsansatz resultieren und der Vektor $C_k$ die unbekannten Koeffizienten.

Eine ausführliche Herleitung der Gleichungen (\ref{eq:displ_hs}) und (\ref{eq:sigma_factorization}), sowie die konkreten Einträge der Vektoren und Matrizen ist neben in der Arbeit von \cite{Fruehe2010} auch in der Dissertation von \cite{Mueller2007} zu finden.
%Für eine genaue Herleitung wird auf die Dissertation von \cite{Fruehe2010} verwiesen.

%Die Gleichung (\ref{eq:sigma_factorization}) enthält ebenfalls den Vektor $C_k$ mit den unbekannten Koeffizienten. Die Einträge der Gleichung (\ref{eq:sigma_factorization} sind ebenfalls im Anhang (\ref{cha:Halbraum}) zu finden. 

Die sechs unbekannten Koeffizienten \(A_1, A_2, B_{\alpha1}, B_{\alpha2}\), die aus dem gewählten Exponentialansatz (\ref{eq:expo_solutions}) resultieren, werden über Randbedingungen bestimmt.

Die drei Koeffizienten \(A_2, B_{x2}, B_{y2}\) ergeben sich unter Heranziehung der Sommerfeldschen Abstrahlbedingung. Diese fordert, dass die Amplitude der Wellen in zunehmender Tiefe abklingt und sich die Wellenausbreitung nur in positiver z-Richtung, also weg von der Halbraumoberfläche, fortsetzt.

Die übrigen unbekannten Koeffizienten \(A_1, B_{x1}, B_{y1}\) werden durch die Randbedingung an der Halbraumoberfläche hergeleitet, da die angreifende Belastung mit der Spannung an der Oberfläche im Gleichgewicht stehen muss \citep{Mueller2007}.

Um nun die Spannungen und Verschiebungen im Originalraum ($x$, $y$ und $z$) zu erhalten, erfolgt eine mehrfache Fourier-Rücktransformation. 



\section{Fundamentalsystem Vollraum mit zylindrischem Hohlraum}
\label{sec:Zylinder}

Im Gegensatz zur Lösungsfindung im Halbraum, werden die im Kapitel (\ref{sec:Helmholtz}) hergeleiteten Potentiale $\Phi$ und $\Psi_{i}$ für das System Vollraum mit zylindrischem Hohlraum in Zylinderkoordinaten beschrieben. 
\begin{figure}[H]
	\centering
	\begin{subfigure}[t]{0.48\textwidth}
		\hspace*{25mm}
		\centering
		\includesvg[width=\linewidth,pretex=\centering]{svg/cha_02_svg_05_fs_cyl}
		\label{fig:cyl_a}
	\end{subfigure}\hfill
	\begin{subfigure}[t]{0.48\textwidth}
		\centering
			\hspace*{-7mm}
		\includesvg[width=\linewidth,pretex=\centering]{svg/cha_02_svg_06_cyl_cos}
		\label{fig:cyl_b}
	\end{subfigure}
	\caption{Vollraum mit zylindrischem Hohlraum (rechts) und Zylinderkoordinaten (links) \citep{Freisinger2022}}
	\label{fig:cyl_pair}
\end{figure}


Wie in der Abbildung (\ref{fig:cyl_pair}) dargestellt, wird die räumliche Lage eines Punktes durch die Koordinaten $x$, $r$ und $\phi$ beschrieben, wobei \(x\) die longitudinale, \(r\) die radiale und \(\phi\) die umlaufende Koordinate darstellt \cite{Freisinger2022}.


Da das Vektorpotential in Zylinderkoordinaten nicht direkt zu entkoppelten Gleichungen führt \citep{Hackenberg2016}, zeigt \cite{Fruehe2010}, dass das Vektorfeld $\Psi_{i}$ unter der Zusatzbedingung, dass es quellfrei ist, durch zwei voneinander unabhängige Skalarfunktionen $\psi$ und $\chi$ dargestellt werden kann.
Mithilfe dieser erweiterten Helmholtz-Zerlegung lassen sich schließlich wieder entkoppelte partielle Differentialgleichungen gewinnen, die sich wie folgt darstellen lassen:
\begin{subequations}\label{eq:cyl_wave}
	{
		\begin{align}
			&\left[\frac{\partial^{2}}{\partial x^{2}}
			+ \frac{\partial^{2}}{\partial r^{2}}
			+ \frac{1}{r}\frac{\partial}{\partial r}
			+ \frac{1}{r^{2}}\frac{\partial^{2}}{\partial \varphi^{2}}
			- \frac{1}{c_p^{2}}\frac{\partial^{2}}{\partial t^{2}}\right]
			\Phi(x,r,\varphi,t)=0 \label{eq:cyl_phi}\\[6pt]
			&\left[\frac{\partial^{2}}{\partial x^{2}}
			+ \frac{\partial^{2}}{\partial r^{2}}
			+ \frac{1}{r}\frac{\partial}{\partial r}
			+ \frac{1}{r^{2}}\frac{\partial^{2}}{\partial \varphi^{2}}
			- \frac{1}{c_s^{2}}\frac{\partial^{2}}{\partial t^{2}}\right]
			\psi(x,r,\varphi,t)=0 \label{eq:cyl_psi}\\[6pt]
			&\left[\frac{\partial^{2}}{\partial x^{2}}
			+ \frac{\partial^{2}}{\partial r^{2}}
			+ \frac{1}{r}\frac{\partial}{\partial r}
			+ \frac{1}{r^{2}}\frac{\partial^{2}}{\partial \varphi^{2}}
			- \frac{1}{c_s^{2}}\frac{\partial^{2}}{\partial t^{2}}\right]
			\chi(x,r,\varphi,t)=0 \label{eq:cyl_chi}
		\end{align}
	}%
\end{subequations}


Analog zum Vorgehen in Kapitel (\ref{sec:Halbraum}) werden die partiellen Differentialgleichungen über eine Überführung mittels Fouriertransformation gelöst. Dabei genügt, anders als in kartesischen Koordinaten, beim zylindrischen Hohlraum
eine zweifache Fouriertransformation bezüglich der Ortskoordinate $x$ in die Wellenzahlen $k_{x}$
sowie die Zeit $t$ in den Frequenzbereich \((x \mapto k_x,\; t \mapto \omega)\). 
Die Ortskoordinate $r$ bleibt dabei untransformiert.

Das resultierdende System gewöhnlicher Differentialgleichungen lässt sich nach \cite{Fruehe2010} darstellen durch:
\begin{subequations}\label{eq:cyl_fourier_odes}
	{%
		\begin{align}
			&\left[-k_x^{2}
			+ \frac{\partial^{2}}{\partial r^{2}}
			+ \frac{1}{r}\frac{\partial}{\partial r}
			- \frac{n^{2}}{r^{2}}
			+ k_p^{2}\right]\,
			\hat{\Phi}\!\left(k_x,r,n,\omega\right) = 0 \label{eq:13a}\\[6pt]
			&\left[-k_x^{2}
			+ \frac{\partial^{2}}{\partial r^{2}}
			+ \frac{1}{r}\frac{\partial}{\partial r}
			- \frac{n^{2}}{r^{2}}
			+ k_s^{2}\right]\,
			\hat{\psi}\!\left(k_x,r,n,\omega\right) = 0 \label{eq:13b}\\[6pt]
			&\left[-k_x^{2}
			+ \frac{\partial^{2}}{\partial r^{2}}
			+ \frac{1}{r}\frac{\partial}{\partial r}
			- \frac{n^{2}}{r^{2}}
			+ k_s^{2}\right]\,
			\hat{\chi}\!\left(k_x,r,n,\omega\right) = 0 \label{eq:13c}
		\end{align}
	}%
\end{subequations}

Die Lösung der Differentialgleichung (\ref{eq:cyl_fourier_odes}) leitet sich durch die Anwendung von Hankel-Funktionen her. Die daraus resultierenden Verschiebungen \(\hat{u}_{z}\) ergeben sich zu:
\begin{equation}\label{eq:solution_cyl_uz}
	\hat{u}_{z} = \bigl[\hat{H}_{z}\bigr]\cdot C_{z}
\end{equation}
Der Index (\({}_z\)) steht dabei für den Vollraum mit zylindrischem Hohlraum, der durch Zylinderkoordinaten beschrieben wird.

Analog zur Herleitung der Spannungen im Halbraum \(\hat{\sigma}_{k}\) ergeben sich die Spannungen in Zylinderkoordinaten \(\hat{\sigma}_{z}\) aus der Verschiebung, der Verschiebungs- und der Spannungs-\\Verzerrungsbeziehung sowie einer Transformation in den Bildraum.
\begin{equation}\label{eq:solution_cyl_sig}
	\hat{\sigma}_{z} = \bigl[\hat{K}_{z}\bigr]\cdot C_{z}
\end{equation}
Dabei enthalten die Matrizen $\bigl[\hat{H}_{z}\bigr]$ und $\bigl[\hat{K}_{z}\bigr]$ wieder Einträge, die aus dem gewählten Lösungsansatz resultieren.

Eine ausführliche Herleitung der Gleichungen (\ref{eq:solution_cyl_uz}) und (\ref{eq:solution_cyl_sig}), sowie die konkreten Einträge der Vektoren und Matrizen ist neben in der Arbeit von \cite{Fruehe2010} auch in der Dissertation von \cite{Mueller2007} zu finden.

Analog zu den Gleichungen (\ref{eq:displ_hs}) und (\ref{eq:sigma_factorization}) enthält der Vektor \(C_{z}\) ebenfalls sechs unbekannte Koeffizienten, die mittels Randbedingungen ermittelt werden.

Drei davon ergeben sich aus der Bedingung, dass sich die Amplituden der Wellen mit zunehmender Entfernung von der Belastung abnehmen und sich nur vom Ort der Belastung ausbreiten.
Die übrigen Unbekannten werden über die Bedingung bestimmt, dass die eingeleitete Last am Zylinderrand mit den Spannungen im Gleichgewicht steht \citep{Fruehe2010}.

Um die Spannungen und Verschiebungen im Orginalraum ($x$, $r$ und $\phi$) zu erhalten erfolgt eine erneute mehrfache Fourier-Rücktransformation. 





\section{Überlagerung der Fundamentalsysteme}
\label{sec:Superposition}

Eine Lösung für das komplexe System Halbraum mit zylindrischem Hohlraum wird durch Überlagerung der in Kapitel (\ref{sec:Halbraum}) und (\ref{sec:Zylinder}) beschriebenen Fundamentalsysteme ermittelt.

Dafür werden die beiden Fundamentalsysteme Halbraum und Vollraum mit zylindrischem Hohlraum an einer gemeinsamen, fiktiven Kopplungsfläche miteinander verknüpft. 
An dieser Fläche müssen zwei Bedingungen erfüllt sein. 
Einerseits muss ein Kräftegleichgewicht herrschen, sodass sich die aufgebrachten Spannungen der beiden Teilsysteme gegenseitig aufheben. 
Andererseits müssen die Verschiebungen der beiden Teilsysteme an der Kopplungsfläche identisch sein.
\begin{figure}[H]
	\hspace*{5mm}
	\centering
	\begin{subfigure}[t]{0.48\textwidth}
		\centering
		\includesvg[width=\linewidth,pretex=\centering]{svg/cha_03_svg_03_hs_sup_stress}
		\label{fig:Halbraum_fiktiveFlächen}
	\end{subfigure}\hfill
	\begin{subfigure}[t]{0.48\textwidth}
		\centering
		\includesvg[width=\linewidth,pretex=\centering]{svg/cha_03_svg_04_fs_cyl_sup_stress}
		\label{fig:Zylinder_fiktiveFlächen}
	\end{subfigure}
	\caption{Fundamentalsysteme Halbraum (rechts) und Vollraum mit zylindrischem Hohlraum (links) mit fiktiven Überlagerungsflächen - basiert auf \cite{Freisinger2022}}
	\label{fig:Überlagerung_fiktiveFlächen}
\end{figure}
Wie in Abbildung (\ref{fig:Überlagerung_fiktiveFlächen}) dargestellt, wird zu diesem Zweck im Halbraum an der Stelle, an der sich der Tunnel befindet, eine fiktive Zylinderoberfläche $\delta\Gamma_{\mathrm z}$ eingeführt. Im Vollraum mit zylindrischem Hohlraum wird entsprechend der Bodenoberfläche eine fiktive Halbraumoberfläche $\delta\Lambda$ eingeführt. 
Entlang dieser Schnittflächen werden die genannten Randbedingungen formuliert und die Lösungen beider Teilsysteme miteinander kombiniert \citep{Mueller2007}.
 
 Da die Spannungs- und Verschiebungsgrößen der beiden Teilsysteme in unterschiedlichen Koordinatensystemen vorliegen, ist eine Koordinatentransformation erforderlich.
 \cite{Fruehe2010} definiert hierfür die Transformationsmatrix (\ref{eq:Transforationsmatrix}), mit der die Größen zwischen den beiden Systemen umgerechnet werden können.
\begin{equation} \label{eq:Transforationsmatrix}
\bigl[\beta\bigr]
= \begin{bmatrix}
	1 & 0 & 0 \\
	0 & -\sin\varphi & \cos\varphi \\
	0 & -\cos\varphi & -\sin\varphi
\end{bmatrix}
\end{equation}
Mithilfe der Matrix (\ref{eq:Transforationsmatrix}) können Koordinatensysteme umgerechnet werden, sodass die Kopplungsbedingungen über die Verschiebungen $\hat{u} $ und die Spannungen $ \hat{\mathbf{\sigma}} $ formuliert und durchgeführt werden können.

Eine detaillierte Herleitung der Kopplungsbedingungen ist in den Dissertationen von \cite{Fruehe2010}, \cite{Hackenberg2016} und \cite{Freisinger2022} zu finden.


\newpage


\section{Dynamische Steifigkeitsmatrix des ITM-Teilsystems}
\label{sec:Steifigkeiten_ITM}

Die Kopplung der Lösungen der ITM und der FEM erfolgt in Kapitel (\ref{cha:Kopplung}) mittels der dynamischen Steifigkeitsmatrix der Berechnungsmethoden. Aus diesem Grund ist es erforderlich, die Steifigkeitsmatrix für die ITM noch zu ermitteln.

Zur besseren Nachvollziehbarkeit im Kopplungskapitel (\ref{cha:Kopplung}) werden die Vektoren und Matrizen mit dem Index ITM versehen.

Die Berechnung der dynamischen Steifigkeitsmatrix erfolgt gemäß der folgenden allgemeinen Beziehung:
\begin{equation}\label{eq:itm_system}
	\hat{\mathbf K}_{\mathrm{ITM}}\;\hat{\mathbf u}_{\mathrm{ITM}}
	= \hat{\mathbf P}_{\mathrm{ITM}}\,
\end{equation}
Wie in der Dissertation von \cite{Hackenberg2016} dargestellt, wird das Verschiebungsfeld $\hat{\mathbf u}_{\mathrm{ITM}}$ mit den Amplituden $\mathbf C$ und dem Verschiebungs-Operator $\hat{\mathbf U}_{\mathrm{ITM}}$. 
Außerdem wird der Lastvektor $\hat{\mathbf P}_{\mathrm{ITM}}$ mit den Amplituden $\mathbf C$ und einem Spannungs-Operator, resultierend aus den Randbedinungungen, $\hat{\mathbf S}_{\mathrm{ITM}}$ dargestellt.
\begin{subequations}\label{eq:itm_system_blocks}
	\begin{align}
		&\hat{\mathbf S}_{\mathrm{ITM}}\,\mathbf C = \hat{\mathbf P}_{\mathrm{ITM}}, \label{eq:itm_system_blocks_a}\\
		&\hat{\mathbf u}_{\mathrm{ITM}} = \hat{\mathbf U}_{\mathrm{ITM}}\,\mathbf C. \label{eq:itm_system_blocks_b}
	\end{align}
\end{subequations}
Im nächsten Schritt werden die Beziehungen aus der Gleichung (\ref{eq:itm_system_blocks}) in die Gleichung (\ref{eq:itm_system}) eingesetzt, um die dynamische Steifigkeitsmatrix zu ermitteln.
\begin{subequations}\label{eq:itm_subeqs}
	\begin{align}
		&\hat{\mathbf u}_{\mathrm{ITM}}
		= \hat{\mathbf U}_{\mathrm{ITM}}\;\hat{\mathbf S}_{\mathrm{ITM}}^{-1}\;\hat{\mathbf P}_{\mathrm{ITM}}
		\label{eq:itm_subeqs_u}\\
		&\text{mit }\;\hat{\mathbf K}_{\mathrm{ITM}}
		= \hat{\mathbf S}_{\mathrm{ITM}}\;\hat{\mathbf U}_{\mathrm{ITM}}^{-1}
		\label{eq:itm_subeqs_K}
	\end{align}
\end{subequations}
Um die in Kapitel (\ref{cha:Kopplung}) beschriebene Kopplung durchzuführen, ist es erforderlich, die herangezogene Gleichung (\ref{eq:itm_system}) in Blockdarstellung, getrennt auf die beiden Koppelflächen $\Lambda$ und $\Gamma$ darzustellen.
\begin{equation}\label{eq:itm_block_system_braced}
	\underbrace{\begin{bmatrix}
			\hat{\mathbf K}_{\Lambda\Lambda_{\mathrm{ITM}}} &
			\hat{\mathbf K}_{\Lambda\Gamma_{\mathrm{ITM}}} \\
			\hat{\mathbf K}_{\Gamma\Lambda_{\mathrm{ITM}}} &
			\hat{\mathbf K}_{\Gamma\Gamma_{\mathrm{ITM}}}
	\end{bmatrix}}_{\hat{\mathbf K}_{\mathrm{ITM}}}\,
	\begin{pmatrix}
		\hat{\mathbf u}_{\Lambda_{\mathrm{ITM}}} \\
		\hat{\mathbf u}_{\Gamma_{\mathrm{ITM}}}
	\end{pmatrix}
	=
	\begin{pmatrix}
		\hat{\mathbf P}_{\Lambda_{\mathrm{ITM}}} \\
		\hat{\mathbf P}_{\Gamma_{\mathrm{ITM}}}
	\end{pmatrix}.
\end{equation}
Für eine detaillierte Herleitung wird auf die Dissertationen von \cite{Hackenberg2016} und \cite{Freisinger2022} verwiesen.