\addchap{Symbolverzeichnis}
\markboth{Symbolverzeichnis}{Symbolverzeichnis}
\label{cha:symbolverzeichnis}

Der Zusatz $\hat{\medspace}$ bezeichnet eine komplexe Größe.

\section*{Griechische Buchstaben}
\begin{longtable}[l]{lcp{8cm}l}
$\Gamma$ & & Berandung des Problems \\
$\delta\left(\left|\vec{x}_{0} - \vec{x}\right|\right)$ & & Dirac-Funktion am Punkt $\vec{x}$ \\
\end{longtable}

\section*{Lateinische Buchstaben}
\begin{longtable}[l]{lcp{8cm}l}
%\hspace*{2.5cm}\= \hspace*{1cm} \=  Schallschnelle senkrecht zu einer Oberfläche \= \kill
$c$ & \einheit{\frac{m}{s}} & Schallgeschwindigkeit & $c = \frac{E}{\rho}$ \\
$E$ & \einheit{\frac{N}{m^{2}}} & Elastizitätsmodul\\
$f$ & \einheit{\frac{1}{s}} & Frequenz & $f = \frac{\omega}{2 \pi}$ \\ 
\end{longtable}



\section*{Mathematische Symbole}
\begin{longtable}[l]{@{}m{3em}p{12cm}@{}}
 $A^{\mathsf T}$ & Transponierte Matrix A \\
 $A^{\mathsf H}$    & Komplex konjugierte Matrix A \\
\end{longtable}





\section*{Diakritische Zeichen}
\newcommand{\symbox}{\mathord{\vcenter{\hbox{\fbox{\rule{0.6ex}{0pt}\rule{0pt}{0.6ex}}}}}}
\begin{longtable}[l]{@{}m{3em}p{12cm}@{}}
	$\overline{\symbox}$ & Kennzeichnung für eine einfach Fouriertransformierte Größe \\
	$\tilde{\symbox}$    & Kennzeichnung für eine zweifach Fouriertransformierte Größe \\
	$\hat{\symbox}$      & Kennzeichnung für eine dreifach Fouriertransformierte Größe \\
\end{longtable}

\section*{Akronyme}
\setlength\LTleft{0pt}
\setlength\LTright{0pt}

\newcolumntype{L}[1]{>{\raggedright\arraybackslash}p{#1}}

\begin{longtable}{@{}L{6em}L{10cm}@{}}
	FE  & Finite Elemente \\
	FEM & Finite Elemente Methode \\
	GP  & Gauß Punkte \\
	ITM & Integraltransformationsmethode \\
\end{longtable}