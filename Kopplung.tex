\chapter{Kopplung der FEM mit der ITM}
\label{cha:Kopplung}

\section{Vorbemerkung}
\label{sec:Vorbem_Kopplung}

Die Auswertung der Lösung für das Gesamtsystem erfordert eine Kopplung der Berechnungsmethoden ITM und FEM.
Die Kopplung der beiden Berechnungsmethoden wird durch die Übergangsbedingungen an der Kopplungsfläche $\Gamma$ erreicht, da an diesen Stellen zum einen ein Kräftegleichgewicht herrschen muss und zudem die Verschiebungen identisch sein müssen.

Die Bedingungen für die Kopplung werden im Kapitel (\ref{sec:Kopplung}) erläutert. Im Vorfeld ist zu berücksichtigen, dass die Teillösungen sich auf die gleichen Koordinaten beziehen müssen. Daher ist im Vorfeld eine Transformation notwendig, die im Kapitel (\ref{sec:Transformation_Kopplung}) vorgestellt wird.

...... Hier noch Größere Abbildung von gekoppelten Mesh reinmachen!!!....MT Pfleger 3.12 für mich abzeichnen!s

\section{Transfomation der Lösung des FE-Systems}
\label{sec:Transformation_Kopplung}

Bei der Kopplung ist zu berücksichtigen, dass die hergeleiteten Lösungen sich auf unterschiedliche Koordinatensysteme beziehen sowie in verschiedenen Fourierräumen liegen, wie in der folgenden Übersicht (\ref{tab:Übersichtstabelle}) dargestellt:
\begin{table}[htb]
	\centering
	\normalsize
	\begin{tabular}{ccc}
		\firsthline
		& \textbf{FEM} & \textbf{ITM} \\\hline
		Koordinatensystem auf $\Gamma$ & kartesisch $(x,y,z)$ & Zylinderkoordinaten $(x,r,\varphi)$ \\
		Fourier-Raum                     & zweifachtransformiert $(k_x,\omega)$        & dreifachtransformiert$(k_x,n,\omega)$ \\\lasthline
	\end{tabular}
	\caption{Übersicht der Fourierräume und Koordinatensysteme von FEM und ITM.}
	\label{tab:Übersichtstabelle}
\end{table}
Im Rahmen der Formulierung der genannten Übergangsbedingungen ist es erforderlich, dass sich die Lösungen der Teilsysteme auf identische Größen beziehen.
Für die Kopplung ist demnach eine Transformation an der Kopplungsoberfläche $\Gamma$ erforderlich. Gemäß \cite{Freisinger_Hackenberg2020} erfolgt die Transformation der Lösung des FEM-Systems in das der ITM.

Zunächst werden die Verschiebungen des FEM-Systems in das Zylinderkoordinatensystem der ITM-Lösung übertragen.
Dafür greift \cite{Fruehe2010} erneut die Transformationsmatrix $\bm{\beta}$ aus Gleichung (\ref{eq:Transforationsmatrix}) auf und formuliert den Wechsel der Koordinatensysteme von kartesisch zu Zylinderkoordinaten wie folgt:
\begin{equation}\label{eq:u_FE_Zylinder}
	\tilde{\mathbf u}_{\Gamma_{\mathrm{FE,k}}}
	= \mathbf T_{1}\,
	\tilde{\mathbf u}_{\Gamma_{\mathrm{FE,z}}}
\end{equation}
Dabei werden die Größen im kartesischen Koordinatensystem fortfolgend mit k bezeichnet, während das Zylinderkoordinatensystem mit z gekennzeichnet wird.
Die Einträge der Matrix $\mathbf{T_1}$ sind im Anhang (\ref{sec:Koordinatentransformationsmatrix}) aufgeführt.

Im nächsten Schritt ist es erforderlich, $\tilde{\mathbf u}_{\Gamma_{\mathrm{FE,z}}}$ in den selben Fourierraum zu übertragen. Zu diesem Zweck wird eine Fourier-Reihenentwicklung bezüglich des Umfangs der zylindrischen Kopplungsfläche $\Gamma$ durchgeführt \citep{Hackenberg2016}:
\begin{equation}\label{eq:u_FE_Fourier}
	\tilde{\mathbf u}_{\Gamma_{\mathrm{FE,z}}}
	= \mathbf T_{2}\,
	\hat{\mathbf u}_{\Gamma_{\mathrm{FE,z}}}
\end{equation}
Wie bereits in den vorangegangenen Kapiteln aufgeführt, werden zweifach fouriertransformierte Größen mit \(\tilde{\cdot}\) bezeichnet,
während dreifach fouriertransformierte Größen mit \(\widehat{\cdot}\) gekennzeichnet werden.
Die konkreten Einträge der Matrix \(\mathbf T_{2}\) sind erneut im Anhang (\ref{sec:Fouriertransformationsmatrix}) zu finden.

Zusammenfassend lässt sich festhalten, dass durch die Anwendung der Transformationsmatrix \(\mathbf T\) eine Transformation auf einheitliche Koordinaten \\$(k_x,\ r,\ n,\ \omega)$ erzielt werden kann.
\begin{equation}\label{eq:tilde_u_FE}
	\tilde{\mathbf u}_{\Gamma,\mathrm{FE},k}
	= \mathbf T_{1}\,\mathbf T_{2}\,\hat{\mathbf u}_{\Gamma,\mathrm{FE},z}
	= \mathbf T\,\hat{\mathbf u}_{\Gamma,\mathrm{FE},z}\,.
\end{equation}

Analog dazu erfolgt ebenfalls die Transformation des Lastvektor in die Koordinaten der ITM-Lösung $(k_x,\ r,\ n,\ \omega)$ mifhilfe der Transformationsmatrix \(\mathbf T\):
\begin{equation}\label{eq:tildeP_FE}
	\tilde{\mathbf P}_{\Gamma,\mathrm{FE,k}}
	= \mathbf T\,
	\hat{\mathbf P}_{\Gamma,\mathrm{FE,z}} \,.
\end{equation}



\section{Kopplung}
\label{sec:Kopplung}

Nachdem sich die Lösungen der Teilsysteme FEM und ITM nun auf die gleichen Koordinatensysteme beziehen, können im weiteren Verlauf die zu Beginn dieses Kapitels (\ref{sec:Vorbem_Kopplung}) angegebenen Kopplungsbedingungen angewendet werden.

Einerseits wird gefordert, dass die Verschiebungen der Teilsysteme an der Kopplungsfläche identisch sind. Daraus ergibt sich die erste Kopplungsbedinung zu:
\begin{equation}\label{eq:Kopplbed_u}
	\hat{\mathbf u}_{\Gamma_{\mathrm{FEM}}}
	= \hat{\mathbf u}_{\Gamma_{\mathrm{ITM}}}\,.
\end{equation}
Des Weiteren ist es essenziell, dass an der Kopplungsfläche $\Gamma$ die auftretenden Kräfte im einem Gleichgewichtszustand sind. Die Lasten der ITM und FEM müssen mit einer potenziellen Randlast auf der Kopplungsfläche $\Gamma$ im Gleichgewicht stehen. Die zweite Kopplungsbedingung lautet daher wie folgt: 
\begin{equation}\label{eq:Kopplbed_p}
	\hat{\mathbf P}_{\Gamma,\mathrm{ITM}}
	+ \frac{1}{\mathrm d s}\,\hat{\mathbf P}_{\Gamma,\mathrm{FE}}
	= \hat{\mathbf P}_{\Gamma}\,.
\end{equation}
Der Nenner $\mathrm{ds}$ wird dabei verwendet, um die knotenweisen FEM-Lasten in stetige Spannungen entlang der Kopplungsfläche umzuwandeln \cite{Hackenberg2016}.

\cite{Hackenberg2016} kombiniert die Koplungsbedingungen (\ref{eq:Kopplbed_u}) sowie (\ref{eq:Kopplbed_p}) mit den Lösungen der Teilsysteme in den Gleichungen (\ref{eq:itm_block_system_braced}) und (\ref{eq:fe_block_system}), sodass das folgende Blocksystem resultiert:
\begin{equation}\label{eq:gekoppelt}
	\begin{bmatrix}
		\hat{\mathbf K}_{\Lambda\Lambda_{\mathrm{ITM}}} &
		\hat{\mathbf K}_{\Lambda\Gamma_{\mathrm{ITM}}}   &
		0 \\[6pt]
		\hat{\mathbf K}_{\Gamma\Lambda_{\mathrm{ITM}}}   &
		\hat{\mathbf K}_{\Gamma\Gamma_{\mathrm{ITM}}}
		+ \dfrac{1}{ds}\,\mathbf T^{-1}\,\tilde{\mathbf K}_{\Gamma\Gamma_{\mathrm{FE}}}\,\mathbf T &
		\dfrac{1}{ds}\,\mathbf T^{-1}\,\tilde{\mathbf K}_{\Gamma\Omega_{\mathrm{FE}}} \\[6pt]
		0 &
		\tilde{\mathbf K}_{\Omega\Gamma_{\mathrm{FE}}}\,\mathbf T &
		\tilde{\mathbf K}_{\Omega\Omega_{\mathrm{FE}}}
	\end{bmatrix}
	\begin{pmatrix}
		\hat{\mathbf u}_{\Lambda_{\mathrm{ITM}}}\\[2pt]
		\hat{\mathbf u}_{\Gamma}\\[2pt]
		\tilde{\mathbf u}_{\Omega_{\mathrm{FE}}}
	\end{pmatrix}
	=
	\begin{pmatrix}
		\hat{\mathbf P}_{\Lambda_{\mathrm{ITM}}}\\[2pt]
		\hat{\mathbf P}_{\Gamma}\\[2pt]
		\tilde{\mathbf P}_{\Omega_{\mathrm{FE}}}
	\end{pmatrix}.
\end{equation}
Das Resultat ist ein System, das Unbekannte auf der Halbraumoberfläche $\Lambda$, der gemeinsamen Kopplungfläche $\Gamma$ sowie im Inneren des FE-Feldes $\Omega$ enthält.

Um eine Endlösung des gekoppelten Systems zu erhalten, müssen die Größen mittels Inverser Fouriertransformation aus dem Fourierraum in den Originalraum zurückgeführt werden. Erst dann sind die Teilsysteme final auswertbar gekoppelt.