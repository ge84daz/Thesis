\chapter{Kopplung der FEM mit der ITM}
\label{cha:Kopplung}

\section{Vorbemerkung}
\label{sec:Vorbem_Kopplung}

Die Auswertung der Lösung für das Gesamtsystem erfordert eine Kopplung der Teilsysteme ITM und FEM.
Die Kopplung der beiden Berechnungsmethoden wird durch die Übergangsbedingungen an der Kopplungsfläche $\Gamma_z$ erreicht, da an diesen Stellen zum einen ein Kräftegleichgewicht herrschen muss und zum anderen die Verschiebungen identisch sein müssen.

Die Übergangsbedingungen der Kopplung werden im Kapitel (\ref{sec:Kopplung}) angewendet. Im Vorfeld ist zu berücksichtigen, dass sich die Teillösungen auf die gleichen Koordinaten beziehen müssen. 
Die erforderliche Transformation ist in Kapitel (\ref{sec:Transformation_Kopplung}) beschrieben.
\begin{figure}[H]
	\hspace*{37mm}
	%	\centering
	\includesvg[height=5cm,keepaspectratio]{svg/Kopplung}
	\caption{ITM und FEM Flächendefinitionen - basiert auf \citep{Freisinger2022}.}
	\label{fig:Kopplung}
\end{figure}

\section{Transfomation der Lösung des FE-Teilsystems}
\label{sec:Transformation_Kopplung}
Bei der Kopplung ist zu berücksichtigen, dass die hergeleiteten Lösungen sich auf unterschiedliche Koordinatensysteme beziehen sowie in verschiedene Fourierräume transformiert sind, wie in der Übersicht (\ref{tab:Übersichtstabelle}) dargestellt.
\begin{table}[htb]
	\centering
	\normalsize
	\begin{tabular}{ccc}
		%\firsthline
		& \textbf{FEM} & \textbf{ITM} \\\hline
		Koordinatensystem auf $\Gamma$ & kartesisch $(x,y,z)$ & Zylinderkoordinaten $(x,r,\varphi)$ \\
		Fourierraum                     & zweifachtransformiert $(k_x,\omega)$        & dreifachtransformiert$(k_x,n,\omega)$ \\\lasthline
	\end{tabular}
	\caption{Übersicht der Fourierräume und Koordinatensysteme von FEM und ITM.}
	\label{tab:Übersichtstabelle}
\end{table}\\

Die Übergangsbedingungen müssen folglich so formuliert werden, dass sich die Lösungen der Teilsysteme auf identische Größen beziehen.
Für die Kopplung sind demnach Transformationen an der Kopplungsoberfläche $\Gamma$ erforderlich. Gemäß \cite{Freisinger_Hackenberg2020} erfolgt die Transformation der Lösung des FEM-Teilsystems in das der ITM.

Zunächst werden die Verschiebungen des FEM-Teilsystems in das Zylinderkoordinatensystem transformiert.
Dafür greift \cite{Fruehe2010} erneut die Transformationsmatrix $\boldsymbol{\beta}$ aus Gleichung (\ref{eq:Transforationsmatrix}) auf und formuliert den Wechsel der Koordinatensysteme von kartesisch zu Zylinderkoordinaten wie folgt:
\begin{equation}\label{eq:u_FE_Zylinder}
	\tilde{\mathbf u}_{\Gamma,\mathrm{FEM,k}}}
	= \mathbf T_{1}\cdot
	\tilde{\mathbf u}_{\Gamma,\mathrm{FEM,z}}
\end{equation}
Die Einträge der Matrix $\mathbf{T_1}$ ergeben sich aus der transponierten Transformationsmatrix $\boldsymbol{\beta}^{\mathsf T}$ und sind im Anhang (\ref{sec:Koordinatentransformationsmatrix}) aufgeführt.

Im nächsten Schritt ist es erforderlich, $\tilde{\mathbf u}_{\Gamma_{\mathrm{FE,z}}}$ in den selben Fourierraum zu übertragen. Zu diesem Zweck wird eine Fourierreihe bezüglich des Umfangs der zylindrischen Kopplungsfläche $\Gamma$ entwickelt \citep{Hackenberg2016}:
\begin{equation}\label{eq:u_FE_Fourier}
	\tilde{\mathbf u}_{\Gamma,\mathrm{FEM,z}}
	= \mathbf T_{2}\cdot
	\hat{\mathbf u}_{\Gamma,\mathrm{FEM,z}}
\end{equation}
Die Einträge der Matrix \(\mathbf T_{2}\) sind im Anhang (\ref{sec:Fouriertransformationsmatrix}) aufgeführt.

Zusammenfassend lässt sich festhalten, dass durch die Anwendung der Transformationsmatrix \(\mathbf T\) eine Transformation auf einheitliche Koordinaten $(k_x,\ r,\ n,\ \omega)$ erzielt wird.
\begin{equation}\label{eq:tilde_u_FE}
	\tilde{\mathbf u}_{\Gamma,\mathrm{FEM},k}
	= \mathbf T_{1}\cdot\mathbf T_{2}\cdot\hat{\mathbf u}_{\Gamma,\mathrm{FEM},z}
	= \mathbf T\cdot\hat{\mathbf u}_{\Gamma,\mathrm{FEM},z}\,
\end{equation}
Analog dazu erfolgt die Transformation des Vektors der Belastungen $\tilde{\mathbf p}_{\Gamma,\mathrm{FEM,k}}$ in die Koordinaten des ITM-Teilsystems $(k_x,\ r,\ n,\ \omega)$ mifhilfe der Transformationsmatrix \(\mathbf T\) wie folgt:
\begin{equation}\label{eq:tildeP_FE}
	\tilde{\mathbf p}_{\Gamma,\mathrm{FEM,k}}
	= \mathbf T\cdot
	\hat{\mathbf p}_{\Gamma,\mathrm{FEM,z}} \,
\end{equation}



\section{Kopplung der Teilsysteme}
\label{sec:Kopplung}

Nachdem sich die Lösungen der Teilsysteme FEM und ITM nun auf die gleichen Koordinaten $(k_x,\ r,\ n,\ \omega)$ beziehen, können im weiteren Verlauf die Kopplungsbedingungen angewandt werden.

Die erste Kopplungsbedingung fordert, dass die Verschiebungen $\hat{\mathbf u}_{\Gamma}$ der Teilsysteme an der Kopplungsfläche $\Gamma$ identisch sind. Diese Kopplungsbedingung ergibt sich demnach zu:
\begin{equation}\label{eq:Kopplbed_u}
	\hat{\mathbf u}_{\Gamma,\mathrm{FEM}}
	= \hat{\mathbf u}_{\Gamma,\mathrm{ITM}}\,
\end{equation}
Darüber hinaus wird gefordert, dass die auftretenden Kräfte der Teillösungen sich im Gleichgewichtszustand mit potenziellen Randlasten an der Kopplungsfläche $\Gamma$ befinden. Die zweite Kopplungsbedingung lässt sich demnach wie folgt formuliern:
%Des Weiteren ist es essenziell, dass an der Kopplungsfläche $\Gamma$ die auftretenden Kräfte im einem Gleichgewichtszustand sind. Die Lasten der ITM und FEM müssen mit einer potenziellen Randlast auf der Kopplungsfläche $\Gamma$ im Gleichgewicht stehen. Die zweite Kopplungsbedingung lautet daher wie folgt: 
\begin{equation}\label{eq:Kopplbed_p}
	\hat{\mathbf p}_{\Gamma,\mathrm{ITM}}
	+ \frac{1}{\mathrm d s}\,\hat{\mathbf p}_{\Gamma,\mathrm{FEM}}
	= \hat{\mathbf p}_{\Gamma}\,
\end{equation}
Um die knotenweisen FEM-Lasten in stetige Spannungen entlang der Kopplungsfläche umzuwandeln, wird der Vektor der Knotenlasten des FEM-Teilsystems durch die Elementlänge $ds$ geteilt.


\cite{Hackenberg2016} kombiniert die Koplungsbedingungen (\ref{eq:Kopplbed_u}) und (\ref{eq:Kopplbed_p}) mit den Lösungen der Teilsysteme in den Gleichungen (\ref{eq:itm_block_system_braced}) und (\ref{eq:fe_block_system}), sodass das folgende System in Blockdarstellung resultiert:
\begin{equation}\label{eq:gekoppelt}
	\begin{bmatrix}
		\hat{\mathbf K}_{\Lambda\Lambda,\mathrm{ITM}} &
		\hat{\mathbf K}_{\Lambda\Gamma,\mathrm{ITM}}   &
		0 \\[6pt]
		\hat{\mathbf K}_{\Gamma\Lambda,\mathrm{ITM}}   &
		\hat{\mathbf K}_{\Gamma\Gamma,\mathrm{ITM}}
		+ \dfrac{1}{\mathrm d s}\,\mathbf T^{-1}\,\tilde{\mathbf K}_{\Gamma\Gamma,\mathrm{FEM}}\,\mathbf T &
		\dfrac{1}{\mathrm d s}\,\mathbf T^{-1}\,\tilde{\mathbf K}_{\Gamma\Omega,\mathrm{FEM}} \\[6pt]
		0 &
		\tilde{\mathbf K}_{\Omega\Gamma,\mathrm{FEM}}\,\mathbf T &
		\tilde{\mathbf K}_{\Omega\Omega,\mathrm{FEM}}
	\end{bmatrix}
	\begin{pmatrix}
		\hat{\mathbf u}_{\Lambda,\mathrm{ITM}}\\[2pt]
		\hat{\mathbf u}_{\Gamma}\\[2pt]
		\tilde{\mathbf u}_{\Omega,\mathrm{FEM}}
	\end{pmatrix}
	=
	\begin{pmatrix}
		\hat{\mathbf p}_{\Lambda,\mathrm{ITM}}\\[2pt]
		\hat{\mathbf p}_{\Gamma}\\[2pt]
		\tilde{\mathbf p}_{\Omega,\mathrm{FEM}}
	\end{pmatrix}
\end{equation}
Das Resultat ist ein System, das die beiden Teilsysteme ITM und FEM miteinander vereint. 
Die Größen werden in Blockdarstellung getrennt auf der Halbraumoberfläche $\Lambda$, der gemeinsamen Kopplungfläche $\Gamma$ sowie dem Inneren des FE-Bereichs $\Omega$ dargestellt.

Um eine auswertbare Endlösung des gekoppelten Systems zu erhalten, müssen die Größen mittels Fourierrücktransformation nach der Definition in der Gleichung (\ref{eq:invfouriertransformation}) aus dem Fourierraum in den Originalraum rücktransformiert werden.

%Um eine Endlösung des gekoppelten Systems zu erhalten, müssen die Größen mittels Inverser Fouriertransformation aus dem Fourierraum in den Originalraum zurückgeführt werden. Erst dann sind die Teilsysteme final auswertbar gekoppelt.