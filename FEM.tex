% !TeX root = a_main_thesis.tex
\chapter{Finite Elemente Methode}
\label{cha:FEM}

\section{Vorbemerkung}
\label{sec:FEM_Vorbemerkung}

Die ITM eignet sich zwar sehr gut für die Modellierung unendlicher Böden, für die Berechnung komplexerer Geometrien ist sie jedoch ungeeignet. Für die Geometrie im Tunnelbereich wird daher die Finite-Elemente-Methode (FEM) herangezogen.

Die Implementierung erfolgt dabei im fouriertransformierten Raum.
Dadurch wird der hohe Rechenaufwand reduziert, da die longitudinale Koordinate $x$ in den Wellenzahlbereich $k_x$ überführt wird \((x \mapto k_x\)). 
Somit kann der konstante Tunnelquerschnitt mittels finiter Elemente auf ein zweidimensionales Problem heruntergebrochen werden \citep{Fruehe2010}. 

Um im Anschluss die Lösungen der ITM- und FEM-Systeme zu einer Lösung des Gesamtsystems zu koppeln, ist zudem eine Teillösung im selben Fourier-Raum erforderlich. 
Daher findet eine zusätzliche Überführung der Zeit $t$ in den Frequenzbereich $\omega$ statt \((t \mapto \omega\)). 

Daher ist lediglich eine Diskretisierung in der y-z-Ebene erforderlich, die für jedes Wertepaar von $k_x$ und $\omega$ separat ausgewertet wird. 
Dieser reduzierte Ansatz wird als 2,5-dimensional bezeichnet \cite{Freisinger_Hackenberg2020}. 


\section{FE-Netz durch degenerierte Dreiecke}
\label{degenerierte Dreiecke}

Die zu untersuchende Geometrie Halbraum mit zylindrischem Hohlraum wurde bereits im gekoppelten ITM-FEM-Ansatz von \cite{Hackenberg2016} und \cite{Freisinger2022} untersucht.
In den vorliegenden Dissertationen wurde der Tunnel durch ein FE-Netz bestehend aus quadranguläre Elementen analysiert (verweis auf 4node FE Mesh darunter).
Die Implementation in \emph{MATLAB\texttrademark} dient dieser Arbeit als Grundlage.

..... hier von FE-mesh 4node einfügen .... 

Die vorliegende Arbeit verfolgt das Ziel , die Analyse des Tunnels weiterzuentwickeln. Zu diesem Zweck wird eine Umstellung des FE-Netzes von quadranguläre auf trianguläre Elemente vorgenommen.

Die Realisierung dieses Ziels könnte durch die Implementierung eines FE-Netzes mit vollständig neuen Elementen erfolgen. Nach diesem Ansatz wird durch den Code die Generierung von finiten Elementen mit drei Knotenpunkten initiiert.

Um eine möglichst hohe Ähnlichkeit zum Originalcode von \cite{Hackenberg2016} und \cite{Freisinger2022} zu gewährleisten, wurde sich für den Degenerationsansatz entschieden.

\cite{Zienkiewicz2013} präsentiert diesen Ansatz als eine Möglichkeit, lineare Dreiecke aus der Degeneration von linearen Vierecken zu gewinnen.
Hierbei werden zwei benachbarte Knoten eines 4-Knoten-Elements übereinandergelegt, sodass diese identische Koordinaten aufweisen. In der Folge degeneriert das Rechteckselement topologisch zu einem Dreieck (vgl Abbildung unten)

... Abbildung aus Zienkiewicz zeichnen --> aus 4node zu 3 node ...

Im Rahmen der Implementierung erfolgt eine Löschung der doppelten Information des vierten Knotens an entsprechender Stelle, sodass im weiteren Verlauf von einem triangulären Element mit drei Knoten ausgegangen werden kann.


Nach erfolgter Degenerierung liegen lineare, trianguläre Elemente mit drei Knotenpunkten vor, die jeweils drei Verschiebungsfreiheitsgrade \(u_x, u_y, u_z\) besitzen (verweis auf Abbildung drunter).

....2,5 dimensionales finites Element dreieck mit verschiebungen an Knoten angezeicchnet! -> Fig 4.2 Hackenberg ...

Diese Freiheitsgrade lassen sich in einen Knotenverschiebungsvektor $\tilde{\mathbf u}_{n}^{T}$ zusammenfassen:
\begin{equation}\label{eq:un_rowvec}
	\tilde{\mathbf u}_{n}^{T}
	=\bigl(\tilde{u}_{x1}\; \tilde{u}_{y1}\; \tilde{u}_{z1}\; \tilde{u}_{x2}\; \tilde{u}_{y2}\; \tilde{u}_{z2}\;
	\tilde{u}_{x3}\; \tilde{u}_{y3}\; \tilde{u}_{z3}\bigr).
\end{equation}
Mittels Ansatzfunktionen werden die Verschiebungsfelder innerhalb der Dreiecke interpoliert. Für degenerierte Dreiecke formuliert \cite{Zienkiewicz2013} die Ansatzfunktionen durch die Koordinaten $\eta,\ \zeta \in [-1,1]$ zu:
\begin{subequations}\label{eq:shape_functions}
	\begin{align}
		N_1 &= \frac{1}{4}\,(1-\zeta)(1-\eta), \label{eq:shape_funcs_a}\\
		N_2 &= \frac{1}{4}\,(1+\zeta)(1-\eta), \label{eq:shape_funcs_b}\\
		N_3 &= \frac{1}{2}\,(1+\eta).        \label{eq:shape_funcs_c}
	\end{align}
\end{subequations}
Es besteht demnach folgende Beziehung zwischen dem Verschiebungsfeld $\tilde{\bm u}$ und dem Knotenverschiebungsvektor $\tilde{\bm u}_{n}$:
\begin{equation}\label{eq:u_interp}
	\tilde{\bm u} \;=\; \mathbf N \tilde{\bm u}_{n}\,.
\end{equation}
Die Matrix \(\mathbf{N}\) enthält dabei die Ansatzfunktionen $N_1, N_2, N_3$ und lässt sich wie folgt formulieren:
\begin{equation}\label{eq:N_matrix}
	\mathbf N =
	\begin{bmatrix}
		N_1(\eta,\zeta) & 0   & 0   & N_2(\eta,\zeta) & 0   & 0   & N_3(\eta,\zeta) & 0   & 0 \\[-2pt]
		0   & N_1(\eta,\zeta) & 0   & 0   & N_2(\eta,\zeta) & 0   & 0   & N_3(\eta,\zeta) & 0 \\[-2pt]
		0   & 0   & N_1(\eta,\zeta) & 0   & 0   & N_2(\eta,\zeta) & 0   & 0   & N_3(\eta,\zeta)\\[-2pt]
	\end{bmatrix}
\end{equation}
Für die nachfolgende Berechnung werden sowohl die Matrix \(\mathbf{N}\) als auch die Matrix \(\bar{\mathbf{B}}\) herangezogen. Diese beinhaltet die Ableitungen der Ansatzfunktionen (\ref{eq:shape_functions}) im Wellenzahlbereich $k_x$, um Verzerrungen aus den Knotenverschiebungen zu berechnen \citep{Hackenberg2016}.
Die Einträge dieser Matrix sind im Anhang (\ref{sec:B_Matrix}) zu finden.



\section{Dynamische Steifigkeitsmatrix des 2,5D FEM-Teilsystems}
\label{sec:twofiveD_FEM}

Die Herleitung der Elementsteifigkeitsmatrix erfolgt gemäß dem Prinzip der virtuellen Arbeiten (PvA) im zweifach fouriertransformierten Wellenzahl-Frequenz-Raum. Dieses Prinzip besagt, dass die Summe aller virtuellen Arbeiten, die durch die im System wirkenden Kräfte verursacht werden, gleich null sein muss.
Für das elastische Kontinuum setzen sich die Arbeitsanteile aus der inneren virtuellen Arbeit $\delta W_i$, der virtuellen Arbeit infolge der d'Alembert'schen Trägheit $\delta W_T$ und der äußeren virtuellen Arbeit $\delta W_a$ zusammen \citep{Klein2003}:
\begin{equation}\label{eq:PvA}
	\delta W \;=\; \delta W_i + \delta W_T + \delta W_a \;=\; 0 \,.
\end{equation}
Da die Berechnung im \(k_x\)-\(\omega\)-Bereich erfolgt, müssen die virtuellen Arbeitsanteile im transformierten Raum formuliert werden. 
Im weiterem Verlauf werden die zweifach fouriertransformierten Größen mit der Notation $\tilde{}$ versehen.

Unter Berücksichtigung der Ansatzfunktionen (\ref{eq:shape_functions}) sowie der Verschiebungs-Verzerrungs-\\beziehung $\tilde{\varepsilon}=\bar{\mathbf B}\,\tilde{\mathbf u}_{n}$ und der Spannungs-Verzerrungs-Beziehung $\tilde{\sigma} = \mathbf D\,\tilde{\varepsilon}$, kann das Prinzip der virtuellen Arbeiten (\ref{eq:PvA}) nach \cite{Freisinger2022} wie folgt umformuliert werden:
\begin{equation}\label{eq:PvA2}
	-\delta \tilde{\mathbf u}_{n}^{\mathsf H}\,
	\underbrace{\left(\int\limits_{(A)} \bar{\mathbf B}^{\mathsf H}\,\mathbf D\,\bar{\mathbf B}\,\mathrm dA\right)}_{\bar{\mathbf{K}}}\,
	\tilde{\mathbf u}_{n}
	\;+\;
	\delta \tilde{\mathbf u}_{n}^{\mathsf H}\,
	\underbrace{\left(\int\limits_{(A)} \mathbf N^{\mathsf H}\,\tilde{\mathbf p}\,\mathrm dA\right)}_{\tilde{\mathbf p}_{n}}
	\;+\;
	\delta \tilde{\mathbf u}_{n}^{\mathsf H}\,\omega^{2}\,
	\underbrace{\left(\int\limits_{(A)} \rho\,\mathbf N^{\mathsf H}\,\mathbf N\,\mathrm dA\right)}_{\mathbf M}\,
	\tilde{\mathbf u}_{n}
	= 0 .
\end{equation}
Das Symbol \(^{(\mathsf H)}\) kennzeichnet dabei hermitesche Matrizen, die transponiert und anschließend komplex konjugiert wurden \citep{Hackenberg2016}.

In Gleichung (\ref{eq:PvA2}) sind zum einen die Matrizen $\mathbf{N}$ und $\bar{\mathbf B}$ enthalten, die sich aus den Ansatzfunktionen ergeben (Gleichugnen (\ref{eq:N_matrix}) und (\ref{eq:Bbar_tri3})). 
Außerdem findet sich auch der Knotenverschiebungsvektor $\tilde{\bm u}_{n}$ aus Gleichung (\ref{eq:u_interp}) wieder.
Des Weiteren ist die Elastizitätsmatrix $\mathbf{D}$ enthalten, welche sich ergibt zu:
\begin{equation}\label{eq:Elastizitätsmatrix}
	\mathbf D =
	\begin{bmatrix}
		\lambda+2\mu & \lambda      & \lambda      & 0 & 0 & 0 \\
		\lambda      & \lambda+2\mu & \lambda      & 0 & 0 & 0 \\
		\lambda      & \lambda      & \lambda+2\mu & 0 & 0 & 0 \\
		0            & 0            & 0            & \mu & 0   & 0 \\
		0            & 0            & 0            & 0   & \mu & 0 \\
		0            & 0            & 0            & 0   & 0   & \mu
	\end{bmatrix}
\end{equation}
Die Laméschen Konstanten $\lambda$ und $\mu$ wurden bereits in Gleichung (\ref{eq:lame_konstanten}) eingeführt und enthalten dabei die elastischen Größen.


Aus der Gleichung (\ref{eq:PvA2}) lassen sich zudem die Beziehungen für die Steifigkeitsmatrix \(\bar{\mathbf{K}}(k_x)\), den nodalen Lastvektor $\tilde{\mathbf p}_{n}$ und die Massenmatrix \(\mathbf{M}\) erkennen. 
Die Lösung dieser Integrale erfolgt numerisch im nachfolgenden Kapitel (\ref{sec:Numerik}).

Die Gleichung (\ref{eq:PvA2}) lässt sich mithilfe der gelösten Integrale in der für die FEM typischen Formulierung darstellen:
\begin{equation}\label{eq:PvA_FEM}
	\bar{\mathbf K}\,\tilde{\mathbf u}_{n}
	- \omega^{2}\,\bar{\mathbf M}\,\tilde{\mathbf u}_{n}
	\;=\;
	\underbrace{\bigl(\bar{\mathbf K}-\omega^{2}\,\bar{\mathbf M}\bigr)}_{\text{dyn.\ Steifigkeitsmatrix}}\,
	\tilde{\mathbf u}_{n}
	\;=\; \tilde{\mathbf p}_{n}\,.
\end{equation}
mit der dynamische Steifigkeitsmatrix $\tilde{\mathbf K}(k_x,\omega)=\bar{\mathbf K}-\omega^{2}\bar{\mathbf M}$.


Die Kopplung der beiden Teilsysteme FEM und ITM wird in Kapitel (\ref{cha:Kopplung}) durch die dynamischen Steifigketismatrizen der Berechnungsmethoden durchgeführt. \\
\cite{Hackenberg2016} formuliert dafür eine Matrix in Blockdarstellung, in deren System die Freiheitsgrade innerhalb des FE-Netzes $\Omega$ von denen an der Kopplungsfläche $\Gamma$ getrennt werden.

Zur besseren Nachvollziehbarkeit im Kopplungskapitel (\ref{cha:Kopplung}) werden die Vektoren und Matrizen mit dem Index FE versehen.
\begin{equation}\label{eq:fe_block_system}
	\begin{bmatrix}
		\tilde{\mathbf K}_{\Gamma\Gamma_{\mathrm{FE}}} & \tilde{\mathbf K}_{\Gamma\Omega_{\mathrm{FE}}} \\[6pt]
		\tilde{\mathbf K}_{\Omega\Gamma_{\mathrm{FE}}} & \tilde{\mathbf K}_{\Omega\Omega_{\mathrm{FE}}}
	\end{bmatrix}
	\begin{pmatrix}
		\tilde{\mathbf u}_{\Gamma_{\mathrm{FE}}} \\[2pt]
		\tilde{\mathbf u}_{\Omega_{\mathrm{FE}}}
	\end{pmatrix}
	=
	\begin{pmatrix}
		\tilde{\mathbf P}_{\Gamma_{\mathrm{FE}}} \\[2pt]
		\tilde{\mathbf P}_{\Omega_{\mathrm{FE}}}
	\end{pmatrix}
\end{equation}




\section{Numerische Implementation}
\label{sec:Numerik}

Im Implementierungsprozess werden die in Gleichung (\ref{eq:PvA2}) dargestellten Integrale der Steifigkeitsmatrix \(\bar{\mathbf{K}}\) und der Massenmatrix \(\mathbf M\) numerisch gelöst.

Für die numerishe Integration existieren mehrere Optionen. Es besteht die Möglichkeit, das Integrationsintervall in n äquisidistant angeordnete Abschnitte zu unterteilen und es anschließend über n+1 Stützstellen zu integrieren. Dieses Verfahren wird auch als Newton-Cotes-Quadratur bezeichnet und wird sowohl in \cite{Klein2003} als auch in \cite{Gross2023} vorgestellt.

Im Rahmen der Integration in der FEM findet in der Regel die Gauß´sche Quadraturformel aufgrund ihrer hohen Genauigkeit Anwendung, welche auf einen gewichteten Ansatz und optimierte Stützstellen zurückgreift.

In der von \cite{Hackenberg2016} entwickelten Implementierung werden dafür vier Gaußpunkte verwendet mit je 2 Gaußpunkte je Koordinatenrichtung




Sodass letzendlich die Integrale aus Gleichung (\ref{eq:PvA2}) numerisch lösen lassen zu:
\begin{subequations}\label{eq:KM_1D_gp}
	\begin{align}
		\mathbf{\bar K}
		&= \sum_{k=1}^{n_{\mathrm{GP}}}
		\mathbf{\bar B(k_x,\eta_k,\zeta_k)}^{\mathsf H}\,
		\mathbf{D'}\,
		\mathbf{\bar B(k_x,\eta_k,\zeta_k)}\,
		\det(\bm J)\,w_k \label{eq:K_1D_gp}\\[0.5ex]
		\mathbf{M}
		&= \sum_{k=1}^{n_{\mathrm{GP}}}
		\rho\,
		\mathbf{N(\eta_k,\zeta_k)}^{\mathsf H}\,
		\mathbf{N(\eta_k,\zeta_k)}\,
		\det(\bm J)\,w_k \label{eq:M_1D_gp}
	\end{align}
\end{subequations}

Wobei die Jacobi Matrix $\bm J$ die Transformation and Relation der Elementkoordinaten und globalen Koordinaten herstellt