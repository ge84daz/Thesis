\chapter{Grundlagen der Elastizitätstheorie}
\label{cha:Grundgleichungen}


\section{Vorbemerkung}
\label{sec:Vorbemerkung_Grundgleichungen}

Das Aufbringen dynamischer Belastungen auf ein Medium führt zur Entstehung von Spannungen und Verschiebungen, die sich im Erdreich in Form von mechanischen Wellen ausbreiten. 
Die folgende Betrachtung basiert auf der Annahme eines linear-elastischen Bodenmodells für das ein isotropes Materialverhalten vorausgesetzt wird. Wie \cite{Haupt1986} beschreibt, kann das Bodenmodell durch diese vereinfachte Betrachtung in guter Näherung beschrieben werden, da die resultierenden Formänderungen infolge der dynamischen Belastungen als gering einzustufen sind. 

Um mithilfe der in den Kapiteln (\ref{cha:ITM}) und (\ref{cha:FEM}) vorgestellten Berechnungsmethoden Systemantworten für Verschiebungen und Spannungen zu ermitteln, werden im Folgenden die Grundlagen der Elastizitätstheorie vorgestellt. 

\section{Lamésche Gleichung}
\label{sec:Lame}
Unter der Annahme eines homogenen, linear-elastischen Bodenmodells mit isotropem Materialverhalten beschreibt die Lamésche Differentialgleichung die wellenförmige Ausbreitung von Verschiebungen und Spannungen in einem dreidimensionalen Kontinuum infolge einer dynamischen Belastung. 

Wie in der Dissertation von \cite{Fruehe2010} beschrieben, wird diese Differentialgleichung in einem kartesischen Koordinatensystem mit Hilfe der Gleichgewichtsbedingung des Kontinuums und des Green-Lagrangeschen Verzerrungstensors hergeleitet. 

Gemäß \cite{Lame1852} ergibt sich dadurch ein System gekoppelter partieller Differentialgleichungen zu: 
	\begin{equation}\label{eq:lame}
	\mu u^{i}{}|_{j}^{\,j} + (\lambda + \mu) u^{j}{}|_{j}^{\,i} - \rho \ddot{u}^{i} = 0
	\end{equation}
Dabei enthalten die Laméschen Konstanten \(\mu\) und \(\lambda\) die elastischen Größen Schubmodul \(G\), Elastizitätsmodul \(E\) und Querdehnzahl \(\nu\), die sich wie folgt zusammensetzen:
\begin{subequations}\label{eq:lame_konstanten}
	\begin{align}
		\mu &= G = \frac{E}{2\,(1+\nu)}  \label{eq:lame_mu}\\
		\lambda &= \frac{E\,\nu}{(1+\nu)\,(1-2\nu)}  \label{eq:lame_lambda}
	\end{align}
\end{subequations}


\section{Satz von Helmholtz}
\label{sec:Helmholtz}
Zur Lösung der partiell gekoppelten Differentialgleichungen (\ref{eq:lame}) wird eine Entkopplung mittels des Satzes von Helmholtz angewendet.
Demzufolge lässt sich das Vektorfeld der Verschiebung \(u^{i}\) in die Summe eines skalaren $\Phi$ und eines vektoriellen Potentials $\Psi_{i}$ zerlegen, wie in Gleichung (\ref{eq:ui_phi_psi}) dargestellt \citep{Mueller2007}.
\begin{equation}\label{eq:ui_phi_psi}
	u^{i} = \Phi\mid^{i} + \Psi_{l}\mid_{k}\,\varepsilon^{ikl}
\end{equation}
Der Index ($\cdot_i$) steht dabei für die drei kartesischen Raumrichtungen \( i = x, y, z\).

Das Einsetzten der Gleichung (\ref{eq:ui_phi_psi}) in die Laméschen Differentialgleichungen (\ref{eq:lame}) führt zu den vier entkoppelten partiellen Differentialgleichungen:
\begin{subequations}\label{eq:phi_psi_wave}
	\begin{align}
		&\Phi\mid^{j}_{j} - \frac{1}{c_p^{2}}\ddot{\Phi} = 0 \label{eq:phi_wave}\\
		&\Psi_{i}\mid^{j}_{j} - \frac{1}{c_s^{2}}\ddot{\Psi}_{i} = 0 \label{eq:psi_wave}
	\end{align}
\end{subequations}
mit den Wellengeschwindigkeiten 
\begin{equation}\label{eq:cp_cs}
	c_p = \sqrt{\frac{\lambda + 2\mu}{\rho}} \quad \text{und} \quad
	c_s = \sqrt{\frac{\mu}{\rho}}\,
\end{equation}
 Eine ausführliche Herleitung ist in der Dissertation von \cite{Fruehe2010} zu finden.
 
 Als Ergebnis dieser Zerlegung erhält man ein System entkoppelter Potentiale, mit dem die Berechnung in den folgenden Kapiteln möglich ist.

Aus physikalischer Sicht lassen sich die Potentiale als verschiedene Wellentypen interpretieren.
Das skalare Potential $\Phi$ beschreibt dabei die Longitudinalwellen, auch P-Wellen oder Kompressionswellen genannt, die eine Volumenänderung bewirken und die Wellengeschwindigkeit $c_p$ besitzen.
Das vektorielle Potential $\Psi_{i}$ beschreibt hingegen die Transversalwellen, auch S-Wellen oder Scherwellen genannt, die mit einer Gestaltsänderung einhergehen und sich mit der Geschwindigkeit $c_s$ ausbreiten. Diese beiden Raumwellentypen treten im Boden auf \citep{Petersen2000}.

An der Bodenoberfläche erscheinen darüber hinaus Love- und Rayleigh-Wellen, die ebenfalls aus den Potentialen abgeleitet werden können. 
Im Vergleich zu den anderen Wellenarten sind die Rayleigh-Wellen besonders hervorzuheben, da sie im homogenen Halbraum den größten Anteil der Schwingungsenergie im Boden überträgt \citep{Haupt1986}.
Die Ausbreitungsgeschwindigkeit der Rayleigh-Wellen lässt sich mithilfe der Scherwellengeschwindigkeit $c_s$ zu $c_r = 0,92 \cdot c_s$ berechnen \citep{Petersen2000}:

Im Allgemeinen besteht bei der Wellenausbreitung der folgende Zusammenhang:
\begin{equation}\label{eq:lambda_cf}
	\lambda = \frac{c}{f}\,
\end{equation}




%durch den rotationsfreien Gradienten des Skalarfelds $\Phi$ und die quellfreien Rotation des Vektrofelds $\Psi_{l\,|k}\,^{ikl}$ zerlegen \cite{Lame1852}. Somit ergibt sich eine Zerlegung zu:

%\begin{equation}\label{eq:Helmholtz}
%	u^{i} &= \Phi\big|^{\,i} + \Psi_{l}\big|_{k}\,\varepsilon^{ikl}
%\end{equation}

%Durch Einsetzen von \ref{eq:Helmholtz} in \ref{eq:lame} zerfällt das gekoppelte Gleichungssystem in voneinander unabhängige Potentiale.

%Das skalare Potential beschreibt die Longitudinalwellen (P-Wellen), also Kompressionswellen, die eine Volumenänderung hervorrufen. Das vektorielle Potential hingegen beschreibt Transversalwellen (S-Wellen), die mit einer Gestaltsänderung einhergehen. 