\chapter{Grundlagen der Elastizitätstheorie}
\label{cha:Grundgleichungen}


\section{Vorbemerkung}
\label{sec:Vorbemerkung_Grundgleichungen}

Das Aufbringen dynamischer Lasten auf ein Medium führt zur Entstehung von Spannungen und Verschiebungen, die sich im Erdreich in Form von mechanischen Wellen ausbreiten. 
Die folgende Betrachtung basiert auf der Annahme eines linear-elastischen Bodenmodells für das ein isotropes Materialverhalten vorausgesetzt wird. Wie \cite{Haupt1986} beschreibt, kann das Bodenmodell durch diese vereinfachte Betrachtung in guter Näherung beschrieben werden, da die resultierenden Formänderungen infolge der dynamischen Lasten als gering einzustufen sind. 

Um mithilfe der in den Kapiteln (\ref{cha:ITM}) und (\ref{cha:FEM}) vorgestellten Berechnungsmethoden Systemantworten für Verschiebungen und Spannungen zu ermitteln, werden im Folgenden die zugrunde liegenden Grundlagen der Elastizitätstheorie vorgestellt. 

\section{Lamésche Gleichung}
\label{sec:Lame}
Unter den oben genannten Annahmen für das Bodenmodell beschreibt die Lamésche Differentialgleichung die wellenförmige Ausbreitung von Verschiebungen und Spannungen in einem dreidimensionalen Kontinuum infolge einer dynamischen Belastung. 

Wie in der Dissertation von \cite{Fruehe2010} beschrieben, werden diese partiellen Differentialgleichungen in einem kartesischen Koordinatensystem mit Hilfe der Gleichgewichtsbedingung des Kontinuums und des Green-Lagrangeschen Verzerrungstensors hergeleitet. 

Gemäß \cite{Lame1852} ergibt sich dadurch ein System gekoppelter partieller Differentialgleichungen zu: 
	\begin{equation}\label{eq:lame}
	\mu u^{i}{}_{|j}^{\,j} + (\lambda + \mu) u^{j}{}_{|j}^{\,i} - \rho \ddot{u}^{i} = 0
	\end{equation}
Dabei enthalten die Laméschen Konstanten \(\mu\) und \(\lambda\) auch die elastischen Konstanten Schubmodul \(G\), Elastizitätsmodul \(E\) und Querzahl \(\nu\), die sich wie folgt zusammensetzen:
\begin{subequations}\label{eq:lame_konstanten}
	\begin{align}
		\mu &= G = \frac{E}{2\,(1+\nu)}  \label{eq:lame_mu}\\
		\lambda &= \frac{E\,\nu}{(1+\nu)\,(1-2\nu)}  \label{eq:lame_lambda}
	\end{align}
\end{subequations}


\section{Satz von Helmholtz}
\label{sec:Helmholtz}
Zur Lösung der partiell gekoppelten Differentialgleichungen (\ref{eq:lame}) wird eine Entkopplung mittels des Satzes von Helmholtz angewendet.
Dem zufolge lässt sich das Vektorfeld der Verschiebung \(u_i \) in die Summe eines skalaren $\Phi$ und eines  vektorielles Potentials $\Psi_{i}$ zerlegen, wie beispielsweise in \cite{Mueller2007} beschrieben.
Der Index (\({}_i\)) steht dabei für die drei kartesischen Raumrichtungen (\( i = x, y, z.\)) und eine ausführliche Herleitung ist zudem in der Dissertation von \cite{Fruehe2010} zu finden.

Aus physikalischer Sicht lassen sich die Potentiale als verschiedene Wellentypen interpretieren.
Das skalare Potential $\Phi$ beschreibt dabei die Longitudinalwellen, auch P-Wellen oder Kompressionswellen genannt, welche eine Volumenänderung bewirken.
Das vektorielle Potential $\Psi_{i}$ beschreibt hingegen die Transversalwellen, auch S-Wellen genannt, die mit einer Gestaltsänderung einhergehen. Diese beiden Raumwellentypen treten im Boden auf.
An der Bodenoberfläche erscheinen darüber hinaus Love- und Rayleigh-Wellen, die ebenfalls aus den Potentialen abgeleitet werden können \citep{Haupt1986}.

Als Ergebnis dieser Zerlegung erhält man ein System entkoppelter Potentiale, mit dem die Berechnung in den folgenden Kapiteln möglich ist.




%durch den rotationsfreien Gradienten des Skalarfelds $\Phi$ und die quellfreien Rotation des Vektrofelds $\Psi_{l\,|k}\,^{ikl}$ zerlegen \cite{Lame1852}. Somit ergibt sich eine Zerlegung zu:

%\begin{equation}\label{eq:Helmholtz}
%	u^{i} &= \Phi\big|^{\,i} + \Psi_{l}\big|_{k}\,\varepsilon^{ikl}
%\end{equation}

%Durch Einsetzen von \ref{eq:Helmholtz} in \ref{eq:lame} zerfällt das gekoppelte Gleichungssystem in voneinander unabhängige Potentiale.

%Das skalare Potential beschreibt die Longitudinalwellen (P-Wellen), also Kompressionswellen, die eine Volumenänderung hervorrufen. Das vektorielle Potential hingegen beschreibt Transversalwellen (S-Wellen), die mit einer Gestaltsänderung einhergehen. 