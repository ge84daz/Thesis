% !TeX document-id = {0d492b7b-1b48-4a52-8f8e-b1e0cd78cc5d}
% !TeX TXS-program:compile = txs:///pdflatex/[--shell-escape]
\documentclass[ 
12pt,                  % Schriftgroesse 12pt
a4paper,               % Layout fuer Din A4
twoside,               % Layout fuer beidseitigen Druck
headinclude,           % Kopfzeile wird Seiten-Layouts mit beruecksichtigt
headsepline,           % horizontale Linie unter Kolumnentitel
plainheadsepline,      % horizontale Linie auch beim plain-Style
BCOR=12mm,             % Korrektur fuer die Bindung
DIV=18,                % DIV-Wert fuer die Erstellung des Satzspiegels, siehe scrguide
parskip=half,          % Absatzabstand statt Absatzeinzug
openany,               % Kapitel können auf geraden und ungeraden Seiten beginnen
bibliography=totoc,    % Literaturverz. wird ins Inhaltsverzeichnis eingetragen
toc=listof,            % Abbildungs- Tabellen- und sonstige verzeichnisse mit ins INhaltsverz.
numbers=noenddot,      % Kapitelnummern immer ohne Punkt
captions=tableheading, % korrekte Abstaende bei TabellenUEBERschriften
fleqn,                 % fleqn: Gleichungen links (statt mittig)
% draft                % Keine Bilder in der Anzeige, overfull hboxes werden angezeigt
% To be edited by the author:
% german,              % deutsche Sprache, global
% bmcolorlinks,
% bmshowlabels
% mpinclude,           % mehr Platz für Randnotzzen
]{bmvorlagen/bmthesis}



%%%%%%%%%%%%%%%%%%%%%%%%%%%%%%%%%%%
% Packages/Settings
%%%%%%%%%%%%%%%%%%%%%%%%%%%%%%%%%%%
\usepackage[utf8]{inputenc}  					 % Input-Encodung: ansinew fuer Windows
\usepackage{amsmath}
\usepackage[ngerman]{babel}  % für deutsche Sprache; von A eingefügt
\usepackage[T1]{fontenc}

% --- svg Dateien ---
%\usepackage{svg}          % braucht Inkscape
%\svgpath{{Bilder_svg/}}   % Suchpfad(n)
%\graphicspath{{bilder/}{bilder_svg/}{plots/}}    % Falls texinput nicht gesetzt -> Bildverzeichnisse
\usepackage{svg}
\usepackage{subcaption}


\usepackage{array,longtable}
%\newcolumntype{L}[1]{>{\raggedright\arraybackslash}p{#1}}
% --- Mac: Pfad zu Inkscape setzen ---
% Variante A: Inkscape aus der .app (meist richtig)
\svgsetup{
	inkscapeexe = {/Applications/Inkscape.app/Contents/MacOS/inkscape},
	inkscapelatex = true % Text via LaTeX setzen (empfohlen)
}

% (optional) Variante B: Homebrew-Installation
% \svgsetup{inkscapeexe = {/opt/homebrew/bin/inkscape}, inkscapelatex=true} % Apple Silicon
% \svgsetup{inkscapeexe = {/usr/local/bin/inkscape},  inkscapelatex=true}   % Intel-Macs

% Suchpfad(e) für deine SVGs (passt zu deiner Struktur)
\svgpath{{svg/}}

% --- Bibliographie ---
%\usepackage[backend=biber, style=numeric]{biblatex}
%\addbibresource{Literaturverzeichnis.bib}  % deine Bib-Datei ohne Pfad und Endung
%\usepackage[backend=biber,style=ieee]{biblatex}
%\addbibresource{Literaturverzeichnis.bib}  % Jabref referenz hier!!!

%\usepackage[backend=biber,style=authoryear,natbib=true]{biblatex}
%\addbibresource{Literaturverzeichnis.bib}



%\usepackage[backend=biber,style=ieee]{biblatex}
%\addbibresource{Literaturverzeichnis.bib}  % Jabref referenz hier!!!
%\usepackage[latin1]{inputenc}    				 % Input-Encodung: latin1 fuer Unix
\newcommand{\bmtitle}{Implementierung eines triangulären FE-Netzes im ITM-FEM-Ansatz zur Untersuchung eines homogenen Halbraums mit zylindrischem Hohlraum}
\newcommand{\bmauthor}{Anton Bönisch}
\newcommand{\bmkeywords}{Thesis, Example}
\newcommand{\bmstartpage}{5}
\graphicspath{{bilder/}{bilder_svg/}{plots/}}    % Falls texinput nicht gesetzt -> Bildverzeichnisse
\hyphenation{Post-pro-cess-ing--In-te-gral}
\raggedbottom



\newcommand{\iu}{\mathrm{i}}   % imaginäre Einheit i aufrecht
\newcommand{\eu}{\mathrm{e}}   % Eulersche Zahl e aufrecht
%%%%%%%%%%%%%%%%%%%%%%%%%%%%%%%%%%%
% Setting for new version of svg-Package
% to handle file names as the old one did
% fixing a bug
% date 04.11.2019
\makeatletter
\def\set@curr@file#1{\def\@curr@file{#1}}
\makeatother
\usepackage{tikz}
\newcommand{\ftmap}{%
	\tikz[baseline=-0.5ex]{
		\draw[semithick] (0,0) circle(0.7ex);
		\draw[semithick] (0,0) -- (1.8em,0);
		\fill (1.8em,0) circle(0.7ex);
	}%
}

%Fouriertransformationen:
\newcommand{\mapto}{\mathrel{\laplace}}
\newcommand{\mapfrom}{\mathrel{\Laplace}}

%\newcommand{\legline}[1]{\tikz[baseline=-0.6ex]\draw[#1,line width=1.2pt] (0,0)--(1.6em,0);}

%Farben Kennzeichnungen
\usepackage{xcolor,tikz}
\definecolor{MatlabBlue}{rgb}{0,0.4470,0.7410}
\definecolor{MatlabOrange}{rgb}{0.8500,0.3250,0.0980}
\DeclareRobustCommand{\legline}[1]{%
	\tikz[baseline=-0.6ex]\draw[#1,line width=1.4pt] (0,0)--(1.8em,0);%
}
\DeclareRobustCommand{\legFour}{\legline{MatlabBlue}}           % blaue Volllinie
\DeclareRobustCommand{\legThree}{\legline{MatlabOrange,dashed}} % orange gestrichelt

%Tabellen
\usepackage{array,ragged2e,tabularx}
% Feste Symbolspalte in Math-Mode, rechtsbündig
\newcolumntype{Sym}[1]{>{\RaggedLeft\arraybackslash$\displaystyle}m{#1}<{$}}
% Beschreibungsspalte mit schönerem Flattersatz
\newcolumntype{Desc}[1]{>{\RaggedRight\arraybackslash}p{#1}}

\usepackage{trfsigns}
%%%%%%%%%%%%%%%%%%%%%%%%%%%%%%%%%%%

%%%%%%%%%%%%%%%%%%%%%%%%%%%%%%%%%%%
% Dokument an sich
%%%%%%%%%%%%%%%%%%%%%%%%%%%%%%%%%%%
% Wenn man nur ein kapitel übersetzten möchte, da reinschreiben
%\includeonly{}
%%%%%%%%%%%%%%%%%%%%%%%%%%%%%%%%%%%%%%%%%%%%%%%%%%%%%%%%%%%%%%%



\begin{document}
	\frontmatter
	\pagenumbering{Roman}
	
	\ifthenelse{\boolean{bmgerman}}
{
\begin{titlepage}
%\layout

\begin{center}
{
\fontsize{18}{18}\selectfont 

\includegraphics*[width=3cm, keepaspectratio=true]{bmvorlagen/logos/tumlogo} \hfill \includegraphics*[width=2.5cm, keepaspectratio=true]{bmvorlagen/logos/bmlogo}
\vspace{0.5cm}
\hrule

\vspace{1cm}
Lehrstuhl für Baumechanik\\
Technische Universität München\\

\vspace{1.4cm}

\vspace{2cm}


\vspace{3mm}



\bmtitle \\[5ex]


\bmauthor \\[7ex]


Masterarbeit im Studiengang Bauingenieurwesen\\
Vertiefungsrichtung Baumechanik\\

\vspace{1.8cm}

{\fontsize{12pt}{12} \selectfont%
\begin{tabular}{rll}
Referent&:& Univ.-Prof. Dr.-Ing. Gerhard Müller\\
& & Dr.-Ing. Francesca Taddei\\[0.5ex]
Betreuer&:& M.Sc. XXX\\[0.5ex]
Eingereicht&:& \today
\end{tabular}
}
                  

}
\end{center}

\end{titlepage}
}
{
\begin{titlepage}
	%\layout
	
	\begin{center}
		{
			\fontsize{18}{18}\selectfont 
			
			\includegraphics*[width=3cm, keepaspectratio=true]{bmvorlagen/logos/tumlogo} \hfill \includegraphics*[width=2.5cm, keepaspectratio=true]{bmvorlagen/logos/bmlogo}
			\vspace{0.5cm}
			\hrule
			
			\vspace{1cm}
			Chair of Structural Mechanics\\
			Technical University of Munich\\
			
			\vspace{1.4cm}
			
			\vspace{2cm}
			
			
			\vspace{3mm}
			
			
			
			\bmtitle \\[5ex]
			
			
			\bmauthor \\[7ex]
			
			
			Masterthesis in the study program civil engineering\\
			Focus area Structural Dynamics\\
			
			\vspace{1.8cm}
			
			{\fontsize{12pt}{12} \selectfont%
				\begin{tabular}{rll}
					Editor&:& Univ.-Prof. Dr.-Ing. Gerhard Müller\\
					& & Dr.-Ing. Francesca Taddei\\[0.5ex]
					Supervisor&:& M.Sc. XXX\\[0.5ex]
					Handed in&:& \today
				\end{tabular}
			}
			
			
		}
	\end{center}
	
\end{titlepage}
}

                % Deckblatt
	\cleardoubleemptypage 
	\addchap{Abstract}
\label{cha:abtract} 

An abstract is a greatly condensed version of a longer piece of writing that highlights the major points covered, and concisely describes the content and scope of the writing. 

Abstracts give readers a chance to quickly see what the main contents and sometimes methods of a piece of writing are. They enable readers to decide whether the work is of interest for them. Using key words in an abstract is important because of today's electronic information systems. A web search will find an abstract containing certain key words.  

Keywords: 
\begin{itemize}
 \item Keyword 1
 \item Keyword 2
\end{itemize}
            % Abstract der Arbeit
	\addchap{Erklärungen}
\markboth{Erklärungen}{Erklärungen}
\label{cha:erkl}

\subsubsection{Danksagung}
An erster Stelle möchte ich mich bei Herrn Professor Dr.-Ing. Gerhard Müller sowie dem gesamten Lehrstuhl für Baumechanik dafür bedanken, dass ich meine Bachelorarbeit dort schreiben durfte. 

Bei meinem Betreuer Tom Hicks möchte ich mich ebenfalls bedanken

\subsubsection{Erklärung}
Hiermit versichere ich, die vorliegende Arbeit selbstständig und ohne fremde Hilfe angefertigt zu haben. Die verwendete Literatur und sonstige Hilfsmittel sind vollständig angegeben.
\vfill
München, \today \\
\vspace{3cm} \\
Anton Bönisch
\vfill

	\cleardoubleemptypage         % Das Inhaltsverzeichnis auf einer rechten Seite beginnen
	
	\begin{spacing}{1.0}          % Verzeichnisse werden mit einzeiligem Abstand gesetzt
		\tableofcontents             % Inhaltsverzeichnis
		% \addcontentsline{toc}{chapter}{Abbildungsverzeichnis} so kann man von hand was in Inhaltsverzeichnis schreiben
		\listoffigures              % Abbildungsverzeichnis
		\listoftables               % Tabellenverzeichnis
		% \lstlistoflistings          % Verzeichnis der Listings
	\end{spacing}
	
	\addchap{Symbolverzeichnis}
\markboth{Symbolverzeichnis}{Symbolverzeichnis}
\label{cha:symbolverzeichnis}

Der Zusatz $\hat{\medspace}$ bezeichnet eine komplexe Größe.

\section*{Griechische Buchstaben}
\begin{longtable}[l]{lcp{8cm}l}
$\Gamma$ & & Berandung des Problems \\
$\delta\left(\left|\vec{x}_{0} - \vec{x}\right|\right)$ & & Dirac-Funktion am Punkt $\vec{x}$ \\
\end{longtable}

\section*{Lateinische Buchstaben}
\begin{longtable}[l]{lcp{8cm}l}
%\hspace*{2.5cm}\= \hspace*{1cm} \=  Schallschnelle senkrecht zu einer Oberfläche \= \kill
$c$ & \einheit{\frac{m}{s}} & Schallgeschwindigkeit & $c = \frac{E}{\rho}$ \\
$E$ & \einheit{\frac{N}{m^{2}}} & Elastizitätsmodul\\
$f$ & \einheit{\frac{1}{s}} & Frequenz & $f = \frac{\omega}{2 \pi}$ \\ 
\end{longtable}             % Symbolverzeichnis  nur temporär ausgeblendet von Anton
	
	\cleardoubleplainpage         % Das erste Kapitel des Hauptteils auf einer rechten Seite beginnen
	
	\mainmatter                   % den Hauptteil beginnen
	\chapter{Einleitung}
\label{kap:einleitung}

\section{Ausgangssituation}
\label{sec:ausgang}

             % Einleitung
	%\chapter{Testchapter}
\label{cha:beispiele}

\section{How to ...}
\label{sec:chief}

\subsection{Citations}
\label{sec:citations}
The usual citation should look like \citep{Stroud1966}.

Additionally one can extend both directions \citep[before][after p.1-3]{Stroud1966}: 
\citep{Unknown2018}

\citep{Unknown2018}

\citep{Hackenberg2016}

\cite{Fruehe2010}


\subsection{Equations}
\label{sec:equations}
In the following different typs of equations ar shown:

Equation:
%
\begin{equation}
		\int\limits_{(\Gamma)} \left( i \rho \omega \hat{v}_{n} \hat{G}\right)~d\Gamma  = - C(\mathbf{P}) \cdot \hat{p}(\mathbf{P})  + \int\limits_{(\Gamma)} \left(\hat{p} \frac{\partial \hat{G}}{\partial \vec{n}}\right)~d\Gamma.
	\label{eq:chief1}
\end{equation}

Please take care that there is no empty line before the equation because \LaTeX will interpret this as a new paragraph. Whether or not there is a free line (paragraph) after an equation is up to the author.

Gather:
\begin{gather}
\left[(\lambda+2\mu)\phi|^j_j-\rho\ddot{\Phi}\right]|^i+\left[\mu\Psi_l|_j^j-\rho\ddot{\Psi}_l\right]|_k\epsilon^{ikl}=0
\end{gather}

Subequations:
\begin{subequations}\label{eq:Wellengleichung_FT}
\begin{align}
		 \left[-k_x^2 - k_y^2 + k_p^2 +\frac{\partial ^2}{\partial z^2}\right]\hat{\Phi} \;(k_x,k_y,z,\omega) &= 0 \label{eq:2.5}\\[10pt]
		 \left[-k_x^2 - k_y^2 + k_s^2 +\frac{\partial ^2}{\partial z^2}\right]\hat{\Psi}_i(k_x,k_y,z,\omega) &= 0 \label{eq:2.6}
\end{align}
\end{subequations}
mit Kompressionswellenzahl $k_p$ und Scherwellenzahl $k_s$.

Matrices with different brackets:
\begin{equation}
\begin{pmatrix}
\hat u_x \\
\hat u_y\\
\hat u_z
\end{pmatrix}
%
=
% 
\begin{bmatrix}
i k_x & 0 & - \frac{\partial}{\partial z} \\
i k_y & \frac{\partial}{\partial z} & 0\\
\frac{\partial}{\partial z} & -i k_y & i k_x
\end{bmatrix}
%
\begin{pmatrix}
\hat \Phi\\
\hat \Psi_x\\
\hat \Psi_y
\end{pmatrix}
%
\label{eq_027} 
\end{equation}

Align $\rightarrow$ multiple equations with numeration each:
\begin{align}
	\hat{\sigma}_{zx}(k_x,k_y,z=0,\omega) &=-\hat{p}_{zx}(k_x,k_y,\omega) \\[10pt]
		 \hat{\sigma}_{zy}(k_x,k_y,z=0,\omega) &=-\hat{p}_{zy}(k_x,k_y,\omega)
\end{align}

Aligned $\rightarrow$ multiple equations with one number:
\begin{equation}\label{eq:RB_OF}
	\begin{aligned}
		 \hat{\sigma}_{zx}(k_x,k_y,z=0,\omega) &=-\hat{p}_{zx}(k_x,k_y,\omega) \\[10pt]
		 \hat{\sigma}_{zy}(k_x,k_y,z=0,\omega) &=-\hat{p}_{zy}(k_x,k_y,\omega) \\[10pt]
		 \hat{\sigma}_{zz}(k_x,k_y,z=0,\omega) &=-\hat{p}_{zz}(k_x,k_y,\omega) 
		\end{aligned}
\end{equation}




\subsection{Self written functions for easy writing}
\label{sec:helpers}

\begin{tabbing}
\hspace*{4cm}\=\hspace*{3cm}\= \\
	 units: \> \einheit{kg} oder \einheit{\frac{kg}{m^2}} (works also in Math-environment)\\
	 birth and death: \> H. A. Schenck \lived{1901}{1970}\\
	 vectors \> $\mathbf{g}$ bold symbol for vector. Usable in text or math enviroment.\\
	 matrices \> $\mathbf{G}$ brackets around Matrix name. Usable in text or math enviroment.\\
\end{tabbing}

\subsection{Possible options in the documentclass bmthesis}

\begin{description}
	\item[bmcolorlinks:]  makes hyperlinks in the pdf colourful. Please skip / delete for the printed version of the theses.
	\item[bmshowlabels:] prints the labels. Please skip / delete for the printed version of the theses.
	\item[german:] used for german language. Has influence on hyphenation, titles and names, style of the bibliography... . For english thesis': skip it.
\end{description}


\section{Some more examples}
\label{sec:more_ex}
\subsection{Table}
\label{sec:table}
Eine einfache Tabelle:

\begin{table}[htb]
	\centering
		\begin{tabular}{ccc} \firsthline
	   &  $ka$ & f [Hz] \\\hline
		$1$ & $\pi$ & $171,5$ \\
		$2$ & $2\pi$ & $343$ \\
		$3$ & $3\pi$ & $514,5$ \\\lasthline
		\end{tabular}
	\caption[Die ersten drei symmetrischen Eigenwerte für das innere Problem des Kugelkörpers]{}
	\label{tab:tabelle1}
\end{table}

Die ersten Kugelflächenfunktionen lauten:
\begin{table}[H]
	\centering
	\begin{small}
	\renewcommand*{\arraystretch}{1.0}
		\begin{tabular}{l||l|l|l|l}
    $Y_m^l(\vartheta, \varphi)$ & $l=0$ & $l=1$ & $l=2$ & $l=3$\\
    \hline \hline
		$m=-3$ &  &  & & $\;\;\:\sqrt{\frac{35}{64 \pi}} \;\sin^{3}{\vartheta}\;e^{-3i \varphi}$\\
		\hline
    $m=-2$ &  &  & $\;\;\:\sqrt{\frac{15}{32\pi}}\;\sin^2{\vartheta}\;e^{-2i\varphi}$ & $\;\;\:\sqrt{\frac{105}{32\pi}} \;\sin^{2}{\vartheta}\cos{\vartheta}\;e^{-2i \varphi}$ \\
		\hline
		$m=-1$ &  & $\;\;\:\sqrt{\frac{3}{8\pi}}\;\sin{\vartheta}\;e^{-i\varphi}$  & $\;\;\:\sqrt{\frac{15}{8\pi}}\;\sin{\vartheta}\;\cos{\vartheta}\;e^{-i\varphi}$ & $\;\;\:\sqrt{\frac{21}{64 \pi}} \;\sin{\vartheta}\left( 5 \cos^{2}{\vartheta} - 1\right)\;e^{-i \varphi}$\\
		\hline
		$m=\;\;\:0$ & $\;\;\:\sqrt{\frac{1}{4\pi}}$ & $\;\;\:\sqrt{\frac{3}{4\pi}}\;\cos{\vartheta}$ & $\;\;\:\sqrt{\frac{5}{16\pi}}\;\left(3\cos^2{\vartheta}-1\right)$ & $\;\;\:\sqrt{\frac{7}{16 \pi}} \; \left( 5 \cos^{3}{\vartheta} - 3 \cos{\vartheta}\right)$\\
		\hline
		$m=\;\;\:1$ &  & $-\sqrt{\frac{3}{8\pi}}\;\sin{\vartheta}\;e^{i\varphi}$ & $-\sqrt{\frac{15}{8\pi}}\;\sin{\vartheta}\;\cos{\vartheta}\;e^{i\varphi}$ & $-\sqrt{\frac{21}{64\pi}} \sin{\vartheta}\left( 5 \cos^{2}{\vartheta} - 1\right)\;e^{i \varphi} $\\
		\hline
		$m=\;\;\:2$ &  &  & $\;\;\:\sqrt{\frac{15}{32\pi}}\;\sin^2{\vartheta}\;e^{2i\varphi}$ & $\;\;\:\sqrt{\frac{105}{32 \pi}}\; \sin^{2}{\vartheta}\cos{\vartheta}\;e^{2i \varphi}$\\
		\hline
		$m=-3$ &  &  & & $-\sqrt{\tfrac{35}{64 \pi}}\; \sin^{3}{\vartheta}\;e^{3i \varphi}$
		\end{tabular}
		\end{small}
	\caption{Kugelflächenfunktionen für $l=0, 1, 2, 3$ und zugehörige $l=-m, ..., m$}
	\label{tab:Kugelflächenfunktionen}
\end{table}

\subsection{Aufzählungen}

Die itemize Umgebung kann in sich selbst bis zu vier Ebenen tief geschachtelt werden. 

\begin{itemize}
\item erste Ebene
\begin{itemize}
\item zweite Ebene
\begin{itemize}
\item dritte Ebene
\begin{itemize}
\item vierte Ebene
\end{itemize}
\end{itemize}
\end{itemize}
\end{itemize}

Die Ausgabe der Label kann verändert werden. Am Anfang ein Beispiel für die Verwendung der Option des item Befehls. item[Option] Hier kann ein Label als Option eingestellt werden. 

\begin{itemize}
\item[a)] Ein Stichpunkt
\item[*)] Noch ein Stichpunkt
\end{itemize}

Die enumerate Umgebung in LATEX stellt eine nummerierte Auflistung zur Verfügung. 

\begin{enumerate}
\item erstes 
\item zweites 
\end{enumerate}

Standardmäßig erfolgt die Nummerierung auf der ersten Ebene mit arabischen Ziffern/Zahlen., auf der zweiten Ebene mit (kleiner lateinischer Buchstabe), auf der dritten Ebene mit kleinen römischen Ziffern/Zahlen. und auf der vierten Ebene mit großen lateinischen Buchstaben.. 

\begin{enumerate}
\item erste Ebene
\begin{enumerate}
\item zweite Ebene
\begin{enumerate}
\item dritte Ebene
\begin{enumerate}
\item vierte Ebene
\end{enumerate}
\item wieder auf dritter Ebene 
\item noch ein Eintrag 
\end{enumerate}
\item hier ist die zweite Ebene
\end{enumerate}
\item und hier die erste Ebene
\end{enumerate}



\section{Including graphics}
\label{sec:figs}

\subsection{Include SVGs}
\begin{figure}[H]
	\begin{center}
		\includesvg[width=0.9\textwidth]{svg/Helmholtz}
	\end{center}
	\caption{Zerlegung des Verschiebungsfeldes mit Satz von Helmholtz vlg. \citep{Mueller1989}}
	\label{abb:Satz_von_Helmholtz}
\end{figure}

\subsection{Include PNGs and PDFs}

\begin{figure}[h]
	\centering
		\includegraphics[width=0.6\textwidth]{bilder/Assoziierte_Legendre.pdf}
	\caption{Assoziierte Legendre Funktionen Pdf}
	\label{fig:Assoziierte_Legendre}
\end{figure}

\begin{figure}[H]
	\centering
		\includegraphics[width=0.8\textwidth]{bilder/Latex_logo.png}
	\label{fig:Latex_Logo}
	\caption{Latex Logo PNG}
\end{figure}

\subsection{Figures with subfigures}
Figures are defined as floating structures. See \citep[][page 91]{Sturm2010}.


\begin{figure}[H]
\centering%
\begin{subfigure}[c]{0.49\textwidth}
                \includegraphics[width=7.6cm, keepaspectratio=true]{Assoziierte_Legendre.pdf} \label{abb:315_re}
                \subcaption{Assoziierte Legendre Funktionen Pdf}
\end{subfigure}
\begin{subfigure}[c]{0.49\textwidth}
\includegraphics[width=7.6cm, keepaspectratio=true]{Assoziierte_Legendre.pdf} \label{pic:497_re}
\subcaption{Assoziierte Legendre Funktionen Pdf}
\end{subfigure}
                \caption{Assoziierte Legendre Funktionen Pdf}
                \label{abb:bild1-2}
\end{figure}


\newpage 
\subsection{Code}
\label{sec:code}

% ----------------------------	
% this command replaces the letters ü and so on with its unicode equivalents to be available in the listings environment
	\lstset{literate=%
  {Ö}{{\"O}}1
  {Ä}{{\"A}}1
  {Ü}{{\"U}}1
  {ß}{{\ss}}1
  {ü}{{\"u}}1
  {ä}{{\"a}}1
  {ö}{{\"o}}1
   }
% ----------------------------	

\begin{lstlisting}[style=matlab,
  label=lst:prozess,
  firstnumber=1,
  float={!htb},
  caption={Ein Matlab-Programm}
  ]
  
%Übungsbeispiel Volumenelemente
%by Martin Buchschmid und Siegfried Seipelt

clear all;

femesh('reset');

%Deklaration der benötigten Startknoten
FEnode=[1  0 0 0  0 0 0 ;
        2  0 0 0  0 0.05 0 ;
        3  0 0 0  0 1 0];%nicht verwendete Knoten stellen kein Problem dar
         
%Vorgabe der zunächst leeren Felder für die FE-Elementierung															   
FEelt=[];
FEel0=[];

%Erzeugen eines Balkenelementes zwischen den Knoten 1 und 2
femesh('objectbeamline 1 2');

%Extrudieren des Grundelementes jeweils 10-fach in die Richtungen x=1 und dann z=1 (Die zuerst erzeugte Fläche wird so zu einem Volumenelement)
femesh('extrude 80 0.05 0 0');
femesh('extrude 80 0 0 0.05');
femesh('repeatsel 4 0 0.05 0');


\end{lstlisting}
	
	
	\chapter{Grundlagen der Elastizitätstheorie}
\label{cha:Grundgleichungen}


\section{Vorbemerkung}
\label{sec:Vorbemerkung_Grundgleichungen}

Das Aufbringen dynamischer Belastungen auf ein Medium führt zur Entstehung von Spannungen und Verschiebungen, die sich im Erdreich in Form von mechanischen Wellen ausbreiten. 
Die folgende Betrachtung basiert auf der Annahme eines linear-elastischen Bodenmodells für das ein isotropes Materialverhalten vorausgesetzt wird. Wie \cite{Haupt1986} beschreibt, kann das Bodenmodell durch diese vereinfachte Betrachtung in guter Näherung beschrieben werden, da die resultierenden Formänderungen infolge der dynamischen Belastungen als gering einzustufen sind. 

Um mithilfe der in den Kapiteln (\ref{cha:ITM}) und (\ref{cha:FEM}) vorgestellten Berechnungsmethoden Systemantworten für Verschiebungen und Spannungen zu ermitteln, werden im Folgenden die Grundlagen der Elastizitätstheorie vorgestellt. 

\section{Lamésche Gleichung}
\label{sec:Lame}
Unter der Annahme eines homogenen, linear-elastischen Bodenmodells mit isotropem Materialverhalten beschreibt die Lamésche Differentialgleichung die wellenförmige Ausbreitung von Verschiebungen und Spannungen in einem dreidimensionalen Kontinuum infolge einer dynamischen Belastung. 

Wie in der Dissertation von \cite{Fruehe2010} beschrieben, wird diese Differentialgleichung in einem kartesischen Koordinatensystem mit Hilfe der Gleichgewichtsbedingung des Kontinuums und des Green-Lagrangeschen Verzerrungstensors hergeleitet. 

Gemäß \cite{Lame1852} ergibt sich dadurch ein System gekoppelter partieller Differentialgleichungen zu: 
	\begin{equation}\label{eq:lame}
	\mu u^{i}{}|_{j}^{\,j} + (\lambda + \mu) u^{j}{}|_{j}^{\,i} - \rho \ddot{u}^{i} = 0
	\end{equation}
Dabei enthalten die Laméschen Konstanten \(\mu\) und \(\lambda\) die elastischen Größen Schubmodul \(G\), Elastizitätsmodul \(E\) und Querdehnzahl \(\nu\), die sich wie folgt zusammensetzen:
\begin{subequations}\label{eq:lame_konstanten}
	\begin{align}
		\mu &= G = \frac{E}{2\,(1+\nu)}  \label{eq:lame_mu}\\
		\lambda &= \frac{E\,\nu}{(1+\nu)\,(1-2\nu)}  \label{eq:lame_lambda}
	\end{align}
\end{subequations}


\section{Satz von Helmholtz}
\label{sec:Helmholtz}
Zur Lösung der partiell gekoppelten Differentialgleichungen (\ref{eq:lame}) wird eine Entkopplung mittels des Satzes von Helmholtz angewendet.
Demzufolge lässt sich das Vektorfeld der Verschiebung \(u^{i}\) in die Summe eines skalaren $\Phi$ und eines vektoriellen Potentials $\Psi_{i}$ zerlegen, wie in Gleichung (\ref{eq:ui_phi_psi}) dargestellt \citep{Mueller2007}.
\begin{equation}\label{eq:ui_phi_psi}
	u^{i} = \Phi\mid^{i} + \Psi_{l}\mid_{k}\,\varepsilon^{ikl}
\end{equation}
Der Index ($\cdot_i$) steht dabei für die drei kartesischen Raumrichtungen \( i = x, y, z\).

Das Einsetzten der Gleichung (\ref{eq:ui_phi_psi}) in die Laméschen Differentialgleichungen (\ref{eq:lame}) führt zu den vier entkoppelten partiellen Differentialgleichungen:
\begin{subequations}\label{eq:phi_psi_wave}
	\begin{align}
		&\Phi\mid^{j}_{j} - \frac{1}{c_p^{2}}\ddot{\Phi} = 0 \label{eq:phi_wave}\\
		&\Psi_{i}\mid^{j}_{j} - \frac{1}{c_s^{2}}\ddot{\Psi}_{i} = 0 \label{eq:psi_wave}
	\end{align}
\end{subequations}
mit den Wellengeschwindigkeiten 
\begin{equation}\label{eq:cp_cs}
	c_p = \sqrt{\frac{\lambda + 2\mu}{\rho}} \quad \text{und} \quad
	c_s = \sqrt{\frac{\mu}{\rho}}\,
\end{equation}
 Eine ausführliche Herleitung ist in der Dissertation von \cite{Fruehe2010} zu finden.
 
 Als Ergebnis dieser Zerlegung erhält man ein System entkoppelter Potentiale, mit dem die Berechnung in den folgenden Kapiteln möglich ist.

Aus physikalischer Sicht lassen sich die Potentiale als verschiedene Wellentypen interpretieren.
Das skalare Potential $\Phi$ beschreibt dabei die Longitudinalwellen, auch P-Wellen oder Kompressionswellen genannt, die eine Volumenänderung bewirken und die Wellengeschwindigkeit $c_p$ besitzen.
Das vektorielle Potential $\Psi_{i}$ beschreibt hingegen die Transversalwellen, auch S-Wellen oder Scherwellen genannt, die mit einer Gestaltsänderung einhergehen und sich mit der Geschwindigkeit $c_s$ ausbreiten. Diese beiden Raumwellentypen treten im Boden auf \citep{Petersen2000}.

An der Bodenoberfläche erscheinen darüber hinaus Love- und Rayleigh-Wellen, die ebenfalls aus den Potentialen abgeleitet werden können. 
Im Vergleich zu den anderen Wellenarten sind die Rayleigh-Wellen besonders hervorzuheben, da sie im homogenen Halbraum den größten Anteil der Schwingungsenergie im Boden überträgt \citep{Haupt1986}.
Die Ausbreitungsgeschwindigkeit der Rayleigh-Wellen lässt sich mithilfe der Scherwellengeschwindigkeit $c_s$ zu $c_r = 0,92 \cdot c_s$ berechnen \citep{Petersen2000}:

Im Allgemeinen besteht bei der Wellenausbreitung der folgende Zusammenhang:
\begin{equation}\label{eq:lambda_cf}
	\lambda = \frac{c}{f}\,
\end{equation}




%durch den rotationsfreien Gradienten des Skalarfelds $\Phi$ und die quellfreien Rotation des Vektrofelds $\Psi_{l\,|k}\,^{ikl}$ zerlegen \cite{Lame1852}. Somit ergibt sich eine Zerlegung zu:

%\begin{equation}\label{eq:Helmholtz}
%	u^{i} &= \Phi\big|^{\,i} + \Psi_{l}\big|_{k}\,\varepsilon^{ikl}
%\end{equation}

%Durch Einsetzen von \ref{eq:Helmholtz} in \ref{eq:lame} zerfällt das gekoppelte Gleichungssystem in voneinander unabhängige Potentiale.

%Das skalare Potential beschreibt die Longitudinalwellen (P-Wellen), also Kompressionswellen, die eine Volumenänderung hervorrufen. Das vektorielle Potential hingegen beschreibt Transversalwellen (S-Wellen), die mit einer Gestaltsänderung einhergehen. 
	\chapter{Integraltransformationsmethode}
\label{cha:ITM}


\section{Vorbemerkung}
\label{sec:Vorbemerkung_ITM}

Um die unendliche Ausdehnung des Bodens und die in Kapitel (\ref{sec:Helmholtz}) beschriebenen elastischen Wellen analytisch zu berechnen, ist die Integraltransformationsmethode (ITM) eine geeignete Berechnungsmethode \citep{Mueller2007}.

Da die ITM allerdings nur einfache Geometrien beschreiben kann, müssen komplexere Strukturen durch eine Überlagerung von Fundamentalsystemen abgebildet werden. 

In der vorliegenden Arbeit wird ein Tunnel im ungeschichteten Boden untersucht.
Für diese geometrische Situation liegt keine Lösung vor, sodass für die Berechnung die Fundamentalsysteme Halbraum und Vollraum mit zylindrischem Hohlraum in den Kapiteln (\ref{sec:Halbraum}) und (\ref{sec:Zylinder}) herangezogen werden.
Im darauffolgenden Kapitel (\ref{sec:Superposition}) werden die Fundamentalsysteme überlagert, um so eine semi-analytische Lösung für die untersuchte geometrische Situation zu erhalten.

Zu beachten ist, dass die in den Kapiteln (\ref{sec:Halbraum}) und (\ref{sec:Zylinder}) beschriebenen Lösungen eine dynamische Belastung  (\(\omega \neq 0\)) voraussetzen. Aufgrund der gewählten Lösungsmethoden ist eine statische Belastung nicht zulässig \citep{Fruehe2010}.

\section{Fundamentalsystem Halbraum}
\label{sec:Halbraum}

\begin{figure}[H]
	\hspace*{45mm}
%	\centering
	\includesvg[height=4.5cm,keepaspectratio]{svg/cha_02_svg_01_hs}
	\caption{Fundamentalsystem homogener Halbraum - basiert auf \citep{Freisinger2022}.}
	\label{fig:cha09_hs}
\end{figure}
Wie \cite{Fruehe2010} beschreibt, werden die im Kapitel (\ref{sec:Helmholtz}) hergeleiteten Potentiale $\Phi$ und $\Psi_{i}$ für den Halbraum in kartesischen Koordinaten \((x^1 = x,\ x^2 = y,\ x^3 = z)\) beschrieben. Dadurch ergibt sich ein System entkoppelter partieller Differentialgleichungen zu:
\begin{subequations}\label{eq:partielle_DGL}
		\begin{align}
			&\left[\frac{\partial^{2}}{\partial x^{2}}
			+ \frac{\partial^{2}}{\partial y^{2}}
			+ \frac{\partial^{2}}{\partial z^{2}}
			- \frac{1}{c_p^{2}}\,\frac{\partial^{2}}{\partial t^{2}}\right]
			\Phi(x,y,z,t) = 0
			\label{eq:phi_wave} \\[6pt]
			&\left[\frac{\partial^{2}}{\partial x^{2}}
			+ \frac{\partial^{2}}{\partial y^{2}}
			+ \frac{\partial^{2}}{\partial z^{2}}
			- \frac{1}{c_s^{2}}\,\frac{\partial^{2}}{\partial t^{2}}\right]
			\Psi_{i}(x,y,z,t) = 0
			\label{eq:psi_wave}
		\end{align}
\end{subequations}
mit $\Psi_{i}$ in den drei kartesichen Raumrichtungen \( i = x, y, z\)

%\newcommand{\mapto}{\mathrel{\laplace}}
%\newcommand{\mapfrom}{\mathrel{\Laplace}}
Um dieses partielle System zu lösen, erfolgt eine Transformation zu gewöhnlichen Differentialgleichungen mittels dreifacher Fouriertransformation. 
Dazu werden die Ortskoordinaten $x$ und $y$ aus dem Originalraum zu Wellenzahlen \(k_x\) und \(k_y\) im fouriertransformierten Bildraum \((x \mapto k_x,\; y \mapto k_y)\) sowie die Zeit $t$ in den Frequenzbereich $\omega$ \((t \mapto \omega\)) transformiert.\\
Um entlang der $z$-Koordinate  im Folgenden Randbedingungen vergeben zu können, bleibt diese untransformiert im Originalraum \citep{Mueller2007}.

Die Definition der Fouriertransformation ist in Gleichung (\ref{eq:fouriertransformation}) aufgeführt.

Die Transformation des Systems partieller Differentialgleichungen (\ref{eq:partielle_DGL}) in den fouriertransfmorierten Bildraum führt zur Umwandlung in ein System gewöhnlicher Differentialgleichungen und lässt sich darstellen durch:
\begin{subequations}\label{eq:gewöhnliche_DGL}
	{	\begin{align}
			&\left[-k_x^{2}-k_y^{2}+k_p^{2}+\frac{\partial^{2}}{\partial z^{2}}\right]
			\hat{\Phi}(k_x,k_y,z,\omega) = 0 \label{eq:phi_helmholtz}\\[6pt]
			&\left[-k_x^{2}-k_y^{2}+k_s^{2}+\frac{\partial^{2}}{\partial z^{2}}\right]
			\hat{\Psi}_{i}(k_x,k_y,z,\omega) = 0 \label{eq:psi_helmholtz}
		\end{align}
	}
\end{subequations}
mit den Wellenzahlen der Kompressionswelle \( k_p = \frac{\omega}{c_p} \;\text{und der Scherwelle}\; k_s = \frac{\omega}{c_s} \).

Das Symbol ($\hat{\cdot}$) kennzeichnet dabei die dreifach fouriertransformierten Größen.

Zur Lösung des Systems gewöhnlicher Differentialgleichungen (\ref{eq:gewöhnliche_DGL}) wird ein analytischer Exponentialansatz (\ref{eq:expo_solutions}) herangezogen \citep{Wolf1985}. Da außerdem gemäß \cite{Long1967} für das Potential $\Psi_{z} = 0$ gilt, wird die folgende Lösung in $\alpha=x,y$ definiert.
\begin{subequations}\label{eq:expo_solutions}
	{\begin{align}
			&\hat{\Phi} = A_{1} e^{\lambda_{1} z} + A_{2} e^{-\lambda_{1} z}\label{eq:phi_sol}\\[4pt]
			&\hat{\Psi}_{\alpha} = B_{\alpha1} e^{\lambda_{2} z} + B_{\alpha2} e^{-\lambda_{2} z}\label{eq:psi_sol}
		\end{align}
	}
\end{subequations}
mit \( \lambda_{1} = \sqrt{k_x^{2}+k_y^{2}-k_p^{2}},\;
\lambda_{2} = \sqrt{k_x^{2}+k_y^{2}-k_s^{2}} \)
sowie den unbekannten Koeffizienten\\
\(A_1, A_2, B_{\alpha1}, B_{\alpha2}\)

Der Lösungsansatz (\ref{eq:expo_solutions}) der Potentiale $\hat{\Phi}$ und $\hat{\Psi}_{\alpha}$ ermöglicht die Berechnung der Verschiebungen \(\hat{\mathbf{u}}_{k}\) und Spannugen \(\hat{\boldsymbol{\sigma}}_{k}\) im fouriertransformierten Raum. 
Der Index (\({\cdot}_k\)) steht dabei für den Halbraum, der im kartesischen Koordinatensystem beschrieben wird.

So lässt sich die Verschiebung im dreifach fouriertransformierten Raum \(\hat{\mathbf{u}}_{k}\) über die Matrix $\hat{\mathbf{H}}_{k}$, dessen Einträge den Exponentialansatz (\ref{eq:expo_solutions}) enthalten, und dem Vektor $\mathbf{c}_k$, der die unbekannten Koeffizienten enthält, wie folgt darstellen \citep{Fruehe2010}: 
\begin{equation}\label{eq:displ_hs}
	\hat{\mathbf{u}}_{k} = \hat{\mathbf{H}}_{k}\cdot \mathbf{c}_{k}
\end{equation}
%Die Einträge der Gleichung (\ref{eq:displ_hs}) sind im Anhang (\ref{cha:Halbraum}) aufgeführt. Eine genauere Herleitung der Lösung ist außerdem in den Dissertationen von \cite{Hackenberg2016} und \cite{Freisinger2022} zu finden.

%ablage:
%Dabei enthalten die Matrizen $\bigl[\hat{H}_{z}\bigr]$ und $\bigl[\hat{K}_{z}\bigr]$ wieder Einträge, die aus dem gewählten Lösungsansatz resultieren.

Über die Verschiebungs-Verzerrungsbeziehung, die Spannungs-Verzerrungsbeziehung sowie einer dreifachen Transfomation in den fouriertransformierten Bildraum beschreibt \cite{Mueller2007} die Spannungen \(\hat{\boldsymbol{\sigma}}_{k}\) wie folgt:
\begin{equation}\label{eq:sigma_factorization}
	\hat{\boldsymbol{\sigma}}_{k} = \hat{\mathbf{K}}_{k}\cdot \mathbf{c}_{k}
\end{equation}
Dabei enthalten die Matrizen $\hat{\mathbf{H}}_{k}$ und $\hat{\mathbf{K}}_{k}$ Einträge, die aus dem gewählten Exponentialansatz (\ref{eq:expo_solutions}) resultieren und der Vektor $\mathbf{c}_k$ die unbekannten Koeffizienten. Deren Einträge sind im Anhang (\ref{cha:Halbraum}) aufgeführt.

Eine ausführliche Herleitung der Gleichungen (\ref{eq:displ_hs}) und (\ref{eq:sigma_factorization}) ist in den Dissertationen von \cite{Mueller2007} und \cite{Fruehe2010} zu finden.
%Für eine genaue Herleitung wird auf die Dissertation von \cite{Fruehe2010} verwiesen.

%Die Gleichung (\ref{eq:sigma_factorization}) enthält ebenfalls den Vektor $C_k$ mit den unbekannten Koeffizienten. Die Einträge der Gleichung (\ref{eq:sigma_factorization} sind ebenfalls im Anhang (\ref{cha:Halbraum}) zu finden. 

Die sechs unbekannten Koeffizienten \(A_1, A_2, B_{\alpha1}, B_{\alpha2}\), die aus dem gewählten Exponentialansatz (\ref{eq:expo_solutions}) resultieren, werden über die lokalen Randbedingungen an der Halbraumoberfläche und nicht lokalen Randbedingungen im Unendlichen bestimmt.

Die drei Koeffizienten \(A_1, B_{x1}, B_{y1}\) ergeben sich unter Heranziehung der Sommerfeldschen Abstrahlbedingung. Diese fordert, dass die Amplitude der Wellen in zunehmender Tiefe abklingt und sich die Wellenausbreitung nur in positiver $z$-Richtung, also weg von der Halbraumoberfläche $\Lambda$, fortsetzt.

Die übrigen unbekannten Koeffizienten \(A_2, B_{x2}, B_{y2}\) werden durch die Randbedingung an der Halbraumoberfläche $\Lambda$ hergeleitet, da die angreifende Belastung mit der Spannung an der Oberfläche im Gleichgewicht stehen muss \citep{Mueller2007}.

Um nun die Spannungen und Verschiebungen im Originalraum ($x$,$y$,$z$,$t$) zu erhalten, erfolgt eine mehrfache Fourierrücktransformation, welche im Anhang (\ref{sec:fouriertransformation}) definiert ist.



\section{Fundamentalsystem Vollraum mit zylindrischem Hohlraum}
\label{sec:Zylinder}

Im Gegensatz zur Lösungsfindung im Halbraum, werden die im Kapitel (\ref{sec:Helmholtz}) hergeleiteten Potentiale $\Phi$ und $\Psi_{i}$ für das System Vollraum mit zylindrischem Hohlraum in Zylinderkoordinaten ($x$,$r$,$\varphi$), die mit dem Index ($\cdot_z$) gekennzeichnet werden, beschrieben.
\begin{figure}[H]
	\centering
	\begin{subfigure}[t]{0.48\textwidth}
		\hspace*{25mm}
		\centering
		\includesvg[width=\linewidth,pretex=\centering]{svg/cha_02_svg_05_fs_cyl}
		\label{fig:cyl_a}
	\end{subfigure}\hfill
	\begin{subfigure}[t]{0.48\textwidth}
		\centering
			\hspace*{-7mm}
		\includesvg[width=\linewidth,pretex=\centering]{svg/cha_02_svg_06_cyl_cos}
		\label{fig:cyl_b}
	\end{subfigure}
	\caption{Vollraum mit zylindrischem Hohlraum (rechts) und Zylinderkoordinaten (links) \citep{Freisinger2022}.}
	\label{fig:cyl_pair}
\end{figure}


Wie in der Abbildung (\ref{fig:cyl_pair}) dargestellt, wird die räumliche Lage eines Punktes durch die Koordinaten $x$, $r$ und $\varphi$ beschrieben, wobei \(x\) die longitudinale, \(r\) die radiale und \(\varphi\) die umlaufende Koordinate darstellt \citep{Freisinger2022}.


Da das Vektorpotential in Zylinderkoordinaten nicht direkt zu entkoppelten Gleichungen führt \citep{Hackenberg2016}, zeigt \cite{Fruehe2010}, dass das Vektorfeld $\Psi_{i}$ unter der Zusatzbedingung, dass es quellfrei ist, durch zwei voneinander unabhängige Skalarfunktionen $\psi$ und $\chi$ dargestellt werden kann.
Mithilfe dieser erweiterten Helmholtz-Zerlegung lassen sich schließlich wieder entkoppelte partielle Differentialgleichungen gewinnen, die sich wie folgt darstellen lassen:
\begin{subequations}\label{eq:cyl_wave}
	{
		\begin{align}
			&\left[\frac{\partial^{2}}{\partial x^{2}}
			+ \frac{\partial^{2}}{\partial r^{2}}
			+ \frac{1}{r}\frac{\partial}{\partial r}
			+ \frac{1}{r^{2}}\frac{\partial^{2}}{\partial \varphi^{2}}
			- \frac{1}{c_p^{2}}\frac{\partial^{2}}{\partial t^{2}}\right]
			\Phi(x,r,\varphi,t)=0 \label{eq:cyl_phi}\\[6pt]
			&\left[\frac{\partial^{2}}{\partial x^{2}}
			+ \frac{\partial^{2}}{\partial r^{2}}
			+ \frac{1}{r}\frac{\partial}{\partial r}
			+ \frac{1}{r^{2}}\frac{\partial^{2}}{\partial \varphi^{2}}
			- \frac{1}{c_s^{2}}\frac{\partial^{2}}{\partial t^{2}}\right]
			\psi(x,r,\varphi,t)=0 \label{eq:cyl_psi}\\[6pt]
			&\left[\frac{\partial^{2}}{\partial x^{2}}
			+ \frac{\partial^{2}}{\partial r^{2}}
			+ \frac{1}{r}\frac{\partial}{\partial r}
			+ \frac{1}{r^{2}}\frac{\partial^{2}}{\partial \varphi^{2}}
			- \frac{1}{c_s^{2}}\frac{\partial^{2}}{\partial t^{2}}\right]
			\chi(x,r,\varphi,t)=0 \label{eq:cyl_chi}
		\end{align}
	}%
\end{subequations}


Analog zum Vorgehen in Kapitel (\ref{sec:Halbraum}) werden die partiellen Differentialgleichungen über drei Überführungen in den fouriertransformierten Bildraum gelöst.
Dabei erfolgt eine zweifache Fouriertransformation bezüglich der Ortskoordinate $x$ in die Wellenzahlen $k_{x}$
sowie die Zeit $t$ in den Frequenzbereich \((x \mapto k_x,\; t \mapto \omega)\).
Außerdem erfolgt eine Fourierserie der umlaufenden Koordinate $\varphi$ in Bezug auf die Umfangsrichtung des Zylinders (\(\varphi \rightarrow n\)) \citep{Kausel2006}.

\cite{Hackenberg2016} stellt die Fourierreihe der zweifach fouriertransformierten Differentialgleichungen wie folgt dar:
\begin{subequations}\label{eq:circ_fourier}
	\begin{align}
		\tilde{\Phi}(k_x,r,\varphi,\omega)
		&= \sum_{n=-\infty}^{\infty} \hat{\Phi}(k_x,r,n,\omega)\,\eu^{\iu n \varphi} \label{eq:circ_fourier_a}\\
		\tilde{\psi}(k_x,r,\varphi,\omega)
		&= \sum_{n=-\infty}^{\infty} \hat{\psi}(k_x,r,n,\omega)\,\eu^{\iu n \varphi} \label{eq:circ_fourier_b}\\
		\tilde{\chi}(k_x,r,\varphi,\omega)
		&= \sum_{n=-\infty}^{\infty} \hat{\chi}(k_x,r,n,\omega)\,\eu^{\iu n \varphi} \label{eq:circ_fourier_c}
	\end{align}
\end{subequations}
Das Symbol ($\tilde{\cdot}$) kennzeichnet dabei die zweifach fouriertransformierten Größen.

Das resultierdende System gewöhnlicher Differentialgleichungen lässt sich nach \cite{Fruehe2010} darstellen durch:
\begin{subequations}\label{eq:cyl_fourier_odes}
	{%
		\begin{align}
			&\left[-k_x^{2}
			+ \frac{\partial^{2}}{\partial r^{2}}
			+ \frac{1}{r}\frac{\partial}{\partial r}
			- \frac{n^{2}}{r^{2}}
			+ k_p^{2}\right]\,
			\hat{\Phi}\!\left(k_x,r,n,\omega\right) = 0 \label{eq:13a}\\[6pt]
			&\left[-k_x^{2}
			+ \frac{\partial^{2}}{\partial r^{2}}
			+ \frac{1}{r}\frac{\partial}{\partial r}
			- \frac{n^{2}}{r^{2}}
			+ k_s^{2}\right]\,
			\hat{\psi}\!\left(k_x,r,n,\omega\right) = 0 \label{eq:13b}\\[6pt]
			&\left[-k_x^{2}
			+ \frac{\partial^{2}}{\partial r^{2}}
			+ \frac{1}{r}\frac{\partial}{\partial r}
			- \frac{n^{2}}{r^{2}}
			+ k_s^{2}\right]\,
			\hat{\chi}\!\left(k_x,r,n,\omega\right) = 0 \label{eq:13c}
		\end{align}
	}%
\end{subequations}

Die Lösung der Differentialgleichung (\ref{eq:cyl_fourier_odes}) leitet \cite{Fruehe2010} durch die Anwendung von Hankel-Funktionen erster Art $H^{(1)}_{n}$ sowie zweiter Art $H^{(2)}_{n}$ her, sodass sich die Potential ergeben zu:
\begin{subequations}\label{eq:hankel_expansions}
	\begin{align}
		\hat{\Phi}(k_x,r,n,\omega)
		&= C_{1n}\, H^{(1)}_{n}(k_{\alpha} r) + C_{4n}\, H^{(2)}_{n}(k_{\alpha} r) \label{eq:hankel_a}\\
		\hat{\psi}(k_x,r,n,\omega)
		&= C_{2n}\, H^{(1)}_{n}(k_{\beta} r) + C_{5n}\, H^{(2)}_{n}(k_{\beta} r) \label{eq:hankel_b}\\
		\hat{\chi}(k_x,r,n,\omega)
		&= C_{3n}\, H^{(1)}_{n}(k_{\beta} r) + C_{6n}\, H^{(2)}_{n}(k_{\beta} r) \label{eq:hankel_c}
	\end{align}
\end{subequations}
mit
$k_{\alpha}=\sqrt{k_{p}^{2}-k_{x}^{2}}=\sqrt{\frac{\omega^{2}}{c_{p}^{2}}-k_{x}^{2}}$ und $k_{\beta}=\sqrt{k_{s}^{2}-k_{x}^{2}}=\sqrt{\frac{\omega^{2}}{c_{s}^{2}}-k_{x}^{2}}$
sowie den unbekannten $C_{in}$ mit $i=1,2,3,4,5,6$ %C_{1n}, C_{2n}, C_{3n}, C_{4n}, C_{5n}, C_{6n}\).

Die Hankel-Funktionen erster Art $H^{(1)}_{n}$ und zweiter Art $H^{(2)}_{n}$ setzen sich aus den Bessel-Funktionen und Neumann-Funktionen zusammen und sind in der Dissertation von \cite{Fruehe2010} aufgeführt. Im Rahmen der vorliegenden Arbeit wird darauf nicht weiter eingegangen.

Mithilfe des in Gleichung (\ref{eq:hankel_expansions}) vorgestellten Lösungsansatzes lassen sich nun die Verschiebungen \(\hat{\mathbf{u}}_{z}\) und Spannungen \(\hat{\boldsymbol{\sigma}}_{z}\) im fouriertransformierten Raum berechnen.

Analog zu den Herleitungen im Halbraum lassen sich die Verschiebung im fouriertransformierten Raum \(\hat{\mathbf{u}}_{z}\) über die Matrix $\hat{\mathbf{H}}_{z}$, dessen Einträge die Hankelfunktionen $H^{(1)}_{n}$ und $H^{(2)}_{n}$ enthalten, und dem Vektor $\mathbf{c}_z$, der die unbekannten Koeffizienten $C_{in}$ enthält, wie folgt darstellen \citep{Fruehe2010}:
\begin{equation}\label{eq:solution_cyl_uz}
	\hat{\mathbf{u}}_{z} = \hat{\mathbf{H}}_{z}\cdot \mathbf{c}_z
\end{equation}

Wie bei der Herleitung der Spannungen im Halbraum $\boldsymbol{\sigma}_{k}$ ergeben sich die Spannungen in Zylinderkoordinaten $\boldsymbol{\sigma}_{z}$ ebenfalls aus der Verschiebung, der Verschiebungs- und der Spannungs-\\Verzerrungsbeziehung sowie drei Transformationen in den Bildraum zu:
\begin{equation}\label{eq:solution_cyl_sig}
	\hat{\boldsymbol{\sigma}}_{z} = \hat{\mathbf{K}}_{z}\cdot \mathbf{c_z}
\end{equation}
Dabei enthalten die Matrizen $\hat{\mathbf{H}}_{z}$ und $\hat{\mathbf{K}}_{z}$ wieder Einträge, die aus dem gewählten Lösungsansatz resultieren, die im Rahmen dieser Arbeit nicht aufgeführt werden und in der Dissertationen von \cite{Fruehe2010} und \cite{Mueller2007} zu finden sind.

Analog zu den Gleichungen (\ref{eq:displ_hs}) und (\ref{eq:sigma_factorization}) enthält der Vektor \(\mathbf{c_z}\) ebenfalls sechs unbekannte Koeffizienten, die mittels Randbedingungen ermittelt werden.

Drei davon ergeben sich aus der Bedingung, dass sich die Amplituden der Wellen mit zunehmender Entfernung von der Belastung abnehmen und sich nur vom Ort der Belastung ausbreiten.
Die übrigen Unbekannten werden über die Bedingung bestimmt, dass die eingeleitete Belastung am Zylinderrand $\Gamma_z$ mit den Spannungen im Gleichgewicht steht \citep{Fruehe2010}.

Um die Spannungen und Verschiebungen im Orginalraum ($x$,$r$,$\varphi$) zu erhalten erfolgt eine erneute mehrfache Fourierrücktransformation nach der Definition in Gleichung (\ref{eq:invfouriertransformation}).





\section{Überlagerung der Fundamentalsysteme}
\label{sec:Superposition}

Eine Lösung für das komplexe System Halbraum mit zylindrischem Hohlraum wird durch Überlagerung der in Kapitel (\ref{sec:Halbraum}) und (\ref{sec:Zylinder}) beschriebenen Fundamentalsysteme ermittelt.

Dafür werden die beiden Fundamentalsysteme Halbraum und Vollraum mit zylindrischem Hohlraum an gemeinsamen, fiktiven Kopplungsflächen miteinander verknüpft. 
An dieser Fläche müssen zwei Bedingungen erfüllt sein. 
Einerseits muss ein Kräftegleichgewicht herrschen, sodass sich die aufgebrachten Spannungen der beiden Fundamentalsysteme gegenseitig aufheben. 
Andererseits müssen die Verschiebungen der beiden Fundamentalsysteme an der Kopplungsfläche identisch sein.
\begin{figure}[H]
	\hspace*{5mm}
	\centering
	\begin{subfigure}[t]{0.48\textwidth}
		\centering
		\includesvg[width=\linewidth,pretex=\centering]{svg/cha_03_svg_03_hs_sup_stress}
		\label{fig:Halbraum_fiktiveFlächen}
	\end{subfigure}\hfill
	\begin{subfigure}[t]{0.48\textwidth}
		\centering
		\includesvg[width=\linewidth,pretex=\centering]{svg/cha_03_svg_04_fs_cyl_sup_stress}
		\label{fig:Zylinder_fiktiveFlächen}
	\end{subfigure}
	\caption{Fundamentalsysteme Halbraum (links) und Vollraum mit zylindrischem Hohlraum (rechts) mit fiktiven Überlagerungsflächen - basiert auf \cite{Freisinger2022}.}
	\label{fig:Überlagerung_fiktiveFlächen}
\end{figure}
Wie in Abbildung (\ref{fig:Überlagerung_fiktiveFlächen}) dargestellt, wird zu diesem Zweck im Halbraum an der Stelle, an der sich der zylindrische Hohlraum befindet, eine fiktive Zylinderoberfläche $\delta\Gamma_{\mathrm z}$ eingeführt. Im Vollraum mit zylindrischem Hohlraum wird entsprechend der Bodenoberfläche eine fiktive Halbraumoberfläche $\delta\Lambda$ eingeführt. 
Entlang dieser Kopplungsflächen werden die genannten Randbedingungen formuliert und die Lösungen beider Teilsysteme miteinander kombiniert \citep{Mueller2007}.
 
 Zur Durchführung der Kopplung werden fiktive Belastungen auf der Halbraumoberfläche $\Lambda$ und der Zylinderoberfläche $\Gamma_z$ aufgebracht. 
 Mithilfe der hergeleiteten Spannungen und Verschiebungen können in den jeweiligen Fundamentalsystemen die Zustandsgrößen an den eingeführten, fiktiven Oberflächen $\delta\Gamma_z$ sowie $\delta\Lambda$ berechnet werden.
 Die Spannungen und Verschiebungen sind für den Halbraum in den Gleichungen (\ref{eq:displ_hs}) und (\ref{eq:sigma_factorization}) sowie für den Vollraum mit zylindrischem Hohlraum in den Gleichungen (\ref{eq:solution_cyl_uz}) und (\ref{eq:solution_cyl_sig}) hergeleitet.
 
 Da die Spannungs- und Verschiebungsgrößen der beiden Fundamentalsysteme in unterschiedlichen Koordinaten vorliegen, ist eine Transformation erforderlich, um die Kopplungsbedingungen durchzuführen. Während die Lösungen des Halbraums in den Koordinaten ($k_x, k_y,z,\omega$) gegeben sind, beziehen sich die Lösungen des Vollraums mit zylindrischem Hohlraum auf die Koordinaten ($k_x, r,n,\omega$).
  
 % \newcommand{\iftmap}{%
%  	\tikz[baseline=-0.5ex]{
%  		\fill (0,0) circle(0.7ex);
%  		\draw[semithick] (0,0) -- (1.8em,0);
%  		\draw[semithick] (1.8em,0) circle(0.7ex);
  %	}%
%  }
%  \newcommand{\mapfrom}{\mathrel{\iftmap}}
 Fiktive Belastungen, die auf die Halbraumoberfläche $\Lambda$ aufgebracht werden, müssen folglich durch eine inverse Fouriertransformation ($k_y\mapfrom k$), eine räumliche Transformation in Zylinderkoordinaten ($y \rightarrow r, z \rightarrow \varphi$) sowie eine Fourierreihe ($\varphi \rightarrow n$) an der fiktiven Zylinderoberfläche $\delta\Gamma_z$ transfomriert werden.

Umgekehrt werden fiktive Belastungen, die auf die Zylinderoberfläche $\Gamma_z$ aufgebracht werden, mithilfe einer inversen Fourierreihe ($n \rightarrow \varphi$), einer räumlichen Transformation in kartesische Koordinaten ($r \rightarrow y, \varphi \rightarrow z$) sowie einer Fouriertransformation ($y \mapto k_y$) an der fiktiven Halbraumoberfläche $\delta\Lambda$ transformiert \citep{Freisinger2022}.


\cite{Fruehe2010} definiert zur räumlichen Transfomation die Matrix (\ref{eq:Transforationsmatrix}), mit der die Größen zwischen den beiden Koordinatensystemen transformiert werden können.
\begin{equation} \label{eq:Transforationsmatrix}
\boldsymbol\beta
= \begin{bmatrix}
	1 & 0 & 0 \\
	0 & -\sin\varphi & \cos\varphi \\
	0 & -\cos\varphi & -\sin\varphi
\end{bmatrix}
\end{equation}
Durch diese Folge an Transformationen, lassen sich die Kopplungsbedingungen über die Verschiebungen und die Spannungen formulieren und durchführen.


Eine detaillierte Herleitung der Kopplungsbedingungen ist in den Dissertationen von \cite{Fruehe2010}, \cite{Hackenberg2016} und \cite{Freisinger2022} zu finden.





\section{Dynamische Steifigkeitsmatrix des ITM-Teilsystems}
\label{sec:Steifigkeiten_ITM}

Die Kopplung der Lösungen der Teilsysteme ITM und der FEM erfolgt in Kapitel (\ref{cha:Kopplung}) mittels der dynamischen Steifigkeitsmatrizen der Berechnungsmethoden. Aus diesem Grund ist es erforderlich, die dynamische Steifigkeitsmatrix $\hat{\mathbf K}_{\mathrm{ITM}}$ für die ITM zu ermitteln.

Zur besseren Nachvollziehbarkeit der Kopplung in Kapitel (\ref{cha:Kopplung}) werden die Vektoren und Matrizen des ITM-Teilsystems mit dem Index ($\cdot_{\mathrm{ITM}}$) versehen.

Die Berechnung der dynamischen Steifigkeitsmatrix $\hat{\mathbf K}_{\mathrm{ITM}}$ erfolgt gemäß der folgenden, allgemeinen Beziehung \citep{Kausel2006}:
\begin{equation}\label{eq:itm_system}
	\hat{\mathbf p}_{\mathrm{ITM}} = 
	\hat{\mathbf K}_{\mathrm{ITM}}\cdot\hat{\mathbf u}_{\mathrm{ITM}}\,
\end{equation}
Die dynamische Steifigkeitsmatrix $\hat{\mathbf K}_{\mathrm{ITM}}$ wird demnach unter Anwendung von Beziehungen für den Vektor der Belastungen $\hat{\mathbf p}_{\mathrm{ITM}}$ und den Vektor der Verschiebungen $\hat{\mathbf u}_{\mathrm{ITM}}$ berechnet. Gemäß \cite{Hackenberg2016} ergeben sich diese Beziehungen zu:
\begin{subequations}\label{eq:itm_system_blocks}
	\begin{align}
		\hat{\mathbf p}_{\mathrm{ITM}} &= \hat{\mathbf S}_{\mathrm{ITM}} \cdot \mathbf c \label{eq:itm_system_block_a}\\
		\hat{\mathbf u}_{\mathrm{ITM}} &= \hat{\mathbf U}_{\mathrm{ITM}} \cdot \mathbf c \label{eq:itm_system_block_b}
	\end{align}
\end{subequations}
Die Ermittlung der Verschiebungsoperatoren $\hat{\mathbf U}_{\mathrm{ITM}}$ und der Spannungsoperatoren $\hat{\mathbf S}_{\mathrm{ITM}}$ erfolgt durch das Aufbringen fiktiver Belastungen auf der Halbraumoberfläche $\Lambda$ und der Zylinderoberfläche $\Gamma_z$ sowie das Berechnen der resultierenden Verschiebungen und Spannungen an den fiktiven Kopplungsflächen $\delta \Lambda_z$ und $\delta\Gamma$.
Der Vektor $\mathbf{c}$ enthält dabei die Amplituden auf der Halbraumoberfläche $\Lambda$ und der Zylinderoberfläche $\Gamma_z$ infolge der fiktiven Belastungen. 

Eine Aufführung der Einträge der Matrizen $\hat{\mathbf U}_{\mathrm{ITM}}$ und $\hat{\mathbf S}_{\mathrm{ITM}}$ sowie des Vektors $\mathbf{c}$ ist in den Dissertationen von \cite{Hackenberg2016} und \cite{Freisinger2022} zu finden und wird im Rahmen der vorliegenden Arbeit nicht weiter dargestellt.

%Die Matrizen $\hat{\mathbf U}_{\mathrm{ITM}}$ und $\hat{\mathbf S}_{\mathrm{ITM}}$ sind Verschiebungs- und Spannungsoperatoren, die aus den Sp


%Die Matrix $\hat{\mathbf S}_{\mathrm{ITM}}$ stellt einen Spannungsoperator dar, resultierend aus den Randbedingungen.

%resultiert aus den, im Rahmen des Kopplungsprozesses, aufgebrachten fiktiven Einheitsspannungen

%Die Matrizen $\hat{\mathbf S}_{\mathrm{ITM}}$ und $\hat{\mathbf U}_{\mathrm{ITM}}$ sowie der Vektor $\mathbf c$ sind dabei im Anhang aufgeführt.


Daraufhin werden die Beziehungen, die in den Gleichungen (\ref{eq:itm_system_blocks}) dargestellt sind, in die Gleichung (\ref{eq:itm_system}) eingesetzt. Anschließend wird nach der dynamischen Steifigkeitsmatrix $\hat{\mathbf K}_{\mathrm{ITM}}$ aufgelöst, die sich damit ergibt zu:
\begin{equation}\label{eq:itm_steifigkeitsm}
	\hat{\mathbf K}_{\mathrm{ITM}} = 
	\hat{\mathbf S}_{\mathrm{ITM}} \cdot \hat{\mathbf U}_{\mathrm{ITM}}^{-1}
\end{equation}
Um die in Kapitel (\ref{cha:Kopplung}) beschriebene Kopplung durchzuführen, ist es erforderlich, die herangezogene Gleichung (\ref{eq:itm_system}) in der Blockdarstellung (\ref{eq:itm_block_system_braced}), getrennt nach der Halbraumoberfläche $\Lambda$ und der Zylinderoberfläche $\Gamma_z$ darzustellen.
\begin{equation}\label{eq:itm_block_system_braced}
	\underbrace{\begin{bmatrix}
			\hat{\mathbf K}_{\Lambda\Lambda_{\mathrm{ITM}}} &
			\hat{\mathbf K}_{\Lambda\Gamma_{\mathrm{ITM}}} \\
			\hat{\mathbf K}_{\Gamma\Lambda_{\mathrm{ITM}}} &
			\hat{\mathbf K}_{\Gamma\Gamma_{\mathrm{ITM}}}
	\end{bmatrix}}_{\hat{\mathbf K}_{\mathrm{ITM}}}\,
	\begin{pmatrix}
		\hat{\mathbf u}_{\Lambda_{\mathrm{ITM}}} \\
		\hat{\mathbf u}_{\Gamma_{\mathrm{ITM}}}
	\end{pmatrix}
	=
	\begin{pmatrix}
		\hat{\mathbf p}_{\Lambda_{\mathrm{ITM}}} \\
		\hat{\mathbf p}_{\Gamma_{\mathrm{ITM}}}
	\end{pmatrix}
\end{equation}
%Für eine detaillierte Herleitung wird auf die Dissertationen von \cite{Hackenberg2016} und \cite{Freisinger2022} verwiesen.
	% !TeX root = a_main_thesis.tex
\chapter{Finite-Elemente-Methode}
\label{cha:FEM}

\section{Vorbemerkung}
\label{sec:FEM_Vorbemerkung}

Die ITM eignet sich zwar sehr gut für die Modellierung unendlicher Böden, für die Berechnung komplexerer Geometrien ist sie jedoch ungeeignet. Für die Geometrie im Tunnelbereich wird daher die Finite-Elemente-Methode (FEM) herangezogen.

Die Implementierung erfolgt dabei im fouriertransformierten Raum.
Dadurch wird der hohe Rechenaufwand reduziert, da die longitudinale Koordinate $x$ in den Wellenzahlbereich $k_x$ überführt wird \((x \mapto k_x\)). 
Somit kann der konstante Tunnelquerschnitt mittels finiter Elemente auf ein zweidimensionales Problem heruntergebrochen werden \citep{Fruehe2010}. 

Um im Anschluss die Lösungen der ITM- und FEM-Teilsysteme zu einer Lösung des Gesamtsystems zu koppeln, ist zudem eine Teillösung im selben Fourierraum erforderlich. 
Daher findet eine zusätzliche Überführung der Zeit $t$ in den Frequenzbereich $\omega$ statt \((t \mapto \omega\)). 

Daher beschränkt sich die Diskretisierung auf die $y$-$z$-Ebene, die für jedes Wertepaar von $k_x$ und $\omega$ separat ausgewertet wird. 
Dieser reduzierte Ansatz wird als 2,5-dimensional bezeichnet \citep{Freisinger_Hackenberg2020}. 


\section{Trianguläre finite Elemente durch degenerierte Vierecke}
\label{sec:degenerierte Dreiecke}

Die zu untersuchende Geometrie Halbraum mit zylindrischem Hohlraum wurde bereits im gekoppelten ITM-FEM-Ansatz von \cite{Hackenberg2016} und \cite{Freisinger2022} untersucht.
Im Rahmen dieser Dissertationen wurde der Tunnel durch ein FE-Netz bestehend aus viereckigen finiten Elementen analysiert, wie in Abbildung (\ref{fig:FE_4node}) dargestellt.
Diese Implementation in \emph{MATLAB\texttrademark} dient der vorliegenden Arbeit als Grundlage.
Das Ziel der vorliegenden Arbeit ist die Weiterentwicklung der Analyse des Tunnels. Zu diesem Zweck wird eine Umstellung des FE-Netztes von viereckige auf dreieckige Elemente vorgenommen.

Für die Implementierung lassen sich zwei verschiedene Herangehensweisen in Betracht ziehen. Die Implementierung des dreieckigen FE-Netzes kann zum einen durch neue, native Dreieckselemente erfolgen. In den Arbeiten von \cite{Gross2023} und \cite{Zienkiewicz2013} wird auf spezifische Formeln verwiesen, die sich auf die baryzentrischen Koordinaten eines Dreiecks beziehen und für die Implementierung nativer Dreieckselemente von Relevanz sind. Die Positionen werden dabei relativ zu den übrigen Punkten des Dreiecks beschrieben.

So beschreibt \cite{Zienkiewicz2013} ein natives Dreieck mit den Knotenkoordinaten $x_a$ und $y_a$ durch die Ansatzfunktionen $N_a(x,y)$ mit $a=1,2,3$ wie folgt:
\begin{subequations}\label{eq:tri_shape_linear}
	\begin{equation}\label{eq:tri_shape_linear_a}
		N_a(x,y)=\frac{1}{2{A_\Delta}}\,\bigl(a_a+b_a\,x+c_a\,y\bigr)
	\end{equation}
	\noindent\text{mit}
	\begin{equation}\label{eq:tri_shape_linear_b}
		\begin{aligned}
			a_1 &= x_2y_3 - x_3y_2, &\qquad b_1 &= y_2 - y_3, &\qquad c_1 &= x_3 - x_2,\\
			a_2 &= x_3y_1 - x_1y_3, &        b_2 &= y_3 - y_1, &        c_2 &= x_1 - x_3,\\
			a_3 &= x_1y_2 - x_2y_1, &        b_3 &= y_1 - y_2, &        c_3 &= x_2 - x_1.
		\end{aligned}
	\end{equation}
	\noindent\text{sowie der Fläche des Dreiecks}
	\begin{equation}\label{eq:tri_shape_linear_c}
		A_\Delta=\frac{1}{2}\bigl(x_1 b_1 + x_2 b_2 + x_3 b_3\bigr)
	\end{equation}
\end{subequations}
Die Implementation nativer, dreieckiger, finiter Elemente geht mit einem höheren Aufwand einher, der sich zum einen aus der Sortierung der Elemente sowie der Umstellung auf dreiecksspezifische Koordinaten ergibt.

Um im Rahmen der vorliegenden Arbeit eine möglichst hohe Ähnlichkeit zu der gegebenen Implementierung von \cite{Hackenberg2016} und \cite{Freisinger2022} zu gewährleisten, wird ein alternativer Ansatz herangezogen.

In der Arbeit von \cite{Zienkiewicz2013} wird der Degenerationsansatz präsentiert, der  eine Degeneration linearer Vierecke zu linearen Dreiecken beschreibt.
Hierbei fallen zwei benachbarte Knoten eines Elements mit vier Knoten zusammen, sodass diese identische Koordinaten aufweisen. In der Folge degeneriert das viereckige Element topologisch zu einem Dreieck, wie in Abbildung (\ref{fig:Degeneration}) dargestellt.
\begin{figure}[H]
	\hspace*{28mm}
	%	\centering
	\includesvg[height=4.5cm,keepaspectratio]{svg/Degeneration}
	\caption{Degeneration eines viereckigen zu einem dreieckigen Element \citep{Zienkiewicz2013}.}
	\label{fig:Degeneration}
\end{figure}
Im Rahmen der Implementierung erfolgt eine Löschung der doppelten Information des vierten Knotens an entsprechender Stelle, sodass im weiteren Verlauf von einem dreieckigen Element mit drei Knoten ausgegangen werden kann.


Nach erfolgter Degenerierung liegen lineare, trianguläre Elemente mit drei Knotenpunkten vor, die jeweils drei Verschiebungsfreiheitsgrade \(u_x, u_y, u_z\) besitzen, wie in der Abbildung (\ref{fig:2,5dim_FE}) dargestellt.
\begin{figure}[H]
	\hspace*{47mm}
	%	\centering
	\includesvg[height=5cm,keepaspectratio]{svg/2,5dim_FE}
	\caption{2,5-dimensionales dreickiges finites Element.}
	\label{fig:2,5dim_FE}
\end{figure}
Diese Freiheitsgrade lassen sich in einen Vektor der Knotenverschiebungen $\tilde{\mathbf u}_{n}^{T}$ zusammenfassen:
\begin{equation}\label{eq:un_rowvec}
	\tilde{\mathbf u}_{n}^{T}
	=\bigl(\tilde{u}_{x1}\; \tilde{u}_{y1}\; \tilde{u}_{z1}\; \tilde{u}_{x2}\; \tilde{u}_{y2}\; \tilde{u}_{z2}\;
	\tilde{u}_{x3}\; \tilde{u}_{y3}\; \tilde{u}_{z3}\bigr)
\end{equation}
\clearpage
Mittels Ansatzfunktionen werden die Verschiebungsfelder innerhalb der Dreiecke interpoliert. Für degenerierte Dreiecke formuliert \cite{Zienkiewicz2013} die Ansatzfunktionen durch die normierten Koordinaten $\eta,\ \zeta \in [-1,1]$ wie folgt:
\begin{subequations}\label{eq:shape_functions}
	\begin{align}
		N_1 &= \frac{1}{4}\,(1-\zeta)(1-\eta) \label{eq:shape_funcs_a}\\
		N_2 &= \frac{1}{4}\,(1+\zeta)(1-\eta) \label{eq:shape_funcs_b}\\
		N_3 &= \frac{1}{2}\,(1+\eta)        \label{eq:shape_funcs_c}
	\end{align}
\end{subequations}
Es besteht demnach folgende Beziehung zwischen dem Vektor der Verschiebungen $\tilde{\mathbf u}$ und dem Vektor der Knotenverschiebungen $\tilde{\mathbf u}_{n}$:
\begin{equation}\label{eq:u_interp}
	\tilde{\mathbf u} \;=\; \mathbf N \cdot \tilde{\mathbf u}_{n}\,
\end{equation}
Die Matrix \(\mathbf{N}\) enthält dabei die Ansatzfunktionen $N_1, N_2, N_3$ und lässt sich wie folgt formulieren:
\begin{equation}\label{eq:N_matrix}
	\begingroup
	\setlength{\arraycolsep}{2pt}      % Spaltenabstand in Matrizen kleiner (Standard ~5pt)
	\renewcommand{\arraystretch}{0.95} % (optional) etwas geringerer Zeilenabstand
	\mathbf N=
	\begin{bmatrix}
		N_1(\eta,\zeta)&0&0& N_2(\eta,\zeta)&0&0& N_3(\eta,\zeta)&0&0\\
		0&N_1(\eta,\zeta)&0& 0&N_2(\eta,\zeta)&0& 0&N_3(\eta,\zeta)&0\\
		0&0&N_1(\eta,\zeta)& 0&0&N_2(\eta,\zeta)& 0&0&N_3(\eta,\zeta)
	\end{bmatrix}
	\endgroup
\end{equation}

Für die nachfolgende Berechnung werden sowohl die Matrix \(\mathbf{N}\) als auch die Matrix \(\bar{\mathbf{B}}\) herangezogen. 
Die Matrix $\bar{\mathbf{B}}$ enthält die Ableitungen der Ansatzfunktionen (\ref{eq:shape_functions}) im Wellenzahlbereich $k_x$, um Verzerrungen aus den Knotenverschiebungen zu berechnen \citep{Hackenberg2016}.
Die Einträge dieser Matrix sind im Anhang (\ref{sec:B_Matrix}) zu finden.



\section{Dynamische Steifigkeitsmatrix des 2,5D FEM-Teilsystems}
\label{sec:twofiveD_FEM}

Die Herleitung der Elementsteifigkeitsmatrix erfolgt gemäß dem Prinzip der virtuellen Arbeiten (PvA) im zweifach fouriertransformierten Wellenzahl-Frequenz-Raum. Dieses Prinzip besagt, dass die Summe aller virtuellen Arbeiten, die durch die im System wirkenden Kräfte verursacht werden, gleich null sein muss.
Für das elastische Kontinuum setzen sich die Arbeitsanteile aus der inneren virtuellen Arbeit $\delta W_i$, der virtuellen Arbeit infolge der d'Alembert'schen Trägheit $\delta W_T$ und der äußeren virtuellen Arbeit $\delta W_a$ zusammen \citep{Klein2003}:
\begin{equation}\label{eq:PvA}
	\delta W \;=\; \delta W_i + \delta W_T + \delta W_a \;=\; 0 \,
\end{equation}
Da die Berechnung im Wellenzahl-Frequenz-Raum erfolgt, müssen die virtuellen Arbeitsanteile im transformierten Raum formuliert werden.
Die zweifach fouriertransformierten Größen werden mit dem Symbol $\tilde{}$ gekennzeichnet.

Unter Berücksichtigung der Ansatzfunktionen (\ref{eq:shape_functions}) sowie der Verschiebungs-Verzerrungs-\\beziehung $\tilde{\boldsymbol{\varepsilon}}=\bar{\mathbf B}\,\tilde{\mathbf u}_{n}$ und der Spannungs-Verzerrungs-Beziehung $\tilde{\sigma} = \mathbf D\,\tilde{\varepsilon}$, kann das Prinzip der virtuellen Arbeiten (\ref{eq:PvA}) nach \cite{Freisinger2022} wie folgt umformuliert werden:
\begin{equation}\label{eq:PvA2}
	-\delta \tilde{\mathbf u}_{n}^{\mathsf H}\,
	\underbrace{\left(\int\limits_{(A)} \bar{\mathbf B}^{\mathsf H}\,\mathbf D\,\bar{\mathbf B}\,\mathrm dA\right)}_{\bar{\mathbf{K}}}\,
	\tilde{\mathbf u}_{n}
	\;+\;
	\delta \tilde{\mathbf u}_{n}^{\mathsf H}\,
	\underbrace{\left(\int\limits_{(A)} \mathbf N^{\mathsf H}\,\tilde{\mathbf p}\,\mathrm dA\right)}_{\tilde{\mathbf p}_{n}}
	\;+\;
	\delta \tilde{\mathbf u}_{n}^{\mathsf H}\,\omega^{2}\,
	\underbrace{\left(\int\limits_{(A)} \rho\,\mathbf N^{\mathsf H}\,\mathbf N\,\mathrm dA\right)}_{\mathbf M}\,
	\tilde{\mathbf u}_{n}
	= 0 
\end{equation}
Das Symbol (\(\cdot^{\mathsf H}\)) kennzeichnet dabei hermitesche Matrizen, die transponiert und anschließend komplex konjugiert wurden \citep{Hackenberg2016}.

In Gleichung (\ref{eq:PvA2}) sind zum einen die Matrizen $\mathbf{N}$ und $\bar{\mathbf B}$ enthalten, die sich aus den Ansatzfunktionen ergeben und in den Gleichungen (\ref{eq:N_matrix}) und (\ref{eq:Bbar_tri3}) gegeben sind. 
Außerdem findet sich auch der Vektor der Knotenverschiebung $\tilde{\mathbf u}_{n}$ aus Gleichung (\ref{eq:u_interp}) wieder.
Des Weiteren ist die Elastizitätsmatrix $\mathbf{D}$ enthalten, welche sich ergibt zu:
\begin{equation}\label{eq:Elastizitätsmatrix}
	\mathbf D =
	\begin{bmatrix}
		\lambda+2\mu & \lambda      & \lambda      & 0 & 0 & 0 \\
		\lambda      & \lambda+2\mu & \lambda      & 0 & 0 & 0 \\
		\lambda      & \lambda      & \lambda+2\mu & 0 & 0 & 0 \\
		0            & 0            & 0            & \mu & 0   & 0 \\
		0            & 0            & 0            & 0   & \mu & 0 \\
		0            & 0            & 0            & 0   & 0   & \mu
	\end{bmatrix}
\end{equation}
Die Laméschen Konstanten $\lambda$ und $\mu$ sind in Gleichung (\ref{eq:lame_konstanten}) gegeben und enthalten die elastischen Größen.


Aus der Gleichung (\ref{eq:PvA2}) lassen sich zudem die Beziehungen für die Steifigkeitsmatrix \(\bar{\mathbf{K}}\), den nodalen Vektor der Belastungen $\tilde{\mathbf p}_{n}$ und die Massenmatrix \(\mathbf{M}\) erkennen. 
Die numerische Lösungen der Integrale ist in Kapitel (\ref{sec:Numerik}) beschrieben.

Die Gleichung (\ref{eq:PvA2}) lässt sich mithilfe der gelösten Integrale in der für die FEM typischen Formulierung darstellen:
\begin{equation}\label{eq:PvA_FEM}
	\bar{\mathbf K}\,\tilde{\mathbf u}_{n}
	- \omega^{2}\,\bar{\mathbf M}\,\tilde{\mathbf u}_{n}
	\;=\;
	\underbrace{\bigl(\bar{\mathbf K}-\omega^{2}\,\bar{\mathbf M}\bigr)}_{\text{dyn.\ Steifigkeitsmatrix}}\,
	\tilde{\mathbf u}_{n}
	\;=\; \tilde{\mathbf p}_{n}\,
\end{equation}
mit der dynamische Steifigkeitsmatrix $\tilde{\mathbf K}(k_x,\omega)=\bar{\mathbf K}-\omega^{2}\bar{\mathbf M}$


Die Kopplung der beiden Teilsysteme FEM und ITM wird in Kapitel (\ref{cha:Kopplung}) mithilfe der dynamischen Steifigkeitsmatrizen $\tilde{\mathbf K}$ der Berechnungsmethoden durchgeführt. 
\cite{Hackenberg2016} formuliert dafür die in Gleichung (\ref{eq:itm_system}) bereits eingeführte allgemeine Beziehung in Blockdarstellung, in deren Matrizen die Freiheitsgrade innerhalb des FE-Netzes $\Omega$ von denen an der Kopplungsfläche $\Gamma_z$ getrennt werden. Diese ergibt sich zu:
\begin{equation}\label{eq:fe_block_system}
	\begin{bmatrix}
		\tilde{\mathbf K}_{\Gamma\Gamma_{\mathrm{FEM}}} & \tilde{\mathbf K}_{\Gamma\Omega_{\mathrm{FEM}}} \\[6pt]
		\tilde{\mathbf K}_{\Omega\Gamma_{\mathrm{FE}}} & \tilde{\mathbf K}_{\Omega\Omega_{\mathrm{FEM}}}
	\end{bmatrix}
	\begin{pmatrix}
		\tilde{\mathbf u}_{\Gamma_{\mathrm{FEM}}} \\[2pt]
		\tilde{\mathbf u}_{\Omega_{\mathrm{FEM}}}
	\end{pmatrix}
	=
	\begin{pmatrix}
		\tilde{\mathbf p}_{\Gamma_{\mathrm{FEM}}} \\[2pt]
		\tilde{\mathbf p}_{\Omega_{\mathrm{FEM}}}
	\end{pmatrix}
\end{equation}
Zur besseren Nachvollziehbarkeit der Kopplung in Kapitel (\ref{cha:Kopplung}) werden die Vektoren und Matrizen des FEM-Teilsystems mit dem Index ($\cdot_{\mathrm{FEM}}$) versehen.



\section{Numerische Implementation}
\label{sec:Numerik}

Im Rahmen des Implementierungsprozesses ist eine analytische Lösung der in Gleichung (\ref{eq:PvA2}) dargestellten Integrale der Steifigkeitsmatrix \(\bar{\mathbf{K}}\) und der Massenmatrix \(\mathbf M\) nicht möglich.
Daher ist die Entwicklung eines numerischen Ansatzes zur Lösung erforderlich.

Für die numerische Integration existieren mehrere Optionen. 
Sofern Randpunkte in die Berechnung miteinbezogen werden, besteht die Möglichkeit, das Integrationsintervall in $n$ äquidistant angeordnete Abschnitte zu unterteilen und es anschließend über $n+1$ Stützstellen zu integrieren. Dieses Verfahren wird als Newton-Cotes-Integration bezeichnet und wird sowohl in \cite{Klein2003} als auch in \cite{Gross2023} vorgestellt.

Im Rahmen der Integration in der FEM findet die Gauß'sche Quadraturformel Anwendung. 
Im Gegensatz zur Newton-Cotes-Integration werden gewichtete, nicht äquidistant verteilte Stützstellen, auch Gaußpunkte (GP) genannt, herangezogen \citep{Gross2023}.

In der von \cite{Hackenberg2016} entwickelten Implementierung werden hierzu $n_{\mathrm{GP}} = 4$ Gaußpunkte verwendet, wobei jeweils zwei Punkte je Koordinatenrichtung definiert sind, wie in der Abbildung (\ref{fig:GP_Hackenberg}) dargestellt.
\begin{figure}[H]
	\hspace*{55mm}
	%	\centering
	\includesvg[height=5cm,keepaspectratio]{svg/GP Hackenberg}
	\caption{Gaußpunkte für viereckige, finite Elemente.}
	\label{fig:GP_Hackenberg}
\end{figure}
Die entsprechenden Stützstellen und Wichtungsfaktoren für $n_{\mathrm{GP}} = 4$ Gaußpunkte, die für die Implementierung viereckiger finiter Elemente von \cite{Hackenberg2016} verwendet werden, sind in der Tabelle (\ref{tab:Gp_Hackenberg}) aufgeführt.\\
\begin{table}[htb]\centering
	{\small
		\setlength{\tabcolsep}{10pt}            % etwas enger
		\renewcommand{\arraystretch}{1.25}      % moderater Zeilenabstand
		\begin{tabular}{@{}lccc@{}}            % @{} entfernt Außenränder
			\firsthline
			& $\eta$                 & $\zeta$                & $w$ \\\hline
			$\mathrm{GP}_1$  & $-\tfrac{1}{\sqrt{3}}$   & $-\tfrac{1}{\sqrt{3}}$   & $1$   \\
			$\mathrm{GP}_2$  & $-\tfrac{1}{\sqrt{3}}$   &  $\tfrac{1}{\sqrt{3}}$     & $1$   \\
			$\mathrm{GP}_3$  &  $\tfrac{1}{\sqrt{3}}$     & $-\tfrac{1}{\sqrt{3}}$   & $1$   \\
			$\mathrm{GP}_4$  &  $\tfrac{1}{\sqrt{3}}$     &  $\tfrac{1}{\sqrt{3}}$     & $1$   \\\lasthline
	\end{tabular}}
	\caption{Koordinaten und Wichtungsfaktoren der Gaußpunkte für viereckige, finite Elemente \citep{Hackenberg2016}.}
	\label{tab:Gp_Hackenberg}
\end{table}

Im Allgemeinen werden für die numerische Integration von nativen Dreiecken alternative Integrationsregeln gewählt. 
In den Arbeiten von \cite{Gross2023} und \cite{Zienkiewicz2013} wird auf spezifische Gaußpunkte für die numerische Integration von Dreiecken verwiesen, die sich auf die in Gleichung (\ref{eq:tri_shape_linear}) vorgestellten Ansatzfunktionen eines nativen Dreiecks beziehen.

In Kapitel (\ref{sec:degenerierte Dreiecke}) wird beschrieben, dass im Rahmen der vorliegenden Arbeit der Degenerationsansatz mit den Ansatzfunktionen (\ref{eq:shape_functions}) verwendet wird. Wie den Ansatzfunktionen zu entnehmen ist, beziehen sich die implementierten Ansatzfunktionen auf die normierten Koordinaten $\eta,\ \zeta \in [-1,1]$ der viereckigen, finiten Elemente.
Die Anwendung spezifischer Formeln, die sich auf baryzentrische Koordinaten beziehen, ist demnach nicht zulässig.

In der entwickelten Implementierung wurde daher auf die Gauß'sche Quadratur mit $n_{\mathrm{GP}} = 4$ Gaußpunkten, analog zu der Implementierung von \cite{Hackenberg2016}, zurück gegriffen.
Die genutzten Gaußpunkte sind in der Abbildung (\ref{fig:GP_Degeneration}) dargestellt und deren Koordinaten $\eta$, $\zeta$ sowie die entsprechenden Wichtungsfaktoren sind bereits in der Tabelle (\ref{tab:Gp_Hackenberg}) aufgeführt.
\begin{figure}[H]
	\hspace*{50mm}
	%	\centering
	\includesvg[height=5cm,keepaspectratio]{svg/GP Degeneration}
	\caption{Gaußpunkte für degenerierte, dreieckige, finite Elemente.}
	\label{fig:GP_Degeneration}
\end{figure}

Die numerische Lösung der in Gleichung (\ref{eq:PvA2}) dargestellten Integralgleichungen erfolgt demnach unter Verwendung der Gauß'schen Quadratur, die in Anhang (\ref{sec:Numerische Integration}) definiert ist, und der in Tabelle (\ref{tab:Gp_Hackenberg})  definierten Gaußpunkte, wie nachfolgend dargestellt:
\begin{subequations}\label{eq:KM_1D_gp}
	\begin{align}
		\mathbf{\bar K}
		&= \sum_{k=1}^{n_{\mathrm{GP}}}
		\mathbf{\bar B}(k_x,\eta_k,\zeta_k)^{\mathsf H}\,
		\mathbf{D'}\,
		\mathbf{\bar B}(k_x,\eta_k,\zeta_k)\,
		\det(\mathbf{J})\,w \label{eq:K_1D_gp}\\[0.5ex]
		\mathbf{M}
		&= \sum_{k=1}^{n_{\mathrm{GP}}}
		\rho\,
		\mathbf{N}(\eta_k,\zeta_k)^{\mathsf H}\,
		\mathbf{N}(\eta_k,\zeta_k)\,
		\det(\mathbf{J})\,w \label{eq:M_1D_gp}
	\end{align}
\end{subequations}
Die Jacobi-Matrix $\mathbf{J}$ stellt die Transformation und Relation der normierten Elementkoordinaten und globalen Koordinaten herstellt.

Der nodale Vektor der Belastungen $\tilde{\mathbf p}_{n}$ wird nicht durch das Integral aus Gleichung (\ref{eq:PvA2}) gelöst.
Stattdessen ergibt er sich aus dem Vektor der Knotenverschiebungen $\tilde{\mathbf{u}}_{n}$ und der dynamischen Steifigkeitsmatrix $\tilde{\mathbf K}$ nach Gleichung (\ref{eq:PvA_FEM}).
	\chapter{Kopplung der FEM mit der ITM}
\label{cha:Kopplung}

\section{Vorbemerkung}
\label{sec:Vorbem_Kopplung}

Die Auswertung der Lösung für das Gesamtsystem erfordert eine Kopplung der Teilsysteme ITM und FEM.
Die Kopplung der beiden Berechnungsmethoden wird durch die Übergangsbedingungen an der Kopplungsfläche $\Gamma_z$ erreicht, da an diesen Stellen zum einen ein Kräftegleichgewicht herrschen muss und zum anderen die Verschiebungen identisch sein müssen.

Die Übergangsbedingungen der Kopplung werden im Kapitel (\ref{sec:Kopplung}) angewendet. Im Vorfeld ist zu berücksichtigen, dass sich die Teillösungen auf die gleichen Koordinaten beziehen müssen. 
Die erforderliche Transformation ist in Kapitel (\ref{sec:Transformation_Kopplung}) beschrieben.
\begin{figure}[H]
	\hspace*{37mm}
	%	\centering
	\includesvg[height=5cm,keepaspectratio]{svg/Kopplung}
	\caption{ITM und FEM Flächendefinitionen - basiert auf \citep{Freisinger2022}.}
	\label{fig:Kopplung}
\end{figure}

\section{Transfomation der Lösung des FE-Teilsystems}
\label{sec:Transformation_Kopplung}
Bei der Kopplung ist zu berücksichtigen, dass die hergeleiteten Lösungen sich auf unterschiedliche Koordinatensysteme beziehen sowie in verschiedene Fourierräume transformiert sind, wie in der Übersicht (\ref{tab:Übersichtstabelle}) dargestellt.
\begin{table}[htb]
	\centering
	\normalsize
	\begin{tabular}{ccc}
		%\firsthline
		& \textbf{FEM} & \textbf{ITM} \\\hline
		Koordinatensystem auf $\Gamma$ & kartesisch $(x,y,z)$ & Zylinderkoordinaten $(x,r,\varphi)$ \\
		Fourierraum                     & zweifachtransformiert $(k_x,\omega)$        & dreifachtransformiert$(k_x,n,\omega)$ \\\lasthline
	\end{tabular}
	\caption{Übersicht der Fourierräume und Koordinatensysteme von FEM und ITM.}
	\label{tab:Übersichtstabelle}
\end{table}\\

Die Übergangsbedingungen müssen folglich so formuliert werden, dass sich die Lösungen der Teilsysteme auf identische Größen beziehen.
Für die Kopplung sind demnach Transformationen an der Kopplungsoberfläche $\Gamma$ erforderlich. Gemäß \cite{Freisinger_Hackenberg2020} erfolgt die Transformation der Lösung des FEM-Teilsystems in das der ITM.

Zunächst werden die Verschiebungen des FEM-Teilsystems in das Zylinderkoordinatensystem transformiert.
Dafür greift \cite{Fruehe2010} erneut die Transformationsmatrix $\boldsymbol{\beta}$ aus Gleichung (\ref{eq:Transforationsmatrix}) auf und formuliert den Wechsel der Koordinatensysteme von kartesisch zu Zylinderkoordinaten wie folgt:
\begin{equation}\label{eq:u_FE_Zylinder}
	\tilde{\mathbf u}_{\Gamma,\mathrm{FEM,k}}}
	= \mathbf T_{1}\cdot
	\tilde{\mathbf u}_{\Gamma,\mathrm{FEM,z}}
\end{equation}
Die Einträge der Matrix $\mathbf{T_1}$ ergeben sich aus der transponierten Transformationsmatrix $\boldsymbol{\beta}^{\mathsf T}$ und sind im Anhang (\ref{sec:Koordinatentransformationsmatrix}) aufgeführt.

Im nächsten Schritt ist es erforderlich, $\tilde{\mathbf u}_{\Gamma_{\mathrm{FE,z}}}$ in den selben Fourierraum zu übertragen. Zu diesem Zweck wird eine Fourierreihe bezüglich des Umfangs der zylindrischen Kopplungsfläche $\Gamma$ entwickelt \citep{Hackenberg2016}:
\begin{equation}\label{eq:u_FE_Fourier}
	\tilde{\mathbf u}_{\Gamma,\mathrm{FEM,z}}
	= \mathbf T_{2}\cdot
	\hat{\mathbf u}_{\Gamma,\mathrm{FEM,z}}
\end{equation}
Die Einträge der Matrix \(\mathbf T_{2}\) sind im Anhang (\ref{sec:Fouriertransformationsmatrix}) aufgeführt.

Zusammenfassend lässt sich festhalten, dass durch die Anwendung der Transformationsmatrix \(\mathbf T\) eine Transformation auf einheitliche Koordinaten $(k_x,\ r,\ n,\ \omega)$ erzielt wird.
\begin{equation}\label{eq:tilde_u_FE}
	\tilde{\mathbf u}_{\Gamma,\mathrm{FEM},k}
	= \mathbf T_{1}\cdot\mathbf T_{2}\cdot\hat{\mathbf u}_{\Gamma,\mathrm{FEM},z}
	= \mathbf T\cdot\hat{\mathbf u}_{\Gamma,\mathrm{FEM},z}\,
\end{equation}
Analog dazu erfolgt die Transformation des Vektors der Belastungen $\tilde{\mathbf p}_{\Gamma,\mathrm{FEM,k}}$ in die Koordinaten des ITM-Teilsystems $(k_x,\ r,\ n,\ \omega)$ mifhilfe der Transformationsmatrix \(\mathbf T\) wie folgt:
\begin{equation}\label{eq:tildeP_FE}
	\tilde{\mathbf p}_{\Gamma,\mathrm{FEM,k}}
	= \mathbf T\cdot
	\hat{\mathbf p}_{\Gamma,\mathrm{FEM,z}} \,
\end{equation}



\section{Kopplung der Teilsysteme}
\label{sec:Kopplung}

Nachdem sich die Lösungen der Teilsysteme FEM und ITM nun auf die gleichen Koordinaten $(k_x,\ r,\ n,\ \omega)$ beziehen, können im weiteren Verlauf die Kopplungsbedingungen angewandt werden.

Die erste Kopplungsbedingung fordert, dass die Verschiebungen $\hat{\mathbf u}_{\Gamma}$ der Teilsysteme an der Kopplungsfläche $\Gamma$ identisch sind. Diese Kopplungsbedingung ergibt sich demnach zu:
\begin{equation}\label{eq:Kopplbed_u}
	\hat{\mathbf u}_{\Gamma,\mathrm{FEM}}
	= \hat{\mathbf u}_{\Gamma,\mathrm{ITM}}\,
\end{equation}
Darüber hinaus wird gefordert, dass die auftretenden Kräfte der Teillösungen sich im Gleichgewichtszustand mit potenziellen Randlasten an der Kopplungsfläche $\Gamma$ befinden. Die zweite Kopplungsbedingung lässt sich demnach wie folgt formuliern:
%Des Weiteren ist es essenziell, dass an der Kopplungsfläche $\Gamma$ die auftretenden Kräfte im einem Gleichgewichtszustand sind. Die Lasten der ITM und FEM müssen mit einer potenziellen Randlast auf der Kopplungsfläche $\Gamma$ im Gleichgewicht stehen. Die zweite Kopplungsbedingung lautet daher wie folgt: 
\begin{equation}\label{eq:Kopplbed_p}
	\hat{\mathbf p}_{\Gamma,\mathrm{ITM}}
	+ \frac{1}{\mathrm d s}\,\hat{\mathbf p}_{\Gamma,\mathrm{FEM}}
	= \hat{\mathbf p}_{\Gamma}\,
\end{equation}
Um die knotenweisen FEM-Lasten in stetige Spannungen entlang der Kopplungsfläche umzuwandeln, wird der Vektor der Knotenlasten des FEM-Teilsystems durch die Elementlänge $ds$ geteilt.


\cite{Hackenberg2016} kombiniert die Koplungsbedingungen (\ref{eq:Kopplbed_u}) und (\ref{eq:Kopplbed_p}) mit den Lösungen der Teilsysteme in den Gleichungen (\ref{eq:itm_block_system_braced}) und (\ref{eq:fe_block_system}), sodass das folgende System in Blockdarstellung resultiert:
\begin{equation}\label{eq:gekoppelt}
	\begin{bmatrix}
		\hat{\mathbf K}_{\Lambda\Lambda,\mathrm{ITM}} &
		\hat{\mathbf K}_{\Lambda\Gamma,\mathrm{ITM}}   &
		0 \\[6pt]
		\hat{\mathbf K}_{\Gamma\Lambda,\mathrm{ITM}}   &
		\hat{\mathbf K}_{\Gamma\Gamma,\mathrm{ITM}}
		+ \dfrac{1}{\mathrm d s}\,\mathbf T^{-1}\,\tilde{\mathbf K}_{\Gamma\Gamma,\mathrm{FEM}}\,\mathbf T &
		\dfrac{1}{\mathrm d s}\,\mathbf T^{-1}\,\tilde{\mathbf K}_{\Gamma\Omega,\mathrm{FEM}} \\[6pt]
		0 &
		\tilde{\mathbf K}_{\Omega\Gamma,\mathrm{FEM}}\,\mathbf T &
		\tilde{\mathbf K}_{\Omega\Omega,\mathrm{FEM}}
	\end{bmatrix}
	\begin{pmatrix}
		\hat{\mathbf u}_{\Lambda,\mathrm{ITM}}\\[2pt]
		\hat{\mathbf u}_{\Gamma}\\[2pt]
		\tilde{\mathbf u}_{\Omega,\mathrm{FEM}}
	\end{pmatrix}
	=
	\begin{pmatrix}
		\hat{\mathbf p}_{\Lambda,\mathrm{ITM}}\\[2pt]
		\hat{\mathbf p}_{\Gamma}\\[2pt]
		\tilde{\mathbf p}_{\Omega,\mathrm{FEM}}
	\end{pmatrix}
\end{equation}
Das Resultat ist ein System, das die beiden Teilsysteme ITM und FEM miteinander vereint. 
Die Größen werden in Blockdarstellung getrennt auf der Halbraumoberfläche $\Lambda$, der gemeinsamen Kopplungfläche $\Gamma$ sowie dem Inneren des FE-Bereichs $\Omega$ dargestellt.

Um eine auswertbare Endlösung des gekoppelten Systems zu erhalten, müssen die Größen mittels Fourierrücktransformation nach der Definition in der Gleichung (\ref{eq:invfouriertransformation}) aus dem Fourierraum in den Originalraum rücktransformiert werden.

%Um eine Endlösung des gekoppelten Systems zu erhalten, müssen die Größen mittels Inverser Fouriertransformation aus dem Fourierraum in den Originalraum zurückgeführt werden. Erst dann sind die Teilsysteme final auswertbar gekoppelt.
	% !TeX root = a_main_thesis.tex
\chapter{Verifizierung}
\label{cha:Verifizierung}

\section{Vorbemerkung}
\label{sec:Vorbem_Verifizierung}
Um eine korrekte Implementierung sicherzustellen, ist eine Verifizierung des in den vorangegangenen Kapiteln beschriebenen ITM-FEM-Ansatzes zur Untersuchung eines homogenen Halbraums mit zylindrischem Hohlraum erforderlich.

Zu diesem Zweck werden validierte Referenzlösungen herangezogen, die im Rahmen der Dissertation von \cite{Freisinger2022} entwickelt wurden. In der vorliegenden Arbeit erfolgt die Verifizierung anhand des gegebenen ITM-FEM-Ansatzes, der den zylindrischen Hohlraum mit viereckigen, finiten Elementen beschreibt, sowie eines ITM-Ansatzes, der einen homogenen Halbraum modelliert.

Die Verifizierung der entwickelten Implementierung erfolgt anhand drei unterschiedlicher Systeme. %der Lösung von drei voneinander unabhängigen Belastungs- und Systemfällen.
Im Rahmen des ersten Falls werden in Kapitel (\ref{sec:verification_c1}) die Auswirklungen einer Belastung auf der Halbraumoberfläche $\Lambda$ im FE-Bereich $\Omega$ untersucht.
Der zweite Fall verifiziert in Kapitel (\ref{sec:verification_c2}) die entwickelte Implementation für eine Belastung innerhalb des FE-Bereichs $\Omega$.
In Kapitel (\ref{sec:verification_c3}) wird die Verifizierung des Sonderfalls eines Grabens, also des halben FE-Bereichs $\Omega$ an der Halbraumoberfläche, unter einer Belastung an der Halbraumoberfläche $\Lambda$ durchgeführt.


Im Rahmen der vorliegenden Arbeit erfolgt die Auswertung der entwickelten Implementierung für eine harmonische Belastung der Frequenz $f = 30\ \mathrm{Hz}$. 
Darüber hinaus werden die Materialparameter, die in der Tabelle (\ref{tab:Material_Parameter}) aufgeführt sind, herangezogen. Diese wurden im Rahmen der gegebenen Implementierung von \cite{Freisinger2022} vordefiniert.
% sowie auf die vordefinierten Materialparameter aus der Arbeit von \cite{Freisinger2022} zurückgegriffen, die in der folgenden Tabelle (\ref{tab:Material_Parameter}) aufgeführt sind.
\begin{table}[htb]\centering\normalsize
	{\renewcommand{\arraystretch}{1.25} % <-- Zeilenabstandfaktor
		\begin{tabular}{lcccccc}
			\firsthline
			& $E\;(\mathrm{N}/\mathrm{m}^{2})$ & $\nu\;(-)$ & $\rho\;(\mathrm{kg}/\mathrm{m}^{3})$ & $c_s\;(\mathrm{m}/\mathrm{s})$ & $c_r\;(\mathrm{m}/\mathrm{s})$ & $\lambda_r\;(\mathrm{m})$ \\\hline
			\texttt{soft\_v01}  & $2,6\times10^{7}$ & 0,3 & 2000 & 70,7 & 65,0 & 2,17 \\
			\texttt{stiff\_v01} & $2,6\times10^{8}$ & 0,3 & 1600 & 250,0 & 230,0 & 7,67 \\\lasthline
	\end{tabular}}
	\caption{Herangezogene Bodenparameter.}	\label{tab:Material_Parameter}
\end{table}
\\Zur Vergleichbarkeit der Ergebnisse wird der Tanimoto-Koeffizient $T$ herangezogen, wie bereits in den Dissertationen von \cite{Hackenberg2016} und \cite{Freisinger2022}. Dieser Wert vergleicht die beiden Vektoren $\mathbf{a}$ und $\mathbf{b}$ über die $n$ Elemente innerhalb des betrachteten Intervalls. Die Berechnung wird wie folgt durchgeführt \citep{Willett1998}:
\begin{equation}\label{eq:tanimoto}
	T \;=\; 
	\frac{\displaystyle \sum_{i=1}^{n} a_i\,b_i}
	{\displaystyle \sum_{i=1}^{n} a_i^{2}
		\;+\; \sum_{i=1}^{n} b_i^{2}
		\;-\; \sum_{i=1}^{n} a_i\,b_i } \,
\end{equation}
Es wird ein Wert von $T = 1$ bei identischen Vektoren erwartet. Im Rahmen der Verifikation sollte sich für die Abweichungen der Verschiebungen folglich ein Tanimoto-Koeffizient von $T \approx 1$ ergeben.





\section{Fall 1: Belastung auf der Halbraumoberfläche}
\label{sec:verification_c1}

Im ersten Verifizierungsfall werden die Abweichungen infolge einer Belastung auf der Halbraumoberfläche $\Lambda$ analysiert. Eine Darstellung der gewählten Systemparameter ist in Abbildung (\ref{fig:Refsystem_v1_hs_cyl}) visualisiert.
\begin{figure}[H]
	\hspace*{35mm}
	%	\centering
	\includesvg[height=6cm,keepaspectratio]{svg/verification_c1_system_hs_cyl}
	\caption{Gewählte Systemparameter zur Verifikation mit viereckigen Elementen im ersten Fall - basiert auf \cite{Freisinger2022}.}
	\label{fig:Refsystem_v1_hs_cyl}
\end{figure}
Die quadratische Belastung weist Breiten von $b_x = b_y = 2\ \mathrm{m}$ auf und die Belastungsamplitude wurde zu $1 \mathrm{N}/\mathrm{m}^{2}$ gewählt.
Darüber hinaus wurde eine Gesamtausdehnung des ITM-Teilsystems zu $B_x = B_y = 128\ \mathrm{m}$ und $N_x = N_y = 2^{9}$ Stützstellen gewählt.

Es wurde der Radius $R = 4\ \mathrm{m}$ gewählt und der zylindrische Hohlraum aus der Mitte heraus um $y_{_{\mathrm{T}}} = 8\ \mathrm{m}$ verschoben. Zudem wurde der Mittelpunkt auf eine Tiefe von $H = 8\ \mathrm{m}$ eingebunden.

Der erste Verifizierungsfall erfolgt mit den den Materialparametern \texttt{stiff\_v01}.


\subsubsection{Einfluss der Knotenanzahl auf der Kopplungsfläche $\Gamma$}
Um den Einfluss der Knoten $N_{\varphi}$ auf der umlaufenden Zylinderkopplungsfläche $\Gamma$ auszuwerten, werden zunächst die Ergebnisse der FE-Netze mit viereckigen und dreieckigen Elementen für $N_{\varphi} = 32$ und $64$ Knoten auf der Mittellinie des FE-Bereichs $\Omega$ analysiert.
\begin{figure}[H]
	\centering
	\begin{subfigure}{0.49\linewidth}
		\centering
		\includesvg[width=\linewidth]{svg/verification_c1_32nodes_svg_tex}
	%	\subcaption{$|u_z(y)|$ für $N_{\phi} = 32$ Knoten}
		\label{fig:c1_a}
	\end{subfigure}\hfill
	\begin{subfigure}{0.49\linewidth}
		\centering
		\includesvg[width=\linewidth]{svg/verification_c1_left_svg_tex}
	%	\subcaption{$|u_z(y)|$ für $N_{\phi} = 64$ Knoten}
		\label{fig:c1_b}
	\end{subfigure}
	\caption{Abweichungen $|\bar{u}_z(y)|$ der dreieckigen Elemente (\legThree) zu der Referenzlösung viereckige Elemente (\legFour) bei unterschiedlicher Diskretisierung $N_{\varphi} = 32$ (links) und $N_{\varphi} = 64$ (rechts).}
	\label{fig:c1_Knoten}
\end{figure}
\begin{table}[htb]
	\centering
	\normalsize
	{\renewcommand{\arraystretch}{1.15}
		\begin{tabular}{ccccc}
			\firsthline
			$N_{\varphi}$ & Maximaler Wert & Maximale Abweichung & Tanimoto-Koeffizient $T$ \\\hline
			$ 32 $ & $ 3,90\cdot10^{-10}$ & $6,07\cdot10^{-12}$ & 0,99 \\
			$ 64 $ & $3,86\cdot10^{-10}$ & $1,77\cdot10^{-12}$ & 1.00 \\\lasthline
	\end{tabular}}
	\caption{Abweichungen $|\bar{u}_z(y)|$ zur Referenzlösung viereckige Elemente bei unterschiedlicher Diskretisierung $N_\varphi$ im ersten Verifizierungsfall.}
	\label{tab:Fehlermessung_c1}
\end{table}
Die resultierenden vertikalen Verschiebungen $|\bar{u}_z(y)|$ für eine unterschiedliche Anzahl von Knoten $N_{\varphi}$ werden in Abbildung (\ref{fig:c1_Knoten}) dargestellt. 
Aus den Graphen ist erkennbar, dass die Angleichung an die Referenzlösung bei gewählten $N_{\varphi} = 64$ Knoten eine höhere Ähnlichkeit aufweist als die Lösung bei $N_{\varphi} = 32$ Knoten.
Begründen lässt sich diese Beobachtung durch die eine präzisere Approximation, da eine größere Anzahl von Knoten bei konstantem Radius $R = 4 m$ zu einer genaueren Diskretisierung führt. 
Diese Beobachtung spiegelt sich ebenfalls in den aufgeführten Werten der Tabelle (\ref{tab:Fehlermessung_c1}) wider, da sowohl die maximale Abweichung, als auch der Tanimoto-Koeffizient $T$ größere Abweichungen aufweisen.

Über diese begründbaren Abweichungen hinaus kann eine hohe Übereinstimmung zwischen der Lösung für viereckige und dreieckige Elemente festgestellt werden. Tanimoto-Koeffizienten von $T \approx 1$ bzw. $T = 1$ sprechen für eine präzise Implementierung in diesem ersten Verifizierungsfall.


\subsubsection{Abweichung zur Referenzlösung homogener Halbraum}
Darüber hinaus wird die Abweichung der Lösung mit der Referenzlösung des homogenen Halbraums untersucht. Dafür wird die Implementierung des ITM-Ansatzes homogener Halbraum herangezogen, wobei die gleiche Belastung an derselben Stelle aufgebracht wird. Um die Verschiebungen in der Tiefe $H = 8\ \mathrm{m}$ auszuwerten, wird in den homogenen Halbraum eine zusätzliche Bodenschicht mit den gleichen Materialparametern \texttt{stiff\_v01} eingefügt. Ein Systemdarstellung ist in der Abbildung (\ref{fig:c1_homog}) links zu finden.
\begin{figure}[H]
	\centering
  \begin{subfigure}[t]{0.49\linewidth}
	\centering
	\raisebox{12mm}{% <-- hier die Höhe anpassen (z.B. 4–10mm)
		\hspace*{9mm}% <-- nach rechts schieben
		\makebox[\linewidth][c]{\includesvg[width=1.1\linewidth]{svg/verification_c1_system_hs}}
	}
\end{subfigure}\hfill
	\begin{subfigure}{0.49\linewidth}
		\centering
		\includesvg[width=\linewidth]{svg/verification_c1_homog_svg_tex}
	\end{subfigure}
	\caption{Gewählte Systemparameter zur Verifikation mit dem homogenen Halbraum im ersten Fall - basiert auf \cite{Freisinger2022} - (links) und Abweichung $|\bar{u}_z(y)|$ der dreieckigen Elemente (\legThree) zu der Referenzlösung homogener Halbraum (\legFour) (rechts).}
	\label{fig:c1_homog}
\end{figure}
\begin{table}[htb]
	\centering
	\normalsize
	{\renewcommand{\arraystretch}{1.15}
		\begin{tabular}{ccccc}
			\firsthline
			Maximaler Wert & Maximale Abweichung & Tanimoto Koeffizient $T$ \\\hline
			$3,86\cdot10^{-10}$ & $1,21\cdot10^{-12}$ & 1,00 \\\lasthline
	\end{tabular}}
	\caption{Abweichungen $|\bar{u}_z(y)|$ zur Referenzlösung homogener Halbraum im ersten Verifizierungsfall.}
	\label{tab:Fehlermessung_c1homog}
\end{table}
Die resultierenden Abweichungen der vertikalen Verschiebungen $|\bar{u}_z(y)|$ in der rechten Abbildung (\ref{fig:c1_homog}) zeigen, dass auch in diesem Fall die entwickelte Implementierung dreieckiger Elemente eine hohe Übereinstimmung mit der Referenzlösung aus dem homogenen Raum aufweist.
Das lässt sich ebenfalls mithilfe des Tanimoto-Koeffizienten von $T = 1$ aus der Tabelle (\ref{tab:Fehlermessung_c1homog}) feststellen.

Zusammenfassend lässt sich festhalten, dass die dreieckige Implementierung im ersten Verifizierungsfall eine sehr gute Approximation der Referenzlösungen von \cite{Freisinger2022} darstellt. 
Insbesondere führt eine höhere Wahl der Knoten $N_{\varphi}$ auf der umlaufenden Zylinderkopplungsfläche $\Gamma$ aufgrund der feineren Diskretisierung zu einem präziseren Ergebnis.




\section{Fall 2: Belastung innerhalb des FE-Bereichs}
\label{sec:verification_c2}

Der zweite Verifizierungsfall befasst sich mit den Auswirkungen infolge einer Belastung innerhalb des FE-Bereichs $\Omega$ bei unterschiedlicher Wahl der Materialparameter des Bodens.
\begin{figure}[H]
	\centering
	\begin{subfigure}[t]{0.49\linewidth}
		\centering
		\includesvg[height=4.0cm]{svg/verification_c2_system_hs_cyl}
	\end{subfigure}\hfill
	\begin{subfigure}[t]{0.49\linewidth}
		\centering
		\includesvg[height=4.0cm]{svg/verification_c2_system_hs}
	\end{subfigure}
	\caption{Gewählte Systemparameter zur Verifikation mit viereckigen Elementen (links) und mit dem homogenen Halbraum (rechts) im zweiten Fall - basiert auf \cite{Freisinger2022}.}
	\label{fig:c2_systeme}
\end{figure}
Analog zum ersten Verifizierungsfall in Kapitel (\ref{sec:verification_c1}) wird eine quadratische Belastung mit den Breiten $b_x = b_y = 2m$ und der Amplitude $1 \mathrm{N}/\mathrm{m}^{2}$ aufgebracht. Die Gesamtausdehnung des ITM-Teilsystems sowie die gewählten Stützstellen bleiben mit $B_x = B_y = 128\ \mathrm{m}$ und $N_x = N_y = 2^{9}$ unverändert.

Die Belastung wird mittig im zylindrischen Hohlraum mit einem Radius von $R = 4\ \mathrm{m}$ in einer Tiefe von $H = 8\ \mathrm{m}$ aufgebracht. Eine Darstellung der gewählten Systemparameter ist in den Abbildungen (\ref{fig:c2_systeme}) gegeben.

Aufgrund der erfolgten Analyse des Einflusses der Knoten $N_{\varphi}$ auf der umlaufenden Zylinderkopplungsfläche $\Gamma$ in Abschnitt (\ref{sec:verification_c1}) wird im Folgenden eine Analyse der vertikalen Verschiebungen für $N_{\varphi} = 64$ Knoten vorgenommen.

In diesem Kapitel erfolgt die Analyse der vertikalen Verschiebungen im Vergleich zu den Referenzlösungen für die unterschiedlichen Materialparameter \texttt{soft\_v01} und \texttt{stiff\_v01}.


\subsubsection{Abweichung zur Referenzlösung eines steiferen Materials}
Zunächst werden die Materialparameter \texttt{stiff\_v01} herangezogen.
\begin{figure}[H]
	\centering
	\begin{subfigure}{0.49\linewidth}
		\centering
		\includesvg[width=\linewidth]{svg/verification_c2_left_svg_tex}
	\end{subfigure}\hfill
	\begin{subfigure}{0.49\linewidth}
		\centering
		\includesvg[width=\linewidth]{svg/verification_c2_homog_svg_tex}
	\end{subfigure}
	\caption{Abweichung $|\bar{u}_z(y)|$ der dreieckigen Elemente (\legThree) zu den Referenzlösungen (\legFour) viereckiges FE-Netz (links) und homogener Halbraum (rechts) bei steiferen Materialparametern im zweiten Verifizierungsfall.}
	\label{fig:c2}
\end{figure}
\begin{table}[htb]
	\centering
	\normalsize
	{\renewcommand{\arraystretch}{1.15}
		\begin{tabular}{ccccc}
			\firsthline
			Referenzlösung & Maximaler Wert & Maximale Abweichung & Tanimoto-Koeffizient $T$ \\\hline
			FE-Netz & $ 3,39\cdot10^{-9}$ & $2,69\cdot10^{-11}$ & 1,00 \\
			homogener Halbraum & $3,39\cdot10^{-9}$ & $1,03\cdot10^{-11}$ & 1,00 \\\lasthline
	\end{tabular}}
	\caption{Abweichungen $|\bar{u}_z(y)|$ bei steiferen Materialparametern im zweiten Verifizierungsfall.}
	\label{tab:Fehlermessung_c2}
\end{table}
Aus den resultierenden Abweichungen der vertikalen Verschiebungen $|\bar{u}_z(y)|$ in den Abbildungen (\ref{fig:c2}) wird ersichtlich, dass die entwickelte Implementierung die herangezogenen Referenzlösungen infolge einer Belastung innerhalb des FE-Bereichs $\Omega$ präzise abbildet.

Sowohl bei der im Vergleich zur Referenzlösung mit viereckigen Elementen als auch der des homogenen Halbraums werden Tanimoto-Koeffizienten von $T = 1,00$ erreicht. Die erfolgte Implementierung kann für die gewählten Materialparameter \texttt{stiff\_v01} demnach als sehr präzise bewertet werden.


\subsubsection{Abweichung zur Referenzlösung eines weicheren Materials}
Nach der Analyse des steiferen Bodens erfolgt nun die Betrachtung der weicheren Materialparameter \texttt{soft\_v01}.
\begin{figure}[H]
	\centering
	\begin{subfigure}{0.49\linewidth}
		\centering
		\includesvg[width=\linewidth]{svg/verification_c2_left_soft_svg_tex}
	\end{subfigure}\hfill
	\begin{subfigure}{0.49\linewidth}
		\centering
		\includesvg[width=\linewidth]{svg/verification_c2_homog_soft_svg_tex}
	\end{subfigure}
	\caption{Abweichung $|\bar{u}_z(y)|$ der dreieckigen Elemente (\legThree) zu den Referenzlösungen (\legFour) viereckiges FE-Netz (links) und homogener Halbraum (rechts) bei weicheren Materialparametern im zweiten Verifizierungsfall.}
	\label{fig:c2_soft}
\end{figure}
\begin{table}[htb]
	\centering
	\normalsize
	{\renewcommand{\arraystretch}{1.15}
		\begin{tabular}{ccccc}
			\firsthline
			Referenzlösung & Maximaler Wert & Maximale Abweichung & Tanimoto-Koeffizient $T$ \\\hline
			FE-Netz & $ 2,05\cdot10^{-8}$ & $6,48\cdot10^{-10}$ & 0,99 \\
			homogener Halbraum & $2,02\cdot10^{-8}$ & $4,24\cdot10^{-10}$ & 0,99 \\\lasthline
	\end{tabular}}
	\caption{Abweichungen $|\bar{u}_z(y)|$ bei weicheren Materialparametern im zweiten Verifizierungsfall.}
	\label{tab:Fehlermessung_c2_soft}
\end{table}
Die resultierenden Abweichungen der vertikalen Verschiebungen $|\bar{u}_z(y)|$ in den Abbildungen (\ref{fig:c2_soft}) zeigen gute Approximation an die Referenzlösungen. Die in Tabelle (\ref{tab:Fehlermessung_c2_soft}) dargestellten maximalen Abweichungen und Tanimoto-Koeffizienten von $T = 0,99$ sprechen ebenfalls für diese Interpretation. 

Größere Abweichungen im Vergleich zu den Ergebnissen der steiferen Materialparameter \texttt{stiff\_v01} sind jedoch erkennbar. Gerade im linken Graphen der Abbildungen (\ref{fig:c2_soft}) sind Abweichungen der vertikalen Verschiebungen $|\bar{u}_z(y)|$ am Scheitelpunkt erkennbar.
Aufgrund der Tatsache, dass sich beide Analysen dieses Kapitels auf das selbe gewählte System (\ref{fig:c2_systeme}) beziehen und identisch diskretisiert wurden, lassen sich die festgestellten Abweichungen demnach auf die weicheren Materialparameter \texttt{soft\_v01} zurückführen.

Diese größeren Abweichungen lassen sich durch die Verwendung einer gröberen Diskretisierung für die gewählten Parameter erklären. Gemäß \cite{Freisinger_Hackenberg2020} ist für eine präzise Approximation die Anzahl von acht bis zehn Elementen je Wellenlänge erforderlich. Aufgrund der definierten Gesamtausdehnung des ITM-Teilsystems zu $B_x = B_y = 128\ \mathrm{m}$ und $N_x = N_y = 2^{9}$ Stützstellen besitzen die Elemente eine Länge von $\mathrm{d}x = \mathrm{d}y = 0,25\ \mathrm{m}$. Die resultierenden Werte für die Wellenlänge der Rayleigh-Wellen $\lambda_r$ sind in der Tabelle (\ref{tab:Material_Parameter}) aufgeführt. Die Wellenlänge beträgt demnach $\lambda_r = 2,17\ \mathrm{m}$. Es liegen folglich für die weicheren Materialparameter acht Elemente je Rayleigh-Wellenlänge vor. 

Im Vergleich beträgt die Rayleigh-Wellenlänge für die steiferen Materialparameter \texttt{stiff\_v01} $\lambda_r = 7,67\ \mathrm{m}$ was bei gleicher Diskretisierung zu dreißig Elementen pro Rayleigh-Wellenlänge führt. Für diesen steiferen Boden ist somit eine präzisere Approximation möglich.

Um eine exaktere Approximation zu erreichen, ist es erforderlich, entweder ein steiferes Material zu verwenden, die Frequenz $f$ zu reduzieren oder eine feinere Diskretisierung zu implementieren. Diese Einflussfaktoren sind allerdings unabhängig von der entwickelten Implementierung dreieckiger, finiter Elemente.




%Eine mögliche Erklärung für die beobachteten Abweichungen könnten die höheren Verschiebungen aufgrund der direkten Belatung im ausgewerteten FE-Bereich $\Omega$, sowie das weichere Bodenverhalten sein.
%Die Approximation höherer Verschiebungen durch diskretisierte finite Elemente und deren Integration ist demnach wengier präzise. 
%Im Rahmen der vorliegenden Arbeit wird auf diese Beobachtung und einer möglichen Erklärung jedoch nicht weiter eingegangen.

Dennoch weisen die Graphen der Abbildungen (\ref{fig:c2_soft}) und die Werte der Tabelle (\ref{tab:Fehlermessung_c2_soft}) auf eine gute Approximation hin, sodass die entwickelte Implementierung die Referenzlösungen in guter Näherung beschreiben kann.

Die in diesem Kapitel durchgeführten Verifizierungen zeigen, dass die Referenzlösungen von \cite{Freisinger2022} auch bei einer Belastung im FE-Bereich $\Omega$ gut angenähert werden können. 
Eine gröbere Diskretisierung bei weicheren Böden kann zu größeren Abweichungen führen, wobei der Tanimoto-Koeffizient von $T = 0,99$ dennoch eine gute Annäherung darstellt.
Der steifere Boden hingegen, weist exaktere Ergebnisse verglichen mit den Referenzlösnugen auf, wobei Tanimoto Koeffizienten von $T = 1,00$ erzielt werden konnten.





\section{Fall 3: Graben mit Belastung auf der Halbraumoberfläche}
\label{sec:verification_c3}

In einem dritten Verifizierungsfall wird der Tunnelmittelpunkt auf die Halbraumoberfläche $\Lambda$ verschoben, sodass dieser zu einem offenen Graben wird. Dabei wird lediglich ein halber Zylinder als FE-Bereich $\Omega$ definiert, wie in den Abbildungen (\ref{fig:c3_systeme}) dargestellt.

Zudem erfolgt der dritte Verifizierungsfall mit den den Materialparametern \texttt{stiff\_v01}.
\begin{figure}[H]
	\centering
	\begin{subfigure}[t]{0.49\linewidth}
		\centering
		\includesvg[height=4.0cm]{svg/verification_c3_system_hs_cyl}
	\end{subfigure}\hfill
	\begin{subfigure}[t]{0.49\linewidth}
		\centering
		\includesvg[height=4.0cm]{svg/verification_c3_system_hs}
	\end{subfigure}
	\caption{Gewählte Systemparameter zur Verifikation mit viereckigen Elementen (links) und mit dem homogenen Halbraum (rechts) im dritten Fall - basiert auf \cite{Freisinger2022}.}
	\label{fig:c3_systeme}
\end{figure}
Aufgrund des Sonderfalls werden die Systemparameter von den vorherigen Systemen abweichend gewählt.
Der Radius wurde auf $R = 4\ \mathrm{m}$ gesetzt und der Zylindermittelpunkt in den Koordinatenursprung gelegt. Da es sich um einen halben Zylinder an der Oberfläche handelt, liegt der Zylindermittelpunkt auf der Halbraumoberfläche $\Lambda$ bei $z = 0\ \mathrm{m}$.
Des weiteren wird die Gesamtausdehnung des ITM-Teilsystems sowie die gewählten Stützstellen erneut zu $B_x = B_y = 128\ \mathrm{m}$ und $N_x = N_y = 2^{9}$ gewählt.
Zudem wird eine Anzahl von $N_{\varphi} = 64$ umlaufenden Knoten auf der Zylinderkopplungsfläche $\Gamma$ definiert.

Die Belastung wurde in diesem Verifizierungsfall auf die Halbraumoberfläche $\Lambda$ aufgebracht, deren Breite $b_x = b_y = 2\ \mathrm{m}$ und deren Amplitude $1 \mathrm{N}/\mathrm{m}^{2}$ beträgt. Die Belastung wurde $y_L = 6\ \mathrm{m}$ vom Koordinatenursprung entfernt aufgebracht, sodass die quadratische Belastung vollständig auf der Halbraumoberfläche $\Lambda$ verortet ist. 


Da der Zylindermittelpunkt an der Halbraumoberfläche $\Lambda$ liegt, werden in den Abbildungen (\ref{fig:c3}) die vertikalen Verschiebungen $|u_z(y)|$ bei $z = 0\ \mathrm{m}$ betrachtet.
\begin{figure}[H]
	\centering
	\begin{subfigure}{0.49\linewidth}
		\centering
		\includesvg[width=\linewidth]{svg/verification_c3_left_svg_tex}
	\end{subfigure}\hfill
	\begin{subfigure}{0.49\linewidth}
		\centering
		\includesvg[width=\linewidth]{svg/verification_c3_homog_svg_tex}
	\end{subfigure}
	\caption{Abweichung $|\bar{u}_z(y)|$ der dreieckigen Elemente (\legThree) zu den Referenzlösungen (\legFour) viereckiges FE-Netz (links) und homogener Halbraum (rechts) im dritten Verifizierungsfall.}
	\label{fig:c3}
\end{figure}
\begin{table}[htb]
	\centering
	\normalsize
	{\renewcommand{\arraystretch}{1.15}
		\begin{tabular}{ccccc}
			\firsthline
			Referenzlösung & Maximaler Wert & Maximale Abweichung & Tanimoto-Koeffizient $T$ \\\hline
			FE-Netz & $ 1,71\cdot10^{-9}$ & $1,30\cdot10^{-11}$ & 0,99 \\
			homogener Halbraum & $1,72\cdot10^{-9}$ & $1,10\cdot10^{-11}$ & 0,99 \\\lasthline
	\end{tabular}}
	\caption{Abweichugnsmessung von $|\bar{u}_z(y)|$ im dritten Verifizierungsfall.}
	\label{tab:Fehlermessung_c3}
\end{table}
Die resultierenden Abweichungen der vertikalen Verschiebungen $|\bar{u}_z(y)|$ in den Abbildungen (\ref{fig:c3}) zeigen gute Approximationen an die Referenzlösungen. Die in Tabelle (\ref{tab:Fehlermessung_c3}) spiegeln diese Interpretation ebenfalls wider, da sowohl die maximalen Abweichungen als auch die Tanimoto-Koeffizienten von $T = 0,99$ für eine gute Näherung an die Referenzlösungen sprechen.

%Die in Graph (\ref{fig:c3_a}) dargestellten Verläufe der vertikalen Verschiebung $|\bar{u}_z(y)|$ der FE-Referenzlösung sowie der entwickelten Lösungen zeigen, dass die trianguläre Implementierung auch im Sonderfall des offenen Grabens eine sehr gute Approximation aufweist.
%Das spiegelt sich in den Werten der Tabelle (\ref{tab:Fehlermessung_c3}) wider, da gegenüber der FE-Referenzlösung ein Tanimoto Koeffizient von $T = 0,99$ die gute Approximation unterstreicht. 

%Ein ebenso positives Fazit lässt sich im Vergleich zur homogenen Referenzlösung im Graph (\ref{fig:c3_b}) erkennen, da sich auch hier die entwickelte Lösung ebenfalls mit der Referenzlösung deckt. Diese Aussage wird durch den Tanimoto Koeffizient $T = 0,99$ bestätigt.

Die in diesem Kapitel durchgeführte Verifizierung zeigt, dass die Referenzlösungen von \cite{Freisinger2022} auch für den Sonderfall eines Grabens mit Belastung auf der Halbraumoberfläche $\Lambda$ gut angenähert werden können. Die entwickelte Implementation dreieckiger, finiter Elemente ist somit auch im dritten Verifizierungsfall als präzise zu bewerten.












	\chapter{Fazit}
\label{cha:Schluss}

In der vorliegenden Arbeit erfolgt die Weiterentwicklung der bestehenden Implementierung eines gekoppelten ITM-FEM-Ansatzes aus den Dissertationen von \cite{Hackenberg2016} und \cite{Freisinger2022}. 
Auf Basis der bestehenden Implementierung wird das FE-Netz von viereckigen auf dreieckige Elemente umgestellt. Der herangezogene Ansatz beruht dabei nicht auf nativen Dreiecken, sondern auf einer Degeneration viereckiger in dreieckige Elementen. 

Im Rahmen der Verifikation der entwickelten Implementierung wurden validierte Referenzlösungen herangezogen. 
Einerseits werden Abweichungen zur bestehenden Implementierung mit viereckigen Elementen verglichen, andererseits wird die semi-analytische Lösung eines homogenen Halbraums herangezogen.
In den verschiedenen Verifikationsfällen wurden gute bis sehr gute Annäherungen an die Referenzlösungen festgestellt, weshalb die entwickelte Implementierung als gelungen bewertet werden kann. 
Die Verwendung eines alternativen FE-Netzes ist demnach eine mögliche Option, um komplexere Geometrien, wie etwa einen Tunnel, zu realisieren, ohne dabei signifikante Einbußen an Präzision zu verzeichnen.

Trotz der überwiegend positiven Ergebnisse sind auch Limitationen und kritische Aspekte der entwickelten Implementierung zu berücksichtigen. 
Ein besonderer Fokus sollte auf den gewählten Degenerationsansatz gelegt werden. Zur numerischen Integration wird eine Gauß'sche Quadratur verwendet, die ursprünglich für viereckige Elemente konzipiert ist. 
Das kann dazu führen, dass Gaußpunkte außerhalb des dreieckigen Elements liegen, was wiederum zu Problemen im Integrationsprozess führen könnte. 
In der vorliegenden Arbeit wird diese potenzielle Fehlerquelle in Kauf genommen und nicht weiter behandelt. 
Die in den durchgeführten Verifizierungsfällen eingesetzte numerische Integration zeigt keine offensichtlichen Abweichungen.
Es besteht jedoch die Möglichkeit, dass ein solcher Ansatz bei komplexeren Geometrien oder feineren Netzen numerische Instabilitäten verursacht.
Die Übertragung einer Integrationsmethode von viereckigen auf dreieckige Elemente stellt somit eine potenzielle Fehlerquelle dar, die in zukünftigen Arbeiten oder Implementierungen vermieden und korrigiert werden sollte.

Auf Basis der entwickelten Implementierung ergeben sich verschiedene Perspektiven für die Weiterentwicklung des gekoppelten ITM-FEM-Ansatzes. 
Es empfiehlt sich, vorrangig eine Überarbeitung der numerischen Integrationsmethode vorzunehmen oder alternativ die Generierung dreieckiger Elemente durch native Dreiecke zu ermöglichen. 
Darüber hinaus könnte eine lokale Verfeinerung des Netzes in relevanten Bereichen zu einer Verbesserung der Auflösung führen, ohne dass der Rechenaufwand für das gesamte Modell unverhältnismäßig erhöht werden müsste. 
Anschließend besteht die Möglichkeit, parametrische Untersuchungen durchzuführen, um den Einfluss lokaler Netzverfeinerungen zu ermitteln.
Auch eine Erweiterung des Modells um höherwertige dreieckige Elemente ist denkbar, um die Genauigkeit weiter zu steigern.
In diesem Ansatz werden die dreieckigen Elemente nicht linear durch drei Knoten beschrieben, sondern parabolisch durch sechs oder kubisch durch neun Knoten je Element. Dieser Ansatz ist in den Arbeiten von cite{Zienkiewicz2013} und \cite{Klein2003} beschrieben. 

 Zusammenfassend lässt sich festhalten, dass die vorliegende Arbeit eine Grundlage für weiterführende Arbeiten im Bereich der Wellenausbreitung in einem Halbraum mit zylindrischem Hohlraum schafft. 
 Die gewonnenen Erkenntnisse und die entwickelte Implementierung können aufgegriffen und weiterentwickelt werden. 
 Dies kann dazu dienen, theoretische Aspekte der Implementierung zu verfeinern oder in praxisorientierten Parameterstudien das Einsatzpotenzial der entwickelten Implementierung zu analysieren.
	\appendix                     % Anhang
	\chapter{Anhang}
\label{cha:anhang}

\section{Verwendete Software}
\label{cha:software}

\subsection{Maple\texttrademark}
\label{app:maple}
\emph{Maple\texttrademark} ist ein kommerzielles Mathematik-Paket der Firma Maplesoft. Seine Stärke liegt darin, dass es mathematische Ausdrücke algebraisch verarbeiten kann. Im vorliegenden Fall werden auf diese Weise Ableitungen und Integrale berechnet, ohne auf numerische Aspekte Rücksicht zu nehmen. Dies ermöglicht einen kurzen und trotzdem voll funktionalen Code. \emph{Maple\texttrademark} wurde in den Version 12 und 14 verwendet.\\
Internet: \url{http://www.maplesoft.com/}

\subsection{gnuplot}
\emph{gnuplot} wurde verwendet, um für diese Arbeit Diagramme und Plots anzufertigen. Es ist ein interaktives, Kommandozeilen-gesteuertes Programm, das auf fast allen gängigen Betriebssystemen lauffähig ist. Es ist in der Lage, sowohl 2D als auch 3D Plots zu erstellen und diese in verschiedenen Dateiformaten zu exportieren. \emph{gnuplot} ist ein Open Source Projekt und wird immer noch weiterentwickelt. Hier wurde die zum Zeitpunkt der Niederschrift aktuelle Version 4.2.3 verwendet.\\
Internet: \url{http://www.gnuplot.info/}
die Anzahl der Knoten auf 1000 beschränkt ist. Glücklicher Weise ist die in der REM eine annehmbare Menge, mit der sich durchaus einige Problemen behandeln lassen.

\clearpage

\section{Maple\texttrademark -Programm für ein 2D-Problem}
\label{sec:maplecode}

%
%\lstinputlisting[style=maple,
  %label=lst:maplecode,
  %firstnumber=1,
  %caption={[Maple\texttrademark -Programm für ein 2D-Problem] \texttt{maple2D.mw}}
  %] {code/maplebsp.txt}
  

%\section{{\texttt{C++}} code of dragLift.C}
%\label{sec:draglift}
%
%\lstinputlisting[style=cppcode,
%  label=draglift,
%  firstnumber=1,
%  ] {code/dragLift.C}
%  
%\section{{\texttt{C++}} code of stlIcoFoam.C}
%\label{sec:stlicofoam}
%
%\lstinputlisting[style=cppcode,
%  label=stlicofoam,
%  firstnumber=1,
%  ] {code/stlIcoFoam.C}

%%% Local Variables: 
%%% mode: latex
%%% TeX-master: "main"
%%% End: 
    
	
	%\backmatter
	%\begin{spacing}{1.0}          % Verzeichnisse werden mit einzeiligem Abstand gesetzt
	%\bibliographystyle{apalike}   % oder abbrv, alpha, plainnat, unsrt, ...
	% \bibliography{Literaturverzeichnis} % ohne .bib-Endung
	%\printbibliography
	%\begin{spacing}{1.0}          % Verzeichnisse werden mit einzeiligem Abstand gesetzt
	%\inputencoding{latin2}
	%\bibliography{Literaturverzeichnis}
	%\inputencoding{utf8}
	%\end{spacing}
	%\end{spacing}
	%\backmatter
	%\begin{spacing}{1.0}          % Verzeichnisse werden mit einzeiligem Abstand gesetzt
	%	\inputencoding{latin2}
	%	\bibliography{Literaturverzeichnis}
	%	\inputencoding{utf8}
	%\end{spacing}
	%\printbibliography[heading=bibintoc,title={Literaturverzeichnis}]
	
	\backmatter
	\begin{spacing}{1.0}          % Verzeichnisse werden mit einzeiligem Abstand gesetzt
		\inputencoding{utf8}
		\bibliography{a_main_thesis}
		\inputencoding{utf8}
	\end{spacing}
	
\end{document}
